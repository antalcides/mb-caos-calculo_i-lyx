%%%%%%%%%%%%% \breakbox
\def\plural@title{}
\def\typebox#1{%
\expandafter\def\csname spacebefore\endcsname{\expandafter\csname spacebefore#1\endcsname}%
\expandafter\def\csname stylebox\endcsname{\expandafter\csname#1@box\endcsname}%
\expandafter\def\csname titlefont\endcsname{\expandafter\csname#1title@font\endcsname}%
\expandafter\def\csname plural\endcsname{\expandafter\csname plural@#1\endcsname}%
\ifnum\thetype@box=0\setlength{\spacebottom}{\expandafter\csname spacebottom#1\endcsname}\fi
\expandafter\def\csname fillbg\endcsname{#1@bg@color}
\expandafter\def\csname titlecolor\endcsname{#1title@color}
\expandafter\def\csname titlebgcolor\endcsname{#1title@bg@color}
}
\newcounter{break@box}
\newcounter{type@box}
\newcommand*{\breakbox}
{%
	
		\ifnum\thetype@box=0 %TODO para "rem" o "meth" o "new"
		\end{minipage}
			\end{lrbox}
			\vspace*{\spacebefore}
			\stepcounter{break@box}
			\begin{tikzpicture}
			\node (texte) {\usebox{\stylebox}\vspace*{\spacebottom}};
			\fill[fill=shadow@color] ($(texte.south west)+(1pt,-1pt)$) -- ($(texte.north west)+(1pt,-1pt)$) -- ($(texte.north east)+(1pt,-1pt)$) -- ($(texte.south east)+(0,6.5pt)+(1pt,-2pt)$) .. controls ($(texte.south east)+(-2pt,2pt)+(1pt,-1pt)$) and ($(texte.south east)+(-3pt,1pt)+(1pt,-1pt)$) .. ($(texte.south east)+(-6.5pt,0)+(1pt,-1pt)$) -- cycle;
			\fill[fill=\fillbg] (texte.south west) -- (texte.north west) -- (texte.north east) -- ($(texte.south east)+(0,5pt)$) -- ($(texte.south east)+(-5pt,0)$) -- cycle;
			\fill[fill=shadow@color!50!black] ($(texte.south east)+(0,5pt)$) .. controls ($(texte.south east)+(-2pt,2pt)$) and ($(texte.south east)+(-3pt,1pt)$) .. ($(texte.south east)+(-5pt,0)$) .. controls ($(texte.south east)+(-4.5pt,2.5pt)$) .. ($(texte.south east)+(-5pt,5pt)$) .. controls ($(texte.south east)+(-2.5pt,4.5pt)$) .. ($(texte.south east)+(0,5pt)$);
			\node at(texte) {\usebox{\stylebox}};
			\node[rounded corners=3pt,fill=\fillbg!75,draw=\titlecolor] at ($(texte.south east)-(1,0)$) {...};
			\node[above right,fill=\titlebgcolor,text=\titlecolor] at (texte.north west) {\titlefont \titletext\plural};
			\end{tikzpicture}\newline
			\begin{lrbox}{\stylebox}
			\begin{minipage}{\linewidth}
		\fi
		\ifnum\thetype@box=1 %TODO para "def" o "prop"
			\end{minipage}
			\end{lrbox}
			\vspace*{\spacebefore}
			\stepcounter{break@box}
			\begin{tikzpicture}
			\node[fill=\fillbg] (texte) {\usebox{\stylebox}};
			%\draw[very thick,draw=\titlebgcolor] (texte.north west) -- (texte.south west);
			\node[rounded corners=3pt,fill=\fillbg!75,draw=\titlebgcolor] at ($(texte.south east)-(1,0)$) {...};
			%\node[fill=deftitle@bg@color,text=deftitle@color,below left,draw=deftitle@bg@color,very thick,text width={\dimexpr\marginleft-2mm},align=center] at (texte.north west) {\deftitle@font \titletext\plural@def \ifnum\thebreak@box>0\par(continuaci\'on)\fi};
			\node[fill=\titlebgcolor,text=\titlecolor,below left,draw=\titlebgcolor,very thick,text width={\dimexpr\marginleft-2mm},align=center] at ($(texte.north west)+(1pt,0)$) {\titlefont \titletext\plural};
			\end{tikzpicture}\newline
			\begin{lrbox}{\stylebox}
			\begin{minipage}{\dimexpr\linewidth-\marginleft}
		\fi
		\ifnum\thetype@box=2 %TODO para "dem" 
			\vskip-1em\hfill{\demtitle@font(...)}
			\end{minipage}
			\end{lrbox}
			\vspace*{\spacebefore}
			\begin{tikzpicture}
			\node[fill=dem@bg@color] (texte) {\usebox{\dem@box}};
			\ifnum\indic=1
				\draw[thick,draw=dem@border@color,decorate, decoration={random steps, segment length=3pt,amplitude=1pt}] (texte.north west) -- (texte.south west);
			\else
				\ifnum\indic=2
					\draw[thick,draw=dem@border@color,decorate, decoration={random steps, segment length=3pt,amplitude=1pt}] (texte.north west) -- (texte.south west) -- (texte.south east);
				\else
					\ifnum\indic=3
						\draw[thick,draw=dem@border@color,decorate, decoration={random steps, segment length=3pt,amplitude=1pt}] (texte.north west) -- (texte.south west);
						\draw[thick,draw=dem@border@color,decorate, decoration={random steps, segment length=3pt,amplitude=1pt}] (texte.north east) -- (texte.south east);
					\else
						\ifnum\indic=4
							\draw[thick,draw=dem@border@color,decorate, decoration={random steps, segment length=3pt,amplitude=1pt}] (texte.north east) -- (texte.north west) -- (texte.south west);
						\else
							\ifnum\indic=5
								\draw[thick,draw=dem@border@color,decorate, decoration={random steps, segment length=3pt,amplitude=1pt}] (texte.north west) -- (texte.south west) -- (texte.south east) -- (texte.north east) -- cycle;
							\fi
						\fi
					\fi
				\fi
			\fi
			\node[below left,minimum width=0.5\marginleft] at (texte.north west) {\phantom{-}};
			\end{tikzpicture}\newline
			\begin{lrbox}{\dem@box}
			\begin{minipage}{\dimexpr\linewidth-0.5\marginleft}
			{\demtitle@font \titletext~(continuaci\'on)}\par\medskip%
		\fi
}

%%%%%%%%%%%%% Definición de una nueva caja

\newsavebox{\new@box}
\newcommand{\DefineNewBoxLikeRem}[5] % #1 : nombre; #2 : título; #3 : color principal ; #4 : color del textot en los items ; #5 : fuente de los  items
{%
	\newenvironment{#1}
	{%
		\setcounter{break@box}{0}
		\setcounter{type@box}{0}
		% Redéfinition de \breakbox
		\renewcommand*{\breakbox}
		{%
			\end{minipage}
				\end{lrbox}
				\vspace*{\spacebeforenew}
				\stepcounter{break@box}
				\begin{tikzpicture}
				\node (texte) {\usebox{\new@box}\vspace*{\spacebottomnew}};
				\fill[fill=shadow@color] ($(texte.south west)+(1pt,-1pt)$) -- ($(texte.north west)+(1pt,-1pt)$) -- ($(texte.north east)+(1pt,-1pt)$) -- ($(texte.south east)+(0,6.5pt)+(1pt,-2pt)$) .. controls ($(texte.south east)+(-2pt,2pt)+(1pt,-1pt)$) and ($(texte.south east)+(-3pt,1pt)+(1pt,-1pt)$) .. ($(texte.south east)+(-6.5pt,0)+(1pt,-1pt)$) -- cycle;
				\fill[fill=#3!20] (texte.south west) -- (texte.north west) -- (texte.north east) -- ($(texte.south east)+(0,5pt)$) -- ($(texte.south east)+(-5pt,0)$) -- cycle;
				\fill[fill=shadow@color!50!black] ($(texte.south east)+(0,5pt)$) .. controls ($(texte.south east)+(-2pt,2pt)$) and ($(texte.south east)+(-3pt,1pt)$) .. ($(texte.south east)+(-5pt,0)$) .. controls ($(texte.south east)+(-4.5pt,2.5pt)$) .. ($(texte.south east)+(-5pt,5pt)$) .. controls ($(texte.south east)+(-2.5pt,4.5pt)$) .. ($(texte.south east)+(0,5pt)$);
				\node at(texte) {\usebox{\new@box}};
				\node[rounded corners=3pt,fill=#3!20,draw=#3,text=#4] at ($(texte.south east)-(1,0)$) {...};
				\node[above right,fill=#3,text=#4] at (texte.north west) {\fontfamily{#5}\selectfont#2};
				\end{tikzpicture}\newline
				\begin{lrbox}{\new@box}
				\begin{minipage}{\linewidth}
		}
		\def\titletext{#2} % título de la caja
		\itemclass{#3}{\fontfamily{#5}\selectfont}
		\begin{lrbox}{\new@box}
		\begin{minipage}{\linewidth}
	}
	{%
		\end{minipage}
		\end{lrbox}
		\vspace*{\spacebeforenew}
		\begin{tikzpicture}
		\node (texte) {\usebox{\new@box}\vspace*{\spacebottomnew}};
		\fill[fill=shadow@color] ($(texte.south west)+(1pt,-1pt)$) -- ($(texte.north west)+(1pt,-1pt)$) -- ($(texte.north east)+(1pt,-1pt)$) -- ($(texte.south east)+(0,6.5pt)+(1pt,-2pt)$) .. controls ($(texte.south east)+(-2pt,2pt)+(1pt,-1pt)$) and ($(texte.south east)+(-3pt,1pt)+(1pt,-1pt)$) .. ($(texte.south east)+(-6.5pt,0)+(1pt,-1pt)$) -- cycle;
		\fill[fill=#3!20] (texte.south west) -- (texte.north west) -- (texte.north east) -- ($(texte.south east)+(0,5pt)$) -- ($(texte.south east)+(-5pt,0)$) -- cycle;
		\fill[fill=shadow@color!50!black] ($(texte.south east)+(0,5pt)$) .. controls ($(texte.south east)+(-2pt,2pt)$) and ($(texte.south east)+(-3pt,1pt)$) .. ($(texte.south east)+(-5pt,0)$) .. controls ($(texte.south east)+(-4.5pt,2.5pt)$) .. ($(texte.south east)+(-5pt,5pt)$) .. controls ($(texte.south east)+(-2.5pt,4.5pt)$) .. ($(texte.south east)+(0,5pt)$);
		\node at(texte) {\usebox{\new@box}};
		\node[text width={\dimexpr\marginleft+2mm},above right,fill=#3,text=#4] at (texte.north west) {\fontfamily{#5}\selectfont#2\ifnum\thebreak@box>0 ~(Continuaci\'on)\fi};
		\end{tikzpicture}
	}
%
}	

\newcommand{\DefineNewBoxLikeDef}[5] % #1 : nombre; #2 : título; #3 : color principal ; #4 : color del textot en los items ; #5 : fuente de los  items
{%
	\newenvironment{#1}
	{%
		\setcounter{break@box}{0}
		\setcounter{type@box}{1}
		\renewcommand*{\breakbox}
		{%
			\end{minipage}
			\end{lrbox}
			\vspace*{\spacebeforenew}
			\stepcounter{break@box}
			\begin{tikzpicture}
			\node[fill=#3!20] (texte) {\usebox{\new@box}};
			\node[rounded corners=3pt,fill=#3!20,draw=#3] at ($(texte.south east)-(1,0)$) {...};
			\node[fill=#3,text=#4,below left,draw=#3,very thick,text width={\dimexpr\marginleft-2mm},align=center] at ($(texte.north west)+(1pt,0)$) {\fontfamily{#5}\selectfont#2};
			\end{tikzpicture}\newline
			\begin{lrbox}{\new@box}
			\begin{minipage}{\dimexpr\linewidth-\marginleft}
		}
		\def\titletext{#2}
		\itemclass{#3}{\fontfamily{#5}\selectfont}
		\begin{lrbox}{\new@box}
		\begin{minipage}{\dimexpr\linewidth-\marginleft}
	}
	{%
		\end{minipage}
		\end{lrbox}
		\vspace*{\spacebeforenew}
		\begin{tikzpicture}
		\node[fill=#3!20] (texte) {\usebox{\new@box}};
		\node[fill=#3,text=#4,below left,draw=#3,very thick,text width={\dimexpr\marginleft+2mm},align=center] at ($(texte.north west)+(1pt,0)$) {\fontfamily{#5}\selectfont#2\ifnum\thebreak@box>0\par(Continuaci\'on)\fi};
		\end{tikzpicture}
	}
}
%TODO %%%%%%%%%%%% Entorno "remark"

\newsavebox{\rem@box}
\newenvironment{remark}[1][\'on]
{%
\setcounter{break@box}{0}
\setcounter{type@box}{0}
\typebox{rem}
\def\titletext{Observaci}
\ifx#1\@empty\xdef\plural@rem{}\else\xdef\plural@rem{#1}\fi
\itemclass{remtitle@bg@color}{\remtitle@font}
\begin{lrbox}{\rem@box}
\begin{minipage}{\linewidth}
}
{%
\end{minipage}
\end{lrbox}
\vspace*{\spacebeforerem}
\begin{tikzpicture}
\node (texte) {\usebox{\rem@box}\vspace*{\spacebottomrem}};
\fill[fill=shadow@color] ($(texte.south west)+(1pt,-1pt)$) -- ($(texte.north west)+(1pt,-1pt)$) -- ($(texte.north east)+(1pt,-1pt)$) -- ($(texte.south east)+(0,6.5pt)+(1pt,-2pt)$) .. controls ($(texte.south east)+(-2pt,2pt)+(1pt,-1pt)$) and ($(texte.south east)+(-3pt,1pt)+(1pt,-1pt)$) .. ($(texte.south east)+(-6.5pt,0)+(1pt,-1pt)$) -- cycle;
\fill[fill=rem@bg@color] (texte.south west) -- (texte.north west) -- (texte.north east) -- ($(texte.south east)+(0,5pt)$) -- ($(texte.south east)+(-5pt,0)$) -- cycle;
\fill[fill=shadow@color!50!black] ($(texte.south east)+(0,5pt)$) .. controls ($(texte.south east)+(-2pt,2pt)$) and ($(texte.south east)+(-3pt,1pt)$) .. ($(texte.south east)+(-5pt,0)$) .. controls ($(texte.south east)+(-4.5pt,2.5pt)$) .. ($(texte.south east)+(-5pt,5pt)$) .. controls ($(texte.south east)+(-2.5pt,4.5pt)$) .. ($(texte.south east)+(0,5pt)$);
\node at(texte) {\usebox{\rem@box}};
\node[above right,fill=remtitle@bg@color,text=remtitle@color] at (texte.north west) {\remtitle@font \titletext\plural@rem \ifnum\thebreak@box>0 ~(continuaci\'on)\fi};
\end{tikzpicture}
}

%TODO %%%%%%%%%%%% Entorno"método"

\newsavebox{\meth@box}
\newenvironment{metodo}[1][]
{%
\setcounter{break@box}{0}
\setcounter{type@box}{0}
\typebox{meth}
\def\titletext{M\'etodo}
\ifx#1\@empty\xdef\plural@meth{}\else\xdef\plural@meth{#1}\fi
\itemclass{methtitle@bg@color}{\methtitle@font}
\begin{lrbox}{\meth@box}
\begin{minipage}{\linewidth}
}
{%
\end{minipage}
\end{lrbox}
\vspace*{\spacebeforemeth}
\begin{tikzpicture}
\node (texte) {\usebox{\meth@box}\vspace*{\spacebottommeth}};
\fill[fill=shadow@color] ($(texte.south west)+(1pt,-1pt)$) -- ($(texte.north west)+(1pt,-1pt)$) -- ($(texte.north east)+(1pt,-1pt)$) -- ($(texte.south east)+(0,6.5pt)+(1pt,-2pt)$) .. controls ($(texte.south east)+(-2pt,2pt)+(1pt,-1pt)$) and ($(texte.south east)+(-3pt,1pt)+(1pt,-1pt)$) .. ($(texte.south east)+(-6.5pt,0)+(1pt,-1pt)$) -- cycle;
\fill[fill=meth@bg@color] (texte.south west) -- (texte.north west) -- (texte.north east) -- ($(texte.south east)+(0,5pt)$) -- ($(texte.south east)+(-5pt,0)$) -- cycle;
\fill[fill=shadow@color!50!black] ($(texte.south east)+(0,5pt)$) .. controls ($(texte.south east)+(-2pt,2pt)$) and ($(texte.south east)+(-3pt,1pt)$) .. ($(texte.south east)+(-5pt,0)$) .. controls ($(texte.south east)+(-4.5pt,2.5pt)$) .. ($(texte.south east)+(-5pt,5pt)$) .. controls ($(texte.south east)+(-2.5pt,4.5pt)$) .. ($(texte.south east)+(0,5pt)$);
\node at(texte) {\usebox{\meth@box}};
\node[above right,fill=methtitle@bg@color,text=methtitle@color] at (texte.north west) {\methtitle@font \titletext\plural@meth \ifnum\thebreak@box>0 ~(continuaci\'on)\fi};
\end{tikzpicture}
}

%TODO %%%%%%%%%% Entorno "notación"

\newsavebox{\nota@box}
\newenvironment{notation}[1][\'on]
{%
	\setcounter{break@box}{0}
	\setcounter{type@box}{1}
	\typebox{nota}
	\def\titletext{Notaci}
	\ifx#1\@empty\xdef\plural@nota{}\else\xdef\plural@nota{#1}\fi
	\itemclass{deftitle@bg@color}{\deftitle@font}
	\begin{lrbox}{\nota@box}
	\begin{minipage}{\dimexpr\linewidth-\marginleft}
}
{%
	\end{minipage}
	\end{lrbox}
	\vspace*{\spacebeforedef}
	\begin{tikzpicture}
	\node[fill=def@bg@color] (texte) {\usebox{\nota@box}};
	\node[fill=deftitle@bg@color,text=deftitle@color,below left,draw=deftitle@bg@color,very thick,text width={\dimexpr\marginleft-2mm},align=center] at ($(texte.north west)+(1pt,0)$) {\deftitle@font \titletext\plural@nota \ifnum\thebreak@box>0\par(continuaci\'on)\fi};
	\end{tikzpicture}
}
%TODO %%%%%%%%%% Entorno "definition"

\newsavebox{\def@box}
\newcounter{def}[chapter]
\newenvironment{definition}[1][\'on]
{%
	\setcounter{break@box}{0}
	\setcounter{type@box}{1}
	\stepcounter{def}
	\typebox{def}
	\def\titletext{Definici}
	\ifx#1\@empty\xdef\plural@def{}\else\xdef\plural@def{#1}\fi
	\itemclass{deftitle@bg@color}{\deftitle@font}
	\begin{lrbox}{\def@box}
	\begin{minipage}{\dimexpr\linewidth-1.4\marginleft}
}
{%
	\end{minipage}
	\end{lrbox}
	\vspace*{\spacebeforedef}
	\begin{tikzpicture}
	\node[fill=def@bg@color] (texte) {\usebox{\def@box}};
	\node[fill=deftitle@bg@color,text=deftitle@color,below left,draw=deftitle@bg@color,very thick,text width={\dimexpr\marginleft+5mm},align=center] at ($(texte.north west)+(1pt,0)$) {\deftitle@font\thechapter.\thedef\ \titletext\plural@def \ifnum\thebreak@box>0\par(continuaci\'on)\fi};
	\end{tikzpicture}
}
%TODO %%%%%%%%%% Entorno "proposition"

\newsavebox{\propo@box}
\newcounter{propo}[chapter]
\newenvironment{proposition}[1][\'on]
{%
	\setcounter{break@box}{0}
	\setcounter{type@box}{1}
	\stepcounter{propo}
	\typebox{propo}
	\def\titletext{Proposici}
	\ifx#1\@empty\xdef\plural@propo{}\else\xdef\plural@propo{#1}\fi
	\itemclass{deftitle@bg@color}{\deftitle@font}
	\begin{lrbox}{\propo@box}
	\begin{minipage}{\dimexpr\linewidth-\marginleft}
}
{%
	\end{minipage}
	\end{lrbox}
	\vspace*{\spacebeforedef}
	\begin{tikzpicture}
	\node[fill=def@bg@color] (texte) {\usebox{\propo@box}};
	\node[fill=deftitle@bg@color,text=deftitle@color,below left,draw=deftitle@bg@color,very thick,text width={\dimexpr\marginleft-2mm},align=center] at ($(texte.north west)+(1pt,0)$) {\deftitle@font\thechapter.\thepropo\ \titletext\plural@prop \ifnum\thebreak@box>0\par(continuaci\'on)\fi};
	\end{tikzpicture}
}

%TODO %%%%%%%%%% Entorno "propiedad"
\newcounter{propi}[chapter]
\newsavebox{\prop@box}
\newenvironment{propiedad}[1][]
{%
	\setcounter{break@box}{0}
	\setcounter{type@box}{1}
	\stepcounter{propi}
	\typebox{prop}
	\def\titletext{Propiedad}
	\ifx#1\@empty\xdef\plural@prop{}\else\xdef\plural@prop{#1}\fi
	\itemclass{proptitle@bg@color}{\proptitle@font}
	\begin{lrbox}{\prop@box}
	\begin{minipage}{\dimexpr\linewidth-\marginleft}
}
{%
	\end{minipage}
	\end{lrbox}
	\vspace*{\spacebeforeprop}
	\begin{tikzpicture}
	\node[fill=prop@bg@color] (texte) {\usebox{\prop@box}};
	\node[fill=proptitle@bg@color,text=proptitle@color,below left,draw=proptitle@bg@color,very thick,text width={\dimexpr\marginleft+3mm},align=center] at ($(texte.north west)+(1pt,0)$) {\proptitle@font\thechapter.\thepropi\ \titletext\plural@prop \ifnum\thebreak@box>0\par(Continuaci\'on)\fi};
	\end{tikzpicture}
}

%TODO %%%%%%%%%% Entorno "theorem"

\newsavebox{\thm@box}
\newcounter{thm}[chapter]
\newenvironment{theorem}[1][]
{%
    \stepcounter{thm}
	\setcounter{break@box}{0}
	\setcounter{type@box}{1}
	\typebox{thm}
	\def\titletext{Teorema}
	\ifx#1\@empty\xdef\plural@thm{}\else\xdef\plural@thm{#1}\fi
	\itemclass{thmtitle@bg@color}{\thmtitle@font}
	\begin{lrbox}{\thm@box}
	\begin{minipage}{\dimexpr\linewidth-1.3\marginleft}
}
{%
	\end{minipage}
	\end{lrbox}
	\vspace*{\spacebeforethm}
	\begin{tikzpicture}
	\node[fill=thm@bg@color] (texte) {\usebox{\thm@box}};
	\node[fill=thmtitle@bg@color,text=thmtitle@color,below left,draw=thmtitle@bg@color,very thick,text width={\dimexpr\marginleft+5mm}]%,align=center] 
	at ($(texte.north west)+(-1pt,0)$) {\thmtitle@font\thechapter.\thethm.\  \titletext\plural@thm \ifnum\thebreak@box>0\par(continuaci\'on)\fi};
	\end{tikzpicture}
}

%TODO %----------------------Lema---------------------------------------------

\newsavebox{\lema@box}
\newcounter{lema}[chapter]
\newenvironment{lemma}[1][]
{%
	\setcounter{break@box}{0}
	\stepcounter{lema}
	\setcounter{type@box}{1}
	\typebox{lema}
	\def\titletext{Lema}
	\ifx#1\@empty\xdef\plural@lema{}\else\xdef\plural@lema{#1}\fi
	\itemclass{thmtitle@bg@color}{\thmtitle@font}
	\begin{lrbox}{\lema@box}
	\begin{minipage}{\dimexpr\linewidth-1.3\marginleft}
}
{%
	\end{minipage}
	\end{lrbox}
	\vspace*{\spacebeforethm}
	\begin{tikzpicture}
	\node[fill=thm@bg@color] (texte) {\usebox{\lema@box}};
	\node[fill=thmtitle@bg@color,text=thmtitle@color,below left,draw=thmtitle@bg@color,very thick,text width={\dimexpr\marginleft+5mm}] at ($(texte.north west)+(1pt,0)$) {\thmtitle@font\thechapter.\thelema.\ \titletext\plural@lema \ifnum\thebreak@box>0\par(continuaci\'on)\fi};
	\end{tikzpicture}
}

%TODO %-----------------------Axioma ---------------------------------------------------

\newsavebox{\axi@box}
\newenvironment{axiom}[1][]
{%
	\setcounter{break@box}{0}
	\setcounter{type@box}{1}
	\typebox{axi}
	\def\titletext{Axioma}
	\ifx#1\@empty\xdef\plural@axi{}\else\xdef\plural@axi{#1}\fi
	\itemclass{thmtitle@bg@color}{\thmtitle@font}
	\begin{lrbox}{\axi@box}
	\begin{minipage}{\dimexpr\linewidth-1.3\marginleft}
}
{%
	\end{minipage}
	\end{lrbox}
	\vspace*{\spacebeforethm}
	\begin{tikzpicture}
	\node[fill=thm@bg@color] (texte) {\usebox{\axi@box}};
	\node[fill=thmtitle@bg@color,text=thmtitle@color,below left,draw=thmtitle@bg@color,very thick,text width={\dimexpr\marginleft+5mm}] at ($(texte.north west)+(1pt,0)$) {\thmtitle@font \titletext\plural@axi \ifnum\thebreak@box>0\par(continuaci\'on)\fi};
	\end{tikzpicture}
}

%TODO %-------------------------- corolario ---------------------------------------------

\newsavebox{\cor@box}
\newcounter{coro}[chapter]
\newenvironment{corollary}[1][]
{%
	\setcounter{break@box}{0}
	\setcounter{type@box}{1}
	\stepcounter{coro}
	\typebox{cor}
	\def\titletext{Corolario}
	\ifx#1\@empty\xdef\plural@cor{}\else\xdef\plural@cor{#1}\fi
	\itemclass{thmtitle@bg@color}{\thmtitle@font}
	\begin{lrbox}{\cor@box}
	\begin{minipage}{\dimexpr\linewidth-1.3\marginleft}
}
{%
	\end{minipage}
	\end{lrbox}
	\vspace*{\spacebeforethm}
	\begin{tikzpicture}
	\node[fill=thm@bg@color] (texte) {\usebox{\cor@box}};
	\node[fill=thmtitle@bg@color,text=thmtitle@color,below left,draw=thmtitle@bg@color,very thick,text width={\dimexpr\marginleft+5mm}] at ($(texte.north west)+(1pt,0)$) {\thmtitle@font\thechapter.\thecoro.\ \titletext\plural@cor \ifnum\thebreak@box>0\par(continuaci\'on)\fi};
	\end{tikzpicture}
}

%TODO %%%%%%%%%% Entorno "ejemplo"

\newsavebox{\ex@box}
\newcounter{exe}[chapter]
\newenvironment{example}[1][]
{%
	\setcounter{break@box}{0}
	\setcounter{type@box}{1}
	\stepcounter{exe}
	\typebox{ex}
	\def\titletext{Ejemplo}
	\ifx#1\@empty\xdef\plural@ex{}\else\xdef\plural@ex{#1}\fi
	\itemclass{ejtitle@bg@color}{\ejtitle@font}
	\begin{lrbox}{\ex@box}
	\begin{minipage}{\dimexpr\linewidth-\marginleft}
}
{%
	\end{minipage}
	\end{lrbox}
	\vspace*{\spacebeforeex}
	\begin{tikzpicture}
	\node[fill=ej@bg@color] (texte) {\usebox{\ex@box}};
	\node[fill=ejtitle@bg@color,text=ejtitle@color,below left,draw=ejtitle@bg@color,very thick,text width={\dimexpr\marginleft+2mm},align=center] at ($(texte.north west)+(1pt,0)$) {\ejtitle@font\thechapter.\theexe.\ \titletext\plural@ex \ifnum\thebreak@box>0\par(continuaci\'on)\fi};
	\end{tikzpicture}
}

%TODO  %%%%%%%%%%%%%%%%%%%%%%%%%% entorno ejercicio %%%%%%%%%%%%%%%%%%%
\newsavebox{\ej@box}
\newcounter{eje}[chapter]
\newenvironment{exercise}[1][]
{%
	\setcounter{break@box}{0}
	\setcounter{type@box}{1}
	\stepcounter{eje}
	\typebox{ej}
	\def\titletext{Ejemplo}
	\ifx#1\@empty\xdef\plural@ej{}\else\xdef\plural@ej{#1}\fi
	\itemclass{ejtitle@bg@color}{\ejtitle@font}
	\begin{lrbox}{\ej@box}
	\begin{minipage}{\dimexpr\linewidth-\marginleft}
}
{%
	\end{minipage}
	\end{lrbox}
	\vspace*{\spacebeforeex}
	\begin{tikzpicture}
	\node[fill=ej@bg@color] (texte) {\usebox{\ej@box}};
	\node[fill=ejtitle@bg@color,text=ejtitle@color,below left,draw=ejtitle@bg@color,very thick,text width={\dimexpr\marginleft+2mm},align=center] at ($(texte.north west)+(1pt,0)$) {\ejtitle@font\thechapter\theeje.\ \titletext\plural@ej \ifnum\thebreak@box>0\par(continuaci\'on)\fi};
	\end{tikzpicture}
}

%TODO  %%%%%%%%%%%%%%%%%%%%%%%%%% entorno problema %%%%%%%%%%%%%%%%%%%
\newsavebox{\prob@box}
\newcounter{prob}[chapter]
\newenvironment{problem}[1][]
{%
	\setcounter{break@box}{0}
	\setcounter{type@box}{1}
	\stepcounter{prob}
	\typebox{prob}
	\def\titletext{Problema}
	\ifx#1\@empty\xdef\plural@prob{}\else\xdef\plural@prob{#1}\fi
	\itemclass{ejtitle@bg@color}{\ejtitle@font}
	\begin{lrbox}{\prob@box}
	\begin{minipage}{\dimexpr\linewidth-\marginleft}
}
{%
	\end{minipage}
	\end{lrbox}
	\vspace*{\spacebeforeex}
	\begin{tikzpicture}
	\node[fill=ej@bg@color] (texte) {\usebox{\probbox}};
	\node[fill=ejtitle@bg@color,text=ejtitle@color,below left,draw=ejtitle@bg@color,very thick,text width={\dimexpr\marginleft+2mm},align=center] at ($(texte.north west)+(1pt,0)$) {\ejtitle@font\thechapter\theprob.\ \titletext\plural@prob \ifnum\thebreak@box>0\par(continuaci\'on)\fi};
	\end{tikzpicture}
}

%TODO %%%%%%%%%% Entorno "demostration"

\newsavebox{\dem@box}
\newenvironment{demostration}[1][1]
{%
	\setcounter{break@box}{0}
	\setcounter{type@box}{2}
	\typebox{dem}
	\def\titletext{Demostraci\'on}
	\ifx#1\@empty\xdef\indic{0}\else\xdef\indic{#1}\fi
	\xdef\indic{#1}
	\begin{lrbox}{\dem@box}
	\begin{minipage}{\dimexpr\linewidth-0.5\marginleft}
	{\demtitle@font\textcolor{dem@line@color}{ \titletext}}\par\medskip%
}
{%
	\vskip-1em\hfill\rule{0.5em}{0.5em}
	\end{minipage}
	\end{lrbox}
	\vspace*{\spacebeforedem}
	\begin{tikzpicture}
	\node[fill=dem@bg@color] (texte) {\usebox{\dem@box}};
	\ifnum\indic=1
		\draw[line width=0.5mm,draw=dem@line@color,decorate, decoration={random steps, segment length=3pt,amplitude=1pt}] (texte.north west) -- (texte.south west);
	\else
		\ifnum\indic=2
			\draw[line width=0.5mm,draw=dem@line@color,decorate, decoration={random steps, segment length=3pt,amplitude=1pt}] (texte.north west) -- (texte.south west) -- (texte.south east);
		\else
			\ifnum\indic=3
				\draw[line width=0.5mm,draw=dem@line@color,decorate, decoration={random steps, segment length=3pt,amplitude=1pt}] (texte.north west) -- (texte.south west);
				\draw[line width=0.5mm,draw=,color=dem@line@color,decorate, decoration={random steps, segment length=3pt,amplitude=1pt}] (texte.north east) -- (texte.south east);
			\else
				\ifnum\indic=4
					\draw[line width=0.5mm,draw=dem@line@color,decorate, decoration={random steps, segment length=3pt,amplitude=1pt}] (texte.north east) -- (texte.north west) -- (texte.south west);
				\else
					\ifnum\indic=5
						\draw[line width=0.5mm,draw=dem@line@color,decorate, decoration={random steps, segment length=3pt,amplitude=1pt}] (texte.north west) -- (texte.south west) -- (texte.south east) -- (texte.north east) -- cycle;
					\fi
				\fi
			\fi
		\fi
	\fi
	\node[below left,minimum width=0.5\marginleft] at (texte.north west) {\phantom{-}};
	\end{tikzpicture}
}

%TODO %%%%%%%%%%% Actividad

\newcounter{actividad}[chapter]

\newcommand{\actividad}[1]
{%
	\refstepcounter{actividad}
	\vspace*{\spacebeforeact}
	\begin{tikzpicture}
	\fill[shadow@color] (0.1,-.15) circle[x radius=1.5em,y radius=0.66em,rotate=10];
	\fill[actividadtitle@color,rotate=30] (0,0) circle[x radius=1.5em,y radius=1em];
	\node[text=actividadnum@color] at (0,0) {\actividad@font\theactividad};
	\node[right,text=actividadtitle@color] at (0.75,0) {\actividad@font#1};
	\end{tikzpicture}
	\itemclass{actividadtitle@color}{\actividad@font}
}

%TODO %%%%%%%%%%%%%%%%% Estilos de itemize

\newcommand{\itemclass}[2] % #1 = color ; #2 = fuente
{
	\setlist[itemize,1]{label={\color{#1}\textbullet}}
	\setlist[itemize,2]{label={\color{#1}$\rightarrow$}}
	\setlist[enumerate,1]
	{%
		label=\fcolorbox{#1}
		{#1!20}
		{\color{#1}#2\normalsize\arabic*}
	}
	\setlist[enumerate,2]{label=\textcolor{#1}{#2\normalsize\alph*.}}
}
\itemclass{black}{}

%TODO %%%%%%%%%%%%%%%%%%%%%%%%%%%%%%%%%%%%%%%%%%%%%%%%%%%%%%%%%% Exercicios

% Declaración de la primera página de ejercicios

\newcounter{indicCorr}
\newcommand{\exostart}[1][0]
{%
	\newpage
	\pagecolor{ejercicios@bg@color}
	\begin{flushright}
	\begin{tikzpicture}
	\node[inner sep=1em] (titre) {\ejerciciotitle@font Ejercicios};
	\fill[rounded corners=2pt,fill=shadow@color] ($(titre.north west)+(2pt,-2pt)$) -- ($(titre.north east)+(1mm,0mm)+(2pt,-2pt)$) -- ($(titre.south east)+(2pt,-2pt)$) -- ($(titre.south west)+(-1mm,0)+(2pt,-2pt)$) -- cycle;
	\fill[rounded corners=2pt,fill=ejerciciotitle@bg@color] (titre.north west) -- ($(titre.north east)+(1mm,0mm)$) -- (titre.south east) -- ($(titre.south west)+(-1mm,0)$) -- cycle;
	\node[inner sep=1em,text=ejerciciotitle@color] at (titre) {\ejerciciotitle@font Ejercicios};
	\end{tikzpicture}
	\end{flushright}
	\ifnum#1=1
		\setcounter{indicCorr}{1}
	\else
		\setcounter{indicCorr}{0}
	\fi
}

%TODO  Entorno"ejercicio"

\newcounter{ejercicio}[chapter]
\newcounter{chapterant}

\newsavebox{\exo@box}
\newenvironment{ejercicio}[1][0]
{%
	\ifnum#1=0
		\refstepcounter{ejercicio}
		\expandafter\xdef\csname nbexos\thechapter\endcsname{\theejercicio}
		\ifnum\theejercicio<10
			\def\num@exo{0\theejercicio}
		\else
			\def\num@exo{\theejercicio}
		\fi
	\else
		\xdef\num@exo{}
	\fi
	\itemclass{ejercicionum@bg@color}{\ejercicionum@font}
	\begin{lrbox}{\exo@box}
	\begin{minipage}{\dimexpr\linewidth-3em-10pt}
}
{%
	\ifnum\theindicCorr=1
	\begin{flushright}
	\textcolor{corref@color}{\corref@font P\'agina con las correcciones \pageref{correcci\'on-\thechapter-\theejercicio}}
	\end{flushright}	
	\fi
	\end{minipage}
	\end{lrbox}
	\ifnum\theejercicio>1
		\ifx\num@exo\@empty
		\else
			\par\vspace*{\spacebeforeexo}
		\fi
	\fi
	\begin{tikzpicture}
	\node (texte) {\usebox{\exo@box}};
	\ifx\num@exo\@empty
		\node[below left,minimum width=2em] at ($(texte.north west)+(-0.5em,0)$) {\phantom{\ejercicionum@font\num@exo}};
	\else
		\node[below left,minimum width=2em] (num) at ($(texte.north west)+(-0.5em,0)$) {\ejercicionum@font\num@exo};
		\fill[rounded corners=2pt,fill=ejercicionum@bg@color] (num.north west) -- ($(num.north east)+(1mm,0mm)$) -- (num.south east) -- ($(num.south west)+(-1mm,0)$) -- cycle;
		\node[text=ejercicionum@color] at (num) {\ejercicionum@font\num@exo};
	\fi
	\end{tikzpicture}	
}

%TODO %%%%%%%%%%%%%% Corregido
% Declaración de la 1ª página de correcciones

\newcommand{\corrstart}
{%
	\newpage
	\pagecolor{corriges@bg@color}
	\begin{flushright}
	\begin{tikzpicture}
	\node[inner sep=1em] (titre) {\corrigetitle@font Correcci\'on de los ejercicios};
	\fill[rounded corners=2pt,fill=shadow@color] ($(titre.north west)+(2pt,-2pt)$) -- ($(titre.north east)+(1mm,0mm)+(2pt,-2pt)$) -- ($(titre.south east)+(2pt,-2pt)$) -- ($(titre.south west)+(-1mm,0)+(2pt,-2pt)$) -- cycle;
	\fill[rounded corners=2pt,fill=corrigetitle@bg@color] (titre.north west) -- ($(titre.north east)+(1mm,0mm)$) -- (titre.south east) -- ($(titre.south west)+(-1mm,0)$) -- cycle;
	\node[inner sep=1em,text=corrigetitle@color] at (titre) {\corrigetitle@font Correcci\'on de los ejercicios};
	\end{tikzpicture}
	\end{flushright}
}

%TODO BreakCorr

\newcommand{\BreakCorr}[1][\newpage]
{%
\end{corrige}
#1
\begin{corrige}[\thechapter]{\i}
\bgroup\ejercicionum@font\textbf{(continuaci\'on)}\egroup~
}
%TODO ---------------------------------------------
%\begin_layout Standard
%\begin_inset CommandInset include
%LatexCommand input
%filename "ejercicios-lyx.tex"
%\end_inset
%\end_layout
%%---------------------------------
%\begin_layout Standard
%\begin_inset CommandInset include
%LatexCommand include
%filename "ejercicios.lyx"
%\end_inset
%\end_layout
%\InputIfFileExists{tipp.def}{}{}
%---------------------------------------------------

\newcommand{\PagCorregidos}[1][]
	{%
		\multido{\i=1+1}{\theejercicio}{%
			\foreach\x/\macr in {#1}%
			{%
				\ifnum\x=\i\macr\fi%
			}%
%			\if@lyx
%	           \InputIfFileExists{corregidos-lyx/\thechapter-\i .lyx}{}{}
%	           \else
%	           \InputIfFileExists{corregidos-tex/\thechapter-\i .tex}%
%               \fi
                \input{\thechapter-\i}
					}
	}

%% Parte del código para escribir en un archivo %% Gracias pág....

\newwrite{\output@stream@corrige}

\newcommand{\OuvrirFlux}[1]
{%flujo del archivo abierto
  \immediate\openout\output@stream@corrige #1%
}
\newcommand{\FermerFlux}
{% cerrar la secuencia de archivos
  \immediate\closeout\output@stream@corrige%
}
\newcommand{\EcritureFlux}[1]
{%escribir un texto breve en el archivo
  \immediate\write\output@stream@corrige{\unexpanded{#1}}%
}
\newcommand{\CommencerExportFlux}
{% comenzar la escritura extendida
   \bgroup\@bsphack
   \let\do\@makeother\dospecials
   \catcode`\^^M\active
   \def\verbatim@processline{%
     \immediate\write\output@stream@corrige{\the\verbatim@line}}%
   \verbatim@start}
\ifdefined\ArreterExportFlux % detener la escritura prolongada
  \newcommand{\ArreterExportFlux}{\@esphack\egroup}
\else
  \def\ArreterExportFlux{\@esphack\egroup}
\fi
%TODO ------------------------------------------------------
\newenvironment{correction}
  {
%  \if@lyx
%  \OuvrirFlux{corregidos-lyx/\thechapter-\theejercicio.lyx}
%   \else
%    
%    \fi
   %
   \OuvrirFlux{\thechapter-\theejercicio.tex}
   \EcritureFlux{\begin{corrige}[\thechapter]{\i}}%
   \CommencerExportFlux}
  {\ArreterExportFlux%
   \EcritureFlux{\end{corrige}\par\vspace*{\spacebeforeexo}}%
   \FermerFlux}

\newsavebox{\corr@box}
\newenvironment{corrige}[2][]
{%
	\label{corregido-#1-#2}%
	\ifnum#2<10%
		\def\num@exo{0#2}%
	\else%
		\def\num@exo{#2}%
	\fi%
	\itemclass{corrigenum@bg@color}{\ejercicionum@font}%
	\begin{lrbox}{\corr@box}
	\begin{minipage}{\dimexpr\linewidth-3em-10pt}
}
{%
	\end{minipage}
	\end{lrbox}
	\begin{tikzpicture}
	\node (texte) {\usebox{\corr@box}};
	\ifx\num@exo\@empty
		\node[below left,minimum width=2em] at ($(texte.north west)+(-0.5em,0)$) {\phantom{\ejercicionum@font\num@exo}};
	\else
		\node[below left,minimum width=2em] (num) at ($(texte.north west)+(-0.5em,0)$) {\ejercicionum@font\num@exo};
		\fill[rounded corners=2pt,fill=corrigenum@bg@color] (num.north west) -- ($(num.north east)+(1mm,0mm)$) -- (num.south east) -- ($(num.south west)+(-1mm,0)$) -- cycle;
		\node[text=corrigenum@color] at (num) {\ejercicionum@font\num@exo};
	\fi
	\end{tikzpicture}
}

%TODO %%%%%%%%%%%%%% INDEX

\newcommand\printindex{%
	\phantomsection\addcontentsline{toc}{chapter}{\indexname}
	\@input@{\jobname.ind}
}
\def\nbcolindex{2}
\renewenvironment{theindex}
{%
	\if@twocolumn
		\@restonecolfalse
	\else
		\@restonecoltrue
	\fi
	\newpage
	\begin{flushright}\begin{tikzpicture}
	\node[fill=indextitle@bg@color,text=indextitle@color, xslant=0.2,rounded corners=2pt,inner xsep=1em,inner ysep=1ex] {\indextitle@font\indexname};
	\end{tikzpicture}\end{flushright}
	\pagecolor{white}
	\thispagestyle{main}\parindent\z@
	\let\item\@idxitem
	\columnseprule \z@
	\columnsep 35\p@
	\ifnum\nbcolindex>1
	\begin{multicols}{\nbcolindex}
	\fi
}
{%
	\ifnum\nbcolindex>1
	\end{multicols}
	\fi
	\if@restonecol\onecolumn\else\clearpage\fi
}

%TODO %%%%%%%%%%%% TOC
\def\micontour{\contour{contentstitleshadow@color}\contentsname}
\def\figcontenido{%
%  \AddToShipoutPicture
%  	{%
%  		\put(\LenToUnit{0mm},\LenToUnit{0mm})
%  		{%
			\begin{tikzpicture}
	    		\clip (0,0) rectangle+(\paperwidth,-\paperheight);
	    		\draw[decorate,decoration={ text effects along path,text along path,text, 
text={|\contenidobgfont\color{contentstitleshadow@color}|\contentsname}}] (1.6,-8.1) .. controls ($(\controltoctitle,-\controltoctitle)+(0.1,-0.1)$) .. (20.1,-2.1);
	    		\draw[decorate,decoration={text along path,
text={|\contenidobgfont\color{contentstitle@color}|\contentsname}}] (1.5,-8) .. controls (\controltoctitle,-\controltoctitle) .. (20,-2);
			\end{tikzpicture}
%		}
%	}
                                                           }

%\renewcommand{\contentsname}{Contenido}
\addto\captionsspanish{% Replace "spanish" with the language you use
  \renewcommand{\contentsname}%
    {Contenido}%
}
\renewcommand\tableofcontents{
\begin{tikzpicture}[remember picture,overlay]
  \node[anchor=north west,inner sep=0pt] at ($(current page.north west)-(-0.1cm,-0.3cm)$){
            \figcontenido
             };
\end{tikzpicture}
		\if@twocolumn
          \@restonecoltrue\onecolumn
        \else
          \@restonecolfalse
        \fi
        \@starttoc{toc}%
        \if@restonecol\twocolumn\fi
        }
%%%------------------------------------------------------
\newcommand{\chaptertoc}[1]{\addtocontents{toc}{#1\par}}

%%%%%%%%%%%%%%%%%%%%%%%%%%%%%%%%%%%%%%%%%%%%%%%%%%%%%%%%%%%%%%%%%%%%%%
