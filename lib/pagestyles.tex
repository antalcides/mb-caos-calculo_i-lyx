%%%%%%%%%%%% Definición de los estilos de páginas

\fancypagestyle{chapterstart}{% primera página del capítulo
	\fancyhf{}%
  	\fancyfoot[RO,RE]{%
  	\vskip0.1pt
  	\begin{tikzpicture}
  	\node[rounded corners=3pt,fill=shadow@color,inner ysep=5pt, inner xsep=2ex,text=shadow@color] (shadow) {\pagenumber@font\thepage};
  	\node[rounded corners=3pt,fill=pagenumber@bg@color,inner ysep=5pt, inner xsep=2ex,text=white] at ($(shadow)+(-2pt,2pt)$) {\pagenumber@font\thepage};
  	\end{tikzpicture}
  	}
}

\fancypagestyle{toc}{% Resumen
	\fancyhf{}%
  	\fancyfoot{}
}

\fancypagestyle{main}{% 
	\fancyhf{}%
  	\fancyfoot[RO]{% páginas impaires
  	\ClearShipoutPicture
  	\vskip0.1pt  	
  	\begin{tikzpicture}
  	\node[rounded corners=3pt,fill=shadow@color,inner ysep=5pt, inner xsep=2ex,text=shadow@color] (shadow) {\pagenumber@font\thepage};
  	\node[rounded corners=3pt,fill=pagenumber@bg@color,inner ysep=5pt, inner xsep=2ex,text=white] at ($(shadow)+(-2pt,2pt)$) {\pagenumber@font\thepage};
  	\end{tikzpicture}
  	}
  	\fancyfoot[LE]{% páginas paires	
  		\ClearShipoutPicture
  		\vskip0.1pt
  		\begin{tikzpicture}
  		\node[rounded corners=3pt,fill=shadow@color,inner ysep=5pt, inner xsep=2ex,text=shadow@color] (shadow) {\pagenumber@font\thepage};
  		\node[rounded corners=3pt,fill=pagenumber@bg@color,inner ysep=5pt, inner xsep=2ex,text=white] at ($(shadow)+(2pt,2pt)$) {\pagenumber@font\thepage};
  		\end{tikzpicture}
  	}
}

%%%%%%%%%%%% Encabezado y pié de página

\renewcommand{\headrulewidth}{0pt}
\renewcommand{\footrulewidth}{0pt}
\pagestyle{main}

%%%%%%%%%%%% Definiciones predeterminadas
\newcommand{\intro}[1]{\def\chapterintro{#1}}
\newcommand{\introauthor}[1]{\def\chapterintroauthor{#1}}
\introauthor{} % autor de la cita al comienzo del capítulo
\intro{} % Cita al comienzo del capítulo; para declarar antes del comando \chapter{}


%%%%%%%%%%%% Página de presentación

%%%%%%%%%%%% Primera página de cada capítulo

\titleformat{\chapter}
{%
    	\thispagestyle{chapterstart}
	\pagecolor{mesh}
	%\pagecolor{fondpaille}
	\bfseries\Huge\raggedleft
	\color{chapter@title@color}
	\chaptertitle@font
}
{%
  \AddToShipoutPicture
  {%
  	\put(\LenToUnit{0mm},\LenToUnit{0mm})
  	{%
	\begin{tikzpicture}
	    \clip (0,0) rectangle+(\paperwidth,-\paperheight);
	    % Rectángulo superior derecho
		\fill[bottom color=chapterdegradado@color,top color=white] (0,0) rectangle+(\paperwidth,-3);
		% Línea inclinada en el rectángulo en la parte superior derecha
	    \draw[chapterline@color] (0,3) -- (\paperwidth,-3);
	    % Pequeño círculo vacío en el rectángulo en la parte superior derecha
	    \draw[chapterline@color,very thick,opacity=0.2,fill=gray] (.8\paperwidth,-2) circle (5mm);
	    % Pequeño círculo relleno en la parte superior del rectángulo creado
	    \fill[chaptertopcircle@color] ($(.8\paperwidth,-2)+(8mm,8mm)$) circle (2mm);
	    %%%%%%%%%%%%%%%%%%%%%%%%%%%%%%%%%%%%%%%%%%%%%%%%%%%%%%%%%%%%%%%%%%%%%%%%%%%%%%	    
	    % Pequeña línea inclinada en el rectángulo en la parte superior derecha
	    \draw[chapterline@color] (8,3) -- (\paperwidth,-0.5);
	    % Pequeño círculo vacío en el rectángulo en la parte superior izquierda
	    \draw[chapterline@color,very thick,opacity=0.2,fill=gray] (.6\paperwidth,-.6) circle (5mm);
	    % Pequeño círculo relleno en la parte superior izquierda del rectángulo creado
	    \fill[chaptertopcircle@color] (.57\paperwidth,-1.5) circle (2mm);
	    % Posible cita
	    \ifx\chapterintro\@empty
	    \else
	    	  \node[below left,text=chapterintro@color] (citation) at (21,-0.5) {" \emph{\chapterintro} "};
	    	  \ifx\chapterintroauthor\@empty
	    	  \else
	    	     \node[below left,text=chapterintro@color] at (citation.south east) {(\chapterintroauthor)};
	    	  \fi
	    \fi
	    % círculo vacío en la parte inferior izquierda
%	    \fill[chapterbotcircle@color] (1,-27) circle (3cm);
%	    \fill[white] (1,-27) circle (2.5cm);
	    % Círculo pequeño casi centrado verticalmente
%	    \fill[chaptermidcircle@color] (1.2,-17) circle (.7cm);
%	    \fill[white] (1.2,-17) circle (0.6cm);
%	     % Líneas verticales a la izquierda
	    \draw[very thick,chapterrule@color] (1cm,0) -- (1cm,-\paperheight);
	    \draw[very thick,chapterrule@color] (1.2cm,0) -- (1.2cm,-\paperheight);
%	    % Elipse arriba a la izquierda
%	    \fill[chapterelipse@bg@color] (0,0) circle(9cm);%ellipse[x radius=9cm,y radius=5cm];
	    %%%%%%%%%%%%%%%%%%%%%%%%%%%%%%%%%%%%%%%%%%%%%%%%%%%%%%%%%%%%%%%%%%%%%%%%%%%%%%%%%%%%
%	    % Sombra de la elipse
%	    \foreach \rx/\ry/\c in { %
%	    8.7/4.7/10,% 
%	    8.66/4.66/30,%
%	    8.62/4.62/50,%
%	    8.58/4.58/70,%
%	    8.54/4.54/90%
%	    }
%	    {
%		\fill[fill=black!\c!chapterelipse@bg@color] (0,0) ellipse[x radius=\rx cm,y radius=\ry cm];
%	    }
%	    % elipse
%	    \fill[chapterelipse@color] (0,0) ellipse[x radius=8.5cm,y radius=4.5cm];    
%	    % En la elipse en la parte superior izquierda...
%	    \clip (0,0) ellipse[x radius=8.5cm,y radius=4.5cm];
	    %%%%%%%%%%%%%%%%%%%%%%%%%%%%%%%%%%%%%%%%%%%%%%%%%%%%%%%%%%%%%%%%%%%%%%%%%%%%%%%%
	    	 \fill[chapterelipse@bg@color] (0,0) circle(5cm);   
	    	    % Sombra del circulo
	    \foreach \rx/\c in { %
	    4.7/10,% 
	    4.66/30,%
	    4.62/50,%
	    4.58/70,%
	    4.54/90%
	    }
	    {
		\fill[fill=ejercicios@bg@color!\c!chapterelipse@bg@color] (0,0) circle(\rx cm);
	    }
	    	     \clip (0,0) circle(5cm);
	    %%%%%%%%%%%%%%%%%%%%%%%%%%%%%%%%%%%%%%%%%%%%%%%%%%%%%%%%%%%%%%%%%%%%%%%%%%%%%%%%%%%
	     % Líneas verticales a la izquierda
	    \draw[very thick,chapterrule@color] (1cm,0) -- (1cm,-\paperheight);
	    \draw[very thick,chapterrule@color] (1.2cm,0) -- (1.2cm,-\paperheight);
	    \fill[chapter@bg@color] (0,-0.8) rectangle+(15,-1.3);
	    \node[text=chaptertitlename@color,right] at (0.1,-1.5) {\huge\chaptertitlename@font\bfseries\chaptertitlename~\thechapter};
  \end{tikzpicture}
	\chaptertoc   }
  }
}
{7ex}
{}

%%%%%%%%%%%%%%%%%%%%%%%%%%%%%%%%%%%%%%%%%%%%%%%%%%%%%%%%%%%%%%%
%%%%%%%%%%%%%%%%%%%%% Tabla de contenido %%%%%%%%%%%%%%%%%%%%%
%%%%%%%%%%%%%%%%%%%%%%%%%%%%%%%%%%%%%%%%%%%%%%%%%%%%%%%%%%%%%%%


\newcommand{\chaptertoccolor}{white}
\newcommand{\sectiontoccolor}{section@title@color}
\newcommand{\subsectiontoccolor}{subsection@title@color}
\newcommand{\subsubsectiontoccolor}{subsubsection@title@color}

\renewcommand*\l@chapter[2]
{%
  \ifnum\c@tocdepth>\m@ne
    \addpenalty{-\@highpenalty}%
    \vskip 1.0em \@plus\p@
    \setlength\@tempdima{1.5em}%
    \begingroup
      \parindent \z@ \rightskip \@pnumwidth
      \parfillskip -\@pnumwidth
      \leavevmode \bfseries \chaptertitle@font \large
      \advance\leftskip\@tempdima
      \hskip -\leftskip
      \def\@linkcolor{\chaptertoccolor}%
     \raisebox{-0.6em}{%
      \begin{tikzpicture}%
      \node[rounded corners=3pt,fill=chapter@title@color,text=white, xslant=0.2] {#1};
      \end{tikzpicture}%
      }
      \color{chapter@title@color}
      \nobreak\
       \leaders\hbox{$\m@th
        \mkern \@dotsep mu\hbox{.}\mkern \@dotsep
        mu$}\hfil\nobreak\hb@xt@\@pnumwidth{\hss
        \def\@linkcolor{chapter@title@color}%
        \color{\chaptertoccolor}#2}\par
      \penalty\@highpenalty
    \endgroup
  \fi
  \medskip
}

%\def\@dottedtocline#1#2#3#4#5
%{%
%    \vskip \z@ \@plus.2\p@
%    {%
%     \leftskip#2\relax\rightskip\@tocrmarg\parfillskip-\rightskip
%     \parindent #2\relax\@afterindenttrue
%     \interlinepenalty\@M
%     \leavevmode
%     \@tempdima #3\relax
%     \advance\leftskip \@tempdima \null\nobreak\hskip -\leftskip
%     {%
%     	\sectiontitle@font#4
%     }
%     \nobreak\sectiontitle@font
%     \leaders\hbox{$\m@th
%        \mkern \@dotsep mu\hbox{.}\mkern \@dotsep
%        mu$}\hfill
%     \nobreak
%     \hb@xt@\@pnumwidth{\hfil #5}%
%     \par}%
%}
  
\renewcommand*\l@section{%
\color{\sectiontoccolor}
\def\@linkcolor{\sectiontoccolor}
\@dottedtocline{1}{0em}{2em}
}

\renewcommand*\l@subsection{%
\color{\subsectiontoccolor}
\def\@linkcolor{\subsectiontoccolor}
\@dottedtocline{1}{2em}{2em}
}

\renewcommand*\l@subsubsection{%
\color{\subsubsectiontoccolor}
\def\@linkcolor{\subsubsectiontoccolor}
\@dottedtocline{1}{4em}{2em}
}

\def\contentsline#1#2#3#4{%
  \ifx\\#4\\%
    \csname l@#1\endcsname{#2}{#3}%
  \else
      \csname l@#1\endcsname{\hyper@linkstart{link}{#4}{#2}\hyper@linkend}{%
        \hyper@linkstart{link}{#4}{#3}\hyper@linkend
      }%
  \fi
}

%%%%%%%%%%%%%%% Portada

\def\coef@bg@pic{1.05}
\def\coef@opacity@pic{0.2}
\newcommand{\titlepic}[2][1]{%
	\FPeval\newscale{\coef@bg@pic*#1}
	\def\@titlepic{\includegraphics[scale=#1]{#2}}
	\def\@titlepicbg{\includegraphics[scale=\newscale]{#2}}
}
\newcommand{\logopic}[2][1]{%
	\FPeval\newscale{\coef@bg@pic*#1}
	\def\@logopic{\includegraphics[scale=#1]{#2}}
	\def\@logopicbg{\includegraphics[scale=\newscale]{#2}}
}
\renewcommand{\and}{\par}

\renewcommand{\maketitle}
{%
	\newgeometry{margin=0pt}
	\begin{tikzpicture}
	\clip (current page.south west) rectangle (current page.north east);
	% Fondo da la portada
	\shade[top color=saddlebrown,bottom color=corref@color] (current page.south west) rectangle (current page.north east);
	% Líneas horizontales y verticales
	\foreach \y in {1,2,...,10}
	{
		\draw[color=firstpage@lines@color] ($(current page.north west)+(0,-\y*0.5)$) -- ($(current page.north east)+(-0.125*\y,-0.5*\y)$);
		\draw[color=firstpage@lines@color] ($(current page.north west)+(\y*0.5,0)$) -- ($(current page.north west)+(0,-\y)$);
	}
	% Título
	\node[below right,text=titledoc@color] at ($(current page.north west)+(\marginlefttitle,-\margintoptitle)$) {\begin{minipage}{\dimexpr\paperwidth-2\marginlefttitle}\titledoc@font\@title\end{minipage}};
	% Autor(es)
	\node[below left,text=titleauthor@color] at ($(current page.north east)+(-\marginlefttitle,-\margintopauthor)$) {%
	\begin{minipage}{\dimexpr\paperwidth-2\marginlefttitle}
	\raggedleft
	\titleauthor@font\@author
	\end{minipage}};
	% Imagen (opcional)
	\ifx\titlepic\@empty\else
		\node[above left] (pic) at ($(current page.south east)+(-\marginrightpic,\marginbottompic)$) {\@titlepic};
		\node[opacity=\coef@opacity@pic] at (pic.center) {\@titlepicbg};
		\node[above left] (pic) at ($(current page.south east)+(-\marginrightpic,\marginbottompic)$) {\@titlepic};
	\fi
	% Date
	\node[left,text=titledate@color] at ($0.5*(current page.north east)+0.5*(current page.south east)+(-\marginrightdate,\marginbottomdate)$) {\titledate@font\@date};
	%logo (opcional)
	%logo (opcional)
	\ifx\logopic\@empty\else
		\node[above right] (pic1) at ($(current page.south west)+(\marginrightpic,\marginbottompic)$) {\@logopic};
		\node[opacity=\coef@opacity@pic] at (pic1.center) {\@titlepicbg};
		\node[above right] (pic1) at ($(current page.south west)+(\marginrightpic,\marginbottompic)$) {\@logopic};
	\fi
	\end{tikzpicture}
	\if@openany
	\else
		\cleardoublepage
	\fi
	\restoregeometry
}
