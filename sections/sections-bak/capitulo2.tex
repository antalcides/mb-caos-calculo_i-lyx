
\chapter{L\'{\i}mite y Continuidad}

En este cap\'{\i}tulo se presenta uno de los conceptos fundamentales del
C\'{a}lculo y, en general, del An\'{a}lisis matem\'{a}tico: El concepto de
l\'{\i}mite. En principio-y ese ser\'{a} el tratamiento del concepto en este
curso- la idea de l\'{\i}mite est\'{a} ligada a la de \textquotedblleft
distancia\textquotedblright\ o, m\'{a}s precisamente, a la de
\textquotedblleft cercan\'{\i}a \textquotedblright, por lo que nuestro
inter\'{e}s inicial estar\'{a} en definir formalmente lo que consideraremos
una \textquotedblleft distancia\textquotedblright\ tanto en el sentido
intuitivo, ligado a problemas de naturaleza f\'{\i}sica o geom\'{e}trica, como
en la abstracci\'{o}n del mismo.

\section{Topolog\'{\i}a de la recta real}

Para dos puntos cualesquiera $P$ y $Q$ de la recta real, la
\index{Distancia|textbf}%
distancia entre ellos viene dada por el real no negativo
\begin{equation}
d(P,Q)=|x_{1}-x_{2}|, \label{distanciaenR}%
\end{equation}
donde $x_{1}$ y $x_{2}$ son las coordenadas de los puntos $P$ y $Q$,
respectivamente. Diremos tambi\'{e}n que tal distancia es la distancia entre
los reales $x_{1}$ y $x_{2}$. La distancia definida en $\rz$ por la
ecuaci\'{o}n (\ref{distanciaenR}) tiene algunas propiedades b\'{a}sicas
importantes, las cuales se desprenden de las propiedades del valor absoluto de
reales. Algunas de \'{e}stas \'{u}ltimas se enuncian a continuaci\'{o}n.
Suponemos $x,y\in\rz$:
\begin{align}
|x|  &  \geq0\\
|x|=0  &  \Longleftrightarrow x=0\\
|x+y|  &  \leq|x|+|y|\label{triangular1}\\
|xy|  &  =|x||y|
\end{align}
De las anteriores se siguen las siguientes propiedades de la distancia en
$\rz$.%
\index{Distancia!Propiedades de la --}%


\begin{theorem}
Sean $x,y,z$ n\'{u}meros reales, entonces:
\begin{align}
|x-y|  &  \geq0\label{nonegativa}\\
|x-y|=0  &  \Longleftrightarrow x=y\label{nulidad}\\
|x-y|  &  =|y-x|\label{simetria}\\
|x-y|  &  \leq|x-z|+|y-z| \label{triangular}%
\end{align}

\end{theorem}

La desigualdad (\ref{triangular1}) (y tambi\'{e}n su consecuencia
(\ref{triangular})) se acostumbra a denominar desigualdad triangular.

Definimos a continuaci\'{o}n conceptos importantes en la forma\-lizaci\'{o}n
del concepto de l\'{\i}mite, entre ellos el de vecindad.

\begin{definition}
Sean $c,r\in\rz$, con $r>0$. El intervalo abierto \textbf{centrado} en $c$ y
con radio $r$ es el intervalo
\[
]c-r,c+r[.
\]
Todo subconjunto de $\rz$ que contenga un intervalo abierto centrado en $c$ se
denomina una
\index{Vecindad|textbf}%
\textbf{vecindad} de $c$. En particular, todo intervalo abierto centrado en
$c$ es una vecindad de $c$.
\end{definition}
%TODO ---------
Se sigue f\'{a}cilmente que el intervalo con centro en $c$ y radio $r$ es el
conjunto de reales cuya distancia al centro es menor que el radio. En efecto,
tenemos:
\begin{align*}
x\in]c-r,c+r[  &  \Longleftrightarrow c-r<x<c+r\\
&  \Longleftrightarrow-r<x-c<r\\
&  \Longleftrightarrow|x-c|<r
\end{align*}
El intervalo $]a,b[=\{x\in\rz\mid a<x<b\}$ es claramente un intervalo abierto
centrado en $c=\frac{a+b}{2}$ y con radio $r=\frac{b-a}{2}$:
\begin{align*}
c-r  &  =\frac{a+b}{2}-\frac{b-a}{2}\\
&  =a\\
c+r  &  =\frac{a+b}{2}+\frac{b-a}{2}\\
&  =b
\end{align*}
%

\begin{figure}[H]
\centering
\includegraphics[scale=0.8]%
{fig2-1.pdf}%
\caption{Vecindades de $c$}%
\label{tiende2}%
\end{figure}



Podemos ahora formalizar, en t\'{e}rmino de vecindades, la noci\'{o}n de
proxi\-midad entre puntos de la recta real o, si se prefiere, entre
n\'{u}meros reales. Si $c$ es un real fijo, un real $x$ es \textquotedblleft%
\textit{cercano}\textquotedblright\ a $c$ si est\'{a} contenido en una
vecindad \textquotedblleft\textit{peque\~{n}a}\textquotedblright\ de $c$,
m\'{a}s precisamente, si pertenece a un intervalo centrado en $c$ y de radio
peque\~{n}o. As\'{\i}, la expresi\'{o}n \textquotedblleft\emph{la variable
real }$x$\emph{ tiende a }$c$\textquotedblright\ se interpreta diciendo que
$x$ toma valores en intervalos centrados en $c$ con radios cada vez menores
(ver figura \ref{tiende2}).

Terminamos esta secci\'{o}n con el concepto de punto de acumulaci\'{o}n de un
conjunto de n\'{u}meros reales. Intuitivamente hablando, un real $c$ es punto
de acumulaci\'{o}n de un conjunto $A$, si en las vecindades de $c$ se acumulan
en \textquotedblleft gran cantidad\textquotedblright\ elementos de tal conjunto.

\begin{definition}
Sean $A\subseteq\rz$ y $c\in\rz$. $c$ es un%
\index{Punto de acumulaci\'{o}n|textbf}
\underline{punto de} \underline{acumulaci\'{o}n} de $A$ si, y s\'{o}lo si toda
vecindad de $c$ contiene elementos de $A$ distintos de $c$.
\end{definition}

De la definici\'{o}n anterior, se sigue que si $c$ es punto de acumulaci\'{o}n
de $A$ toda vecindad de $c$ contiene infinitos elementos de $A$. En
consecuencia, ning\'{u}n conjunto finito tiene puntos de acumulaci\'{o}n.

\section{L\'{\i}mite de sucesiones}

Consideremos, como ejemplo introductorio, la sucesi\'{o}n $\left\{  \frac
{1}{n}\right\}  _{n\in\nz}$. Intuitivamente es claro que a medida que $n$ toma
valores m\'{a}s grandes, los valores de la sucesi\'{o}n son cada vez m\'{a}s
cercanos a cero. La tabla siguiente lo ilustra.

\begin{center}%
\begin{tabular}
[c]{|l||l|l|l|l|l|l|l|}\hline
$n$ & $1$ & $2$ & $3$ & $\dots$ & $50$ & $10^{6}$ & $10^{10}$\\\hline
$\frac{1}{n}$ & $1$ & $0.5$ & $0.333...$ & $\dots$ & $0.02$ & $0.000001$ &
$0.0000000001$\\\hline
\end{tabular}



\end{center}

El que, como lo sugiere la tabla anterior, $\frac{1}{n}$ tienda a cero
significa que toda vecindad de cero contiene a la \textquotedblleft
mayor\'{\i}a\textquotedblright\ de los t\'{e}rminos de la sucesi\'{o}n.
N\'{o}tese, en efecto, que si conside\-ramos intervalos centrados en cero y
con radios arbitrariamente peque\~{n}os, en cada caso solo un n\'{u}mero
finito de t\'{e}rminos de la sucesi\'{o}n se quedan por fuera del intervalo.
As\'{\i}, por ejemplo, para un radio $r=\frac{1}{5}$, todos los t\'{e}rminos
$\frac{1}{n}$ para $n\geq6$, est\'{a}n en el intervalo con centro en $0$ y
dicho radio $\frac{1}{5}$ (ver figura \ref{vecinocero2}).%


%TODO
\begin{figure}[H]
\centering
\includegraphics[scale=0.8]%
{fig-2-2.pdf}%
\caption{Vecindades de $c$}%
\label{vecinocero2}%
\end{figure}
%TODO 
En general, si $r>0$, existe un n\'{u}mero natural $N$ tal que $N>\frac{1}{r}%
$, por lo que para todo $n\geq N$ se tiene que
\begin{align*}
n>\frac{1}{r}  &  \Longrightarrow\frac{1}{n}<r\\
&  \Longrightarrow\left\vert \frac{1}{n}-0\right\vert <r
\end{align*}
lo que muestra que todos los t\'{e}rminos de la sucesi\'{o}n, a partir del
valor $n=N$, est\'{a}n en el intervalo centrado en $0$ y con radio $r$. Este
hecho se expresa diciendo que
\[
\lim_{n\rightarrow\infty}\frac{1}{n}=0
\]


La definici\'{o}n formal de l\'{\i}mite de una sucesi\'{o}n es como sigue. Sin
p\'{e}rdida de generalidad, como se explic\'{o} en el cap\'{\i}tulo anterior,
supondremos que el dominio de la sucesi\'{o}n es $\nz$.

\begin{definition}
\label{limitesucesion}%
\index{Sucesi\'{o}n!-- convergente}
Sea $\{x_{n}\}_{n\in\nz}$ una sucesi\'{o}n real. Decimos que la sucesi\'{o}n
es \textbf{convergente} si, y s\'{o}lo si existe un real $L$, tal que para
todo $r>0$, existe un n\'{u}mero natural $N$ tal que
\begin{equation}
|x_{n}-L|<r,\mbox{ \ siempre que \ }n\geq N \label{deflimitesucesion}%
\end{equation}
El real $L$ es el l\'{\i}mite de la sucesi\'{o}n%
\index{Sucesi\'{o}n!L\'{\i}mite de una --}
$\{x_{n}\}_{n\in\nz}$. Tal l\'{\i}mite es \'{u}nico (ver ejercicios) y decimos
que $x_{n}$ converge a $L$ y escribiremos
\[
\lim_{n\rightarrow\infty}x_{n}=L.
\]
Tambi\'{e}n es costumbre abreviar la expresi\'{o}n anterior escribiendo
$x_{n}\rightarrow L$.
\end{definition}

Como se anticip\'{o}, la ecuaci\'{o}n \ref{deflimitesucesion} establece que
todo intervalo abierto centrado en (toda vecindad de ) $L$ contiene a todos
los t\'{e}rminos de la sucesi\'{o}n, excepto a un n\'{u}mero finito de ellos.
Una sucesi\'{o}n no convergente se denomina divergente%
\index{Sucesi\'{o}n!-- divergente}%
.

La definici\'{o}n \ref{limitesucesion} no suministra herramientas
pr\'{a}cticas para el c\'{a}lculo de l\'{\i}mites. De hecho, la existencia
misma del l\'{\i}mite, como veremos, podr\'{\i}a establecerse sin conocerlo
espec\'{\i}ficamente. En la pr\'{a}ctica, necesitamos algunas herramientas que
nos permitan determinar tanto la existencia del l\'{\i}mite, como el
l\'{\i}mite mismo. Los teoremas que siguen apuntan en esa direcci\'{o}n,
aunque debe advertirse que, en general, no existen f\'{o}rmulas o
\textquotedblleft reglas m\'{a}gicas\textquotedblright\ que nos permitan
determinar l\'{\i}mites de sucesiones reales en cualquier caso.

Una de las propiedades b\'{a}sicas de una sucesi\'{o}n convergente es el que
su rango es un conjunto acotado%
\index{Conjunto!-- acotado}%
, lo cual significa que est\'{a} contenido en un intervalo de longitud finita.

\begin{theorem}
Sea $\{x_{n}\}_{n\in\nz}$ una sucesi\'{o}n convergente, entonces existe una
constante positiva $K$ tal que para todo $n\in\nz$ se tiene
\begin{equation}
|x_{n}|\leq K \label{acotada}%
\end{equation}

\end{theorem}

\begin{proof}
Sea $L$ el l\'{\i}mite de la sucesi\'{o}n $\{x_{n}\}_{n\in\nz}$. Entonces,
para $r=1$, se tiene que existe un n\'{u}mero natural $N$ tal que
\[
|x_{n}-L|<1,
\]
para $n\geq N$. Se tiene entonces que para $n\geq N$:
\begin{align*}
|x_{n}|  &  =|x_{n}-L+L|\\
&  \leq|x_{n}-L|+|L|\\
&  \leq1+L
\end{align*}
Escogiendo $K=Max\{1+L,|x_{n}|\mid n<N\}$ se tiene que $|x_{n}|\leq K$, para
todo natural $n$.
\end{proof}

El siguiente resultado puede ser de gran utilidad. B\'{a}sicamente, transforma
el problema de la convergencia de una sucesi\'{o}n a un real $L$ en el de la
convergencia a cero. Es una consecuencia inmediata de la definici\'{o}n.

\begin{theorem}
Sean $\{x_{n}\}_{n\in\nz}$ una sucesi\'{o}n y $L$ un n\'{u}mero real.
Entonces:
\begin{equation}
\lim_{n\rightarrow\infty}x_{n}=L\Longleftrightarrow\lim_{n\rightarrow\infty
}|x_{n}-L|=0
\end{equation}

\end{theorem}

El teorema anterior establece un hecho intuitivamente claro: Los valores de la
sucesi\'{o}n se aproximan a $L$ si, y solo si la distancia entre los
t\'{e}rminos de la sucesi\'{o}n y $L$ es pr\'{o}xima a cero. El teorema
siguiente ser\'{a} de gran utilidad en el c\'{a}lculo de l\'{\i}mites de
sucesiones cuando se conocen los l\'{\i}mites de otras sucesiones dadas.

\begin{theorem}%
\index{Sucesiones!Algebra de --}%
\label{algebrasucesiones} Sean $\{x_{n}\}_{n\in\nz},\{y_{n}\}_{n\in\nz}$
sucesiones reales convergentes a $L_{1}$ y $L_{2}$, respectivamente.
Entonces:
\begin{align}
x_{n}+y_{n}  &  \rightarrow L_{1}+L_{2}\label{sumasucesion}\\
x_{n}-y_{n}  &  \rightarrow L_{1}-L_{2}\\
x_{n}y_{n}  &  \rightarrow L_{1}L_{2}\label{productosucesion}\\
cx_{n}  &  \rightarrow cL_{1},c\in\rz
\end{align}

\end{theorem}

Tambi\'{e}n es claro que si una sucesi\'{o}n $\{x_{n}\}$ converge a cero su
producto por una sucesi\'{o}n convergente $\{y_{n}\}$ es tambi\'{e}n
convergente a cero. Sin embargo, una condici\'{o}n m\'{a}s d\'{e}bil que la
convergencia para $\{y_{n}\}$, como el ser acotada, garantiza tambi\'{e}n la
convergencia a cero del producto. Antes demostramos un importante criterio de comparaci\'{o}n.

\begin{theorem}
\label{emparedado}%
\index{Teorema!-- del emparedado}
Sean $\{x_{n}\}_{n\in\nz},\{y_{n}\}_{n\in\nz}$ sucesiones reales. Si existe un
$n_{0}\in\nz$ tal que
\[
|y_{n}|\leq|x_{n}|,\mbox{ \ para todo \ }n\geq n_{0},
\]
entonces $y_{n}\rightarrow0$ si $x_{n}\rightarrow0$.
\end{theorem}

\begin{proof}
Supongamos que $x_{n}$ converge a cero. Entonces para $r>0$ se tiene que
existe $N\in\nz$ tal que $|x_{n}|<r$, siempre que $n\geq N$. Si escogemos
$N_{1}=Max\{N,n_{0}\}$ entonces para $n\geq N_{1}$ tenemos
\[
|y_{n}|\leq|x_{n}|<r
\]
y as\'{\i} $y_{n}\rightarrow0$.
\end{proof}

El teorema anterior es usualmente conocido como un criterio tipo
\textquotedblleft\textit{emparedado}\textquotedblright. La sucesi\'{o}n
$x_{n}$ es una sucesi\'{o}n \textquotedblleft mayorante\textquotedblright\ de
$y_{n}$ que al ser convergente a cero, \textquotedblleft
obliga\textquotedblright\ a $y_{n}$ a converger tambi\'{e}n a cero.

Si $\{y_{n}\}_{n=n_{0}}^{\infty}$ es una sucesi\'{o}n real, hemos visto que
existe una sucesi\'{o}n $\{z_{n}\}_{n\in\nz}$ tal que los t\'{e}rminos de
ambas son iguales, por lo que podr\'{\i}an considerarse iguales. Es claro que
ambas sucesiones son o convergentes al mismo real o divergentes ambas. En tal
sentido, si $\{y_{n}\}_{n\in\nz}$ es una sucesi\'{o}n real tal que para
alg\'{u}n $n_{0}$ se tiene que $y_{n}\neq0$ si $n\geq n_{0}$, entonces podemos
definir la sucesi\'{o}n cociente
\[
\left\{  \frac{x_{n}}{y_{n}}\right\}  _{n=n_{0}}^{\infty}%
\]
la cual produce los mismos valores que la sucesi\'{o}n $\left\{  \frac{x_{n}%
}{z_{n}}\right\}  _{n\in\nz}.$ Supongamos, en particular, que se tiene el
cociente $\frac{1}{y_{n}}$, con $\{y_{n}\}_{n=n_{0}}^{\infty}$ y $y_{n}\neq0$.
Si $y_{n}\rightarrow L$ con $L\neq0$, entonces existe $N\in\nz$ tal que para
todo $n\geq N$ se cumple
\[
|y_{n}-L|<|L/2|.
\]
En consecuencia tenemos $-|L/2|\leq y_{n}-L\leq|L/2|,$ de donde se obtiene que
$|y_{n}|>|L/2|$ y, por lo tanto,
\[
\left\vert \frac{1}{y_{n}}\right\vert \leq\left\vert \frac{2}{L}\right\vert
=K;
\]
lo que significa que $\left\{  \frac{1}{y_{n}}\right\}  _{n=n_{0}}^{\infty}$
es una sucesi\'{o}n acotada. Como consecuencia:
\begin{align*}
\left\vert \frac{1}{y_{n}}-\frac{1}{L}\right\vert  &  =\left\vert \frac
{1}{y_{n}}\left(  \frac{L-y_{n}}{L}\right)  \right\vert \\
&  \leq\frac{K}{L_{1}}|y_{n}-L|\rightarrow0
\end{align*}
Tenemos entonces:

\begin{theorem}
Si $y_{n}\rightarrow L\neq0$, entonces $\dfrac{1}{y_{n}}\rightarrow\dfrac
{1}{L}$
\end{theorem}

Como un corolario tenemos entonces:

\begin{corollary}
Si $x_{n}\to L_{1}, y_{n}\to L_{2}$ y si $L_{2}\neq0$, entonces
\[
\frac{x_{n}}{y_{n}}\to\frac{L_{1}}{L_{2}}.
\]

\end{corollary}

Una sucesi\'{o}n real $\{x_{n}\}_{n\in\nz}$ diverge a $+\infty$, si para todo
real $L>0$, existe un $N\in\nz$ tal que $x_{n}>L$ para todo $n\geq N$. Esto
significa que la sucesi\'{o}n puede tomar va\-lores arbitrariamente grandes.
En tal caso escribimos
\[
\lim_{n\rightarrow\infty}x_{n}=+\infty.
\]


El siguiente teorema permite conocer algunos l\'{\i}mites notables.

\begin{theorem}
\label{enealap} Sea $p$ un n\'{u}mero real. Consideremos la sucesi\'{o}n
$\{x_{n}\}_{n\in\nz}$, definida por $x_{n}=n^{p}$. Entonces:

\begin{enumerate}
\item Si $p>0$, entonces $x_{n}$ es divergente a $+\infty$.

\item Si $p=0$, entonces $x_{n}\rightarrow1$.

\item Si $p<0$, entonces $x_{n}\rightarrow0$.
\end{enumerate}
\end{theorem}

\begin{proof}
\begin{enumerate}
\item Sea $p>0$, entonces para todo real positivo $L$ existe un n\'{u}mero
natural $N$ tal que $N\geq L^{1/p}$ ( si no fuera as\'{\i} $\nz$ ser\'{\i}a un
conjunto acotado superiormente). Se sigue que $N^{p}\geq L$ para alg\'{u}n
natural $N$ y, por tanto, para $n\geq N$, se tiene $n^{p}\geq L$, por lo que
la sucesi\'{o}n $\{n^{p}\}_{n\in\nz}$ diverge a $+\infty$.

\item Para $p=0$ el resultado es claro, pues se tiene la sucesi\'{o}n
constante $1$.

\item Consideremos finalmente el caso $p<0$. Como $-p>0$, la sucesi\'{o}n
$\{n^{-p}\}_{n\in\nz}$ es divergente a $+\infty$. Por lo tanto si $r>0$,
existe un n\'{u}mero natural $N$ tal que $N^{-p}>\frac{1}{r}$. Si $n\in\nz$ es
tal que $n\geq N$, entonces se tiene que
\[
n^{-p}\geq N^{-p}>\frac{1}{r}.
\]
As\'{\i}, para todo $n\geq N$:
\[
|n^{p}|=(n^{-p})^{-1}<r
\]
En consecuencia $n^{p}\rightarrow0$ si $p<0$.
\end{enumerate}
\end{proof}

En la demostraci\'{o}n del teorema siguiente se usa el Teorema del binomio%
\index{Teorema!-- del binomio}%
: Si $a,b\in\rz$ y $n\in\nz$, entonces:
\begin{equation}
(a+b)^{n}=\sum_{k=0}^{n}\dbinom{n}{k}a^{k}b^{n-k},
\end{equation}
donde $\dbinom{n}{k}=\dfrac{n!}{k!(n-k)!}$ es el coeficiente binomial. En
particular, si $a,b\geq0$, entonces para cada $k=0,\dots,n$ y $m\leq n$ se
tienen
\begin{align}
(a+b)^{n}  &  \geq\dbinom{n}{k}a^{k}b^{n-k}\\
(a+b)^{n}  &  \geq\sum_{k^{=}0}^{m}\dbinom{n}{k}a^{k}b^{n-k}%
\end{align}


\begin{theorem}
\label{raizndep} \hfil
\begin{enumerate}
\item Si $p>0$, entonces $\sqrt[n]{p}\rightarrow1$.

\item $\sqrt[n]{n}\rightarrow1$.

\item Sea $p>0$, entonces
\[
p^{n}\rightarrow\left\{
\begin{tabular}
[c]{ll}%
$+\infty$ & , si $p>1$\\
$0$ & , si $p<1$\\
$1$ & , si $p=1$%
\end{tabular}
\right.
\]


\item Si $x_{n}\geq0$ y $x_{n}\rightarrow L$, entonces $L\geq0$ y
$\sqrt[k]{x_{n}}\rightarrow\sqrt[k]{L}$.
\end{enumerate}
\end{theorem}

\begin{proof}
\footnote{La mayor parte de la demostraci\'{o}n de este teorema es tomada de
\cite{Rudin}} \hfill

\begin{enumerate}
\item Sea $x_{n}=\sqrt[n]{p}-1$. Si $p>1$, entonces $x_{n}>0$ y
\[
p=(1+x_{n})^{n}\geq1+nx_{n},
\]
de donde se sigue que $0\leq|x_{n}|\leq\frac{p-1}{n}$ y, por lo tanto,
$|x_{n}|\rightarrow0$, o sea $\sqrt[n]{p}\rightarrow1$. Si $p=1$, entonces
$x_{n}$ es la sucesi\'{o}n constante nula y el resultado se sigue. Para
$0<p<1$, se tiene $p^{-1}>1$, por lo que
\[
\lim_{n\rightarrow\infty}\sqrt[n]{p}=\lim_{n\rightarrow\infty}\frac
{1}{\sqrt[n]{p^{-1}}}=\frac{1}{1}=1.
\]


\item Como en la prueba anterior, sea $x_{n}=\sqrt[n]{n}-1\geq0$, entonces
\[
n=(1+x_{n})^{n}\geq\frac{n(n-1)}{2}x_{n}^{2},
\]
de donde
\[
0\leq x_{n}\leq\frac{\sqrt{2}}{\sqrt{n-1}}.
\]
Como
\[
\left\{  \frac{1}{n-1}\right\}  _{n=2}^{\infty}=\left\{  \frac{1}{n}\right\}
_{n\in\nz}%
\]
entonces $\sqrt{2}/\sqrt{n-1}\rightarrow0$ y, en consecuencia, $x_{n}%
\rightarrow0$.

\item Si $p>1$, entonces $p-1>0$. Supongamos ahora $L>0$, entonces
$\lim\limits_{n\rightarrow\infty}\sqrt[n]{L}=1$ y existe un n\'{u}mero natural
$N$ tal que
\[
|\sqrt[n]{L}-1|<p-1,
\]
siempre que $n\geq N$, de donde se sigue que para todo $n\geq N$
\[
0<\sqrt[n]{L}<p\ ;
\]
es decir, $L<p^{n}$. As\'{\i}, $p^{n}\rightarrow+\infty$. Para $p=1$, el
resultado es trivial. Si $0<p<1$, entonces $p^{-1}>1$ y $(p^{-1})^{n}$ diverge
a $+\infty$, por lo que dado un real positivo $r$, existe $n\in\nz$ tal que
\[
0<\frac{1}{r}<p^{-n},
\]
para todo $n\in\nz$. As\'{\i}, para $n\geq N$ se tiene $0<p^{n}<r$. Por lo
tanto $p^{n}\rightarrow0$.

\item Se deja de ejercicio.
\end{enumerate}
\end{proof}

\section{La serie geom\'{e}trica}%

\index{Serie!-- geom\'{e}trica}%
La sucesi\'{o}n $\{x^{n}\}_{n\in\nz_{0}}$ para $x\in\rz$, genera la serie
\[
\sum_{n=0}^{\infty}x^{n}.
\]
Recordemos que tal serie es la sucesi\'{o}n de sumas parciales
\[
1,1+x,1+x+x^{2},\dots,\sum_{k=0}^{n}x^{k},\dots
\]
Del teorema \ref{raizndep} se sigue que
\begin{equation}
x^{n}\rightarrow0\mbox{ \ si \ }|x|<1 \label{geometricaconvergente}%
\end{equation}
Es claro, adem\'{a}s, que $1^{n}\rightarrow1$ y que $(-1)^{n}$ es divergente.
Para $|x|>1$, la sucesi\'{o}n $\{x^{n}\}_{n\in\nz_{0}}$ no es acotada, pues la
sucesi\'{o}n $|x^{n}|$ diverge a $+\infty$. Por lo que $\{x_{n}\}_{n\in
\nz_{0}}$ es divergente \textquestiondown C\'{o}mo son las series asociadas?

Consideremos, para $x\neq1$ la identidad
\[
1-x^{n+1}=(1-x)(1+x+x^{2}+\dots+x^{n})=(1-x)\sum_{k=0}^{n}x^{k}.
\]
Despejando la $n-$\'{e}sima suma parcial de la sucesi\'{o}n$\{x^{n}%
\}_{n\in\nz}$ tenemos:
\begin{align*}
\sum_{k=0}^{n}x^{k}  &  =1+x+x^{2}+\dots+x^{n}\\
&  =\frac{1-x^{n+1}}{1-x}\\
&  =\frac{1}{1-x}-\frac{x}{1-x}x^{n}%
\end{align*}
Es claro que la serie converge si, y solo si lo hace la sucesi\'{o}n $\frac
{x}{1-x}x^{n}$. Esta \'{u}ltima, a su vez, converge si lo hace $x^{n}$, pues
$\frac{x}{1-x}$ es constante. Por lo tanto, tenemos

\begin{corollary}
\label{seriegeometrica} $\sum_{k=0}^{n} x^{k}$ converge si, y solo si $\vert x
\vert<1$. En tal caso
\begin{equation}
\label{convergeometrica}\sum_{n=0}^{\infty}x^{n}=\lim_{n\to\infty}
(1+x+\dots+x^{n})=\frac{1}{1-x}%
\end{equation}

\end{corollary}

As\'{\i}, por ejemplo,
\begin{align*}
\lim_{n\rightarrow\infty}\sum_{k=0}^{n}\left(  \frac{1}{2}\right)  ^{k}  &
=1+\frac{1}{2}+\frac{1}{4}+\frac{1}{8}+\dots\\
&  =\frac{1}{1-\frac{1}{2}}\\
&  =2\\
\sum_{n=0}^{\infty}\left(  -\frac{1}{3}\right)  ^{n}  &  =\frac{1}{1+(1/3)}\\
&  =\frac{3}{4}%
\end{align*}


\section*{El n\'{u}mero de Euler}

Una sucesi\'{o}n $\{x_{n}\}_{n\in\nz}$ de n\'{u}meros reales se denomina
creciente%
\index{Sucesi\'{o}n!-- creciente}
si $x_{n}\geq x_{m}$, siempre que $x_{n}\geq x_{m}$. Si $\{-x_{n}\}_{n\in\nz}$
es creciente, decimos que $\{x_{n}\}_{n\in\nz}$ es decreciente%
\index{Sucesi\'{o}n!-- decreciente}%
. Una sucesi\'{o}n
\index{Sucesi\'{o}n!-- mon\'{o}tona}%
es mon\'{o}tona si es o creciente o decreciente. Sucesiones mon\'{o}tonas son
convergentes si, y solo si son acotadas, como lo establece el siguiente
teorema. Lo demostramos solo para sucesiones crecientes.

\begin{theorem}
Una sucesi\'{o}n mon\'{o}tona%
\index{Sucesi\'{o}n!-- mon\'{o}tona!convergente|textit}
de n\'{u}meros reales converge si, y solo si es acotada.
\end{theorem}

\begin{proof}
Sea $\{x_{n}\}_{n\in\nz}$ creciente y acotada. Entonces, existe una cota
superior m\'{\i}nima, $L$, del rango de la sucesi\'{o}n. Es decir, $x_{n}\leq
L$ para todo $n\in\nz_{0}$ y para todo $K$ tal que $x_{n}\leq K$ para todo
$n$, se tiene que $L\leq K$.\newline Supongamos ahora que $r>0$, entonces
$L-r<L$ y, por lo tanto, no puede ser una cota superior para el rango de la
sucesi\'{o}n. As\'{\i}, existe $N\in\nz$, tal que
\[
L-r<x_{N}\leq L<L+r,
\]
de donde se sigue que para todo $n\geq N$
\[
L-r<x_{n}<L+r\ ;
\]
es decir, $|x_{n}-L|<r$, para todo $n\geq N$. Por lo tanto
\[
\lim_{n\rightarrow\infty}x_{n}=L.
\]

\end{proof}

Como una aplicaci\'{o}n importante del teorema anterior, consideremos la
sucesi\'{o}n
\[
\left\{  \frac{1}{n!}\right\}  _{n\in\nz_{0}}=\left\{  1,1,\frac{1}{2!}%
,\frac{1}{3!},\dots,\frac{1}{1\ast2\ast3\ast\dots\ast n},\dots\right\}  .
\]
Es claro que la serie asociada a la sucesi\'{o}n anterior es una serie de
t\'{e}rminos positivos y es, por lo tanto, creciente. Mostremos que es
acotada, mostrando que lo es superiormente. En efecto:
\begin{align*}
\frac{1}{0!}  &  =1\\
\frac{1}{1!}  &  =\frac{1}{2^{0}}\\
\frac{1}{2!}  &  \leq\frac{1}{2^{1}}\\
\frac{1}{3!}  &  \leq\frac{1}{2^{2}}\\
&
\begin{tabular}
[c]{lll}%
$\vdots$ & $\vdots$ &
\end{tabular}
\\
\frac{1}{n!}  &  \leq\frac{1}{2^{n-1}},\ n\geq1
\end{align*}
por lo que
\[
\sum_{k=0}^{n}\frac{1}{n!}\leq1+\sum_{k=0}^{n-1}\frac{1}{2^{k}}.
\]
Como la sucesi\'{o}n $\left\{  \frac{1}{2^{n}}\right\}  _{n\in\nz_{0}}$ es
convergente a $2$, la serie considerada est\'{a} acotada superiormente por $3$
y es, por tanto, convergente. El valor del l\'{\i}mite es as\'{\i} un
n\'{u}mero positivo menor que $3$ y mayor que $2$. Es denominado el n\'{u}mero
de Euler y se simbolizar\'{a} por $e$.

\begin{definition}%
\index{Euler!N\'{u}mero de --}%
\label{numeroeuler}
\[
e=\lim_{n\rightarrow\infty}\left(  1+1+\frac{1}{2}+\dots+\frac{1}{n!}\right)
=\sum_{n=0}^{\infty}\frac{1}{n!}%
\]

\end{definition}

Puede obtenerse un estimado del n\'{u}mero de Euler escogiendo valores grandes
para $n$ en la definici\'{o}n \ref{numeroeuler}. La tabla siguiente muestra
algunos con precisi\'{o}n de al menos veinte d\'{\i}gitos decimales.
Comp\'{a}rense los valores obtenidos con el valor de $e$ con cuarenta
d\'{\i}gitos de precisi\'{o}n:
\[
e\approx2.718281828459045235360287471352662497757.
\]%
\[%
\begin{tabular}
[c]{|c||c|}\hline
n & $1+1+\frac{1}{2!}+\dots+\frac{1}{n!}$\\\hline
3 & 2.6666666666666666667\\
4 & 2.7083333333333333333\\
5 & 2.7166666666666666667\\
6 & 2.7180555555555555556\\
7 & 2.7182539682539682540\\
8 & 2.7182787698412698413\\
9 & 2.7182815255731922399\\
10 & 2.7182818011463844797\\
20 & 2.7182818284590452353\\\hline
\end{tabular}
\
\]


\section{L\'{\i}mites de funciones reales de varia\-ble real}

La definici\'{o}n del l\'{\i}mite de una sucesi\'{o}n puede extenderse para
funciones cuyo dominio es un intervalo de la forma $\left[  c,+\infty\right[
$. Ahora, el dominio no es un conjunto \textquotedblleft
discreto\textquotedblright, como lo era $\nz$ en el caso de sucesiones. En ese
sentido, la variable real $x$ puede tomar valores cada vez m\'{a}s grandes en
el intervalo $\left[  0,+\infty\right[  $.

\begin{definition}
Sean $c$ un n\'{u}mero real y $f:\left[  c,\infty\right[  \longrightarrow\rz$
una funci\'{o}n. Si $L\in\rz$, decimos que $f$ converge a $L$, cuando
$x\rightarrow+\infty$ si,y solo si para cada real $r>0$, existe un real $N>0$,
tal que $|f(x)-L|<r$, siempre que $x\geq N$. En tal caso escribimos%
\index{L\'{\i}mite!c@-- cuando $x\rightarrow+\infty$}
\[
\lim_{x\rightarrow+\infty}f(x)=L.
\]

\end{definition}

Por ejemplo, para $f(x)=\frac{x^{2}}{1+x^{2}}$ se observa en el comportamiento
de la gr\'{a}fica de la funci\'{o}n (ver figura \ref{xpoeealax1}), que cuando
$x$ se acerca a $\pm\infty$ la funci\'{o}n toma valores cada vez cercanos a
$1.$ Es decir $\lim\limits_{x\rightarrow\infty}\frac{x^{2}}{1+x^{2}}=1$ y
$\lim\limits_{x\rightarrow-\infty}\frac{x^{2}}{1+x^{2}}=1$.

\begin{figure}[H]
\centering
\includegraphics[scale=0.6]%
{fig-2-3.pdf}%
\caption{Gr\'{a}fica de $f\left(  x\right)  =\frac{x^{2}}{1+x^{2}}$}%
\label{xpoeealax1}%
\end{figure}
   
%TODO -------------------------------------   
    
N\'{o}tese la similitud de la definici\'{o}n anterior, con la definici\'{o}n
del l\'{\i}mite de una sucesi\'{o}n convergente. La diferencia b\'{a}sica
est\'{a} en que la variable independiente toma ahora valores no solamente
enteros positivos. M\'{a}s a\'{u}n, si $f$ se restringe al conjunto de valores
enteros positivos
\[
J=\{n\in\nz\mid n\geq c\}
\]
entonces obtenemos una sucesi\'{o}n $\{f(n)\}_{n\in J}$. Es claro, entonces que:

\begin{theorem}
Si $\lim\limits_{x\rightarrow+\infty}f(x)=L$, entonces $\lim
\limits_{n\rightarrow\infty}f(n)=L$
\end{theorem}

El rec\'{\i}proco del teorema anterior es claramente falso: Una sucesi\'{o}n
real puede extenderse arbitrariamente al intervalo $\left[  0,+\infty\right[
$ por lo que no necesariamente la convergencia de la sucesi\'{o}n garantiza la
convergencia de la funci\'{o}n con variable real. Considere, por ejemplo, la
funci\'{o}n $f$ definida a continuaci\'{o}n:
\[
f(x)=\left\{
\begin{tabular}
[c]{ll}%
$\frac{1}{x}$ & , si $x\in\nz$\\
$x$ & , en cualquier otro caso
\end{tabular}
\ \ \right.
\]
Es claro que aunque $\frac{1}{x}\rightarrow0$, cuando $x\in\nz$, la
funci\'{o}n $f$ no converge a cero, pues puede tomar valores arbitrariamente
grandes, cuando $x$ aumenta. Por su parte, la funci\'{o}n $g$ definida por
$g(x)=\frac{1}{x}$, $x\in\left]  0,+\infty\right[  ,g(0)=0$ converge a cero
cuando $x\rightarrow+\infty$. En efecto, como en el caso de la sucesi\'{o}n
definida por $g$, para un real $r>0$, existe $N\in\nz$, tal que $N>\frac{1}%
{r}$, por lo que para todo real $x\geq N$, se tiene:
\[
\left\vert \frac{1}{x}\right\vert \leq\frac{1}{N}<r,
\]
por lo que $\lim_{x\rightarrow+\infty}\frac{1}{x}=0$. N\'{o}tese, sin embargo,
que las funci\'{o}n $f$ no tiene rango acotado, pues en las vecindades de cero
$\frac{1}{x}$ toma valores arbitrariamente grandes. Existe, sin embargo, como
en el caso de sucesiones, una restricci\'{o}n acotada de la funci\'{o}n. En
efecto, si $\lim_{x\rightarrow+\infty}f(x)=L$, entonces, existe un n\'{u}mero
real, $N$ tal que
\[
|f(x)-L|<1,\mbox{ \ siempre que \ }x\geq N,
\]
lo que muestra que $f$ es acotada en\ $\left[  N,+\infty\right[  $.
Condiciones de suficiencia para la acotaci\'{o}n y para la convergencia de
extensiones de sucesiones convergentes a intervalos no acotados pueden ser
dadas por la \textquotedblleft continuidad\textquotedblright\ de dichas
extensiones, condici\'{o}n que ser\'{a} definida posteriormente.

Sea $f:\left]  -\infty,c\right]  \longrightarrow\rz$ una funci\'{o}n, entonces
la funci\'{o}n definida por $g(x)=f(-x)$ tiene dominio en $\left[
-c,+\infty\right[  $ y le son aplicables las definiciones de esta secci\'{o}n.
Tenemos la siguiente definici\'{o}n:

\begin{definition}
$\lim\limits_{x\rightarrow-\infty}f(x)=L$ si, y solo si $\lim
\limits_{x\rightarrow+\infty}g(x)=L$.
\end{definition}

Una funci\'{o}n real $f$ con dominio $\left[  c,+\infty\right[  $
(respectivamente, $\left]  -\infty,c\right]  $) se dice divergente a $+\infty$
cuando $x\rightarrow+\infty$ (resp, $x\rightarrow-\infty$) si, para todo real
$L>0$, existe un real $N>0$ (resp. $N<0$) tal que $f(x)\geq L$, siempre que
$x\geq N$ (resp $x\leq N$). La funci\'{o}n $f$ se dice divergente a $-\infty$
si $-f$ diverge a $+\infty$.

Podemos tambi\'{e}n considerar el comportamiento de una funci\'{o}n real en
las vecindades de un real \textquotedblleft finito\textquotedblright\ $c$.
Consideraremos primero el comportamiento de una funci\'{o}n en vecindades%
\index{Vecindad!-- lateral}
\textquotedblleft laterales\textquotedblright\ de $c$.

\begin{definition}
Sea $f:\left]  c,d\right[  \longrightarrow\rz$ una funci\'{o}n real. Si
$L\in\rz$, entonces escribiremos
\[
\lim_{x\rightarrow c^{+}}f(x)=L
\]
si, y solo si para todo $r>0$, existe $\delta>0$ tal que
\[
|f(x)-L|<r
\]
siempre que $x\in\left]  c,d\right[  $ y \ $x-c<\delta$.
\end{definition}

En la definici\'{o}n anterior $L$ es el l\'{\i}mite de $f$ cuando $x$ tiende a
$c$ por la derecha. La definici\'{o}n establece que para todo intervalo
centrado en $L$ (con radio $r$) existe una vecindad%
\index{L\'{\i}mite!-- lateral derecho}
\index{L\'{\i}mite!-- lateral izquierdo}%
a la derecha de $c$ (con longitud $\delta$) tal que todas las im\'{a}genes de
los elementos de \'{e}sta \'{u}ltima est\'{a}n contenidos en el intervalo
centrado en $L$. Dicho en t\'{e}rminos de distancia significa que $f(x)$ puede
aproximarse a $L$ \textquotedblleft tanto como se quiera\textquotedblright%
\ (seg\'{u}n $r$) escogiendo $x$ \textquotedblleft suficientemente
cerca\textquotedblright\ por la derecha de $c$ (seg\'{u}n $\delta$). Ver
figura \ref{limitederecha}%


\begin{figure}[H]
\centering
\includegraphics[scale=0.6]%
{fig-2-4.pdf}%
\caption{L\'{\i}mites unilaterales}%
\label{limitederecha}%
\end{figure}

%TODO 
De manera similar tenemos la definici\'{o}n de l\'{\i}mite por la izquierda.

\begin{definition}
Sea $f:\left]  b,c\right[  \longrightarrow\rz$ una funci\'{o}n. Si $L\in\rz$,
entonces
\[
\lim_{x\rightarrow c^{-}}f(x)=L
\]
si, y solo si para todo $r>0$, existe $\delta>0$, tal que
\[
|f(x)-L|<r
\]
siempre que $x\in\left]  b,c\right[  $ y \ $c-x<\delta$.%
\index{L\'{\i}mite!Definici\'{o}n de --}%

\end{definition}

As\'{\i}, la definici\'{o}n establece que para todo intervalo centrado en $L$,
existe una vecindad a la izquierda de $c$, tal que las im\'{a}genes de los
elementos en esta \'{u}ltima est\'{a}n contenidas en dicho intervalo. En la
figura \ref{limitederecha}, $K$ es el l\'{\i}mite de $f$ cuando $x$ tiende a
$c$, por la izquierda.

Si $f:\left]  b,c\right[  \cup\left]  c,d\right[  \longrightarrow\rz$ es una
funci\'{o}n real, entonces escribiremos
\[
\lim_{x\rightarrow c}f(x)=L
\]
si, y solo si, los l\'{\i}mites unilaterales de $f$ cuando $x$ tiende a $c$
existen ambos y son iguales a $L$. Es decir, para todo $r>0$, existe
$\delta>0$ tal que $|f(x)-L|<r$, siempre que $0<|x-c|<\delta$. As\'{\i}, para
todo intervalo centrado en $L$, existe un intervalo centrado en $c$, tal que
las im\'{a}genes de todo elemento en \'{e}ste \'{u}ltimo est\'{a}n contenidas
en el primero.

Debe notarse que en las definiciones anteriores la hip\'{o}tesis no incluye el
que $c$ est\'{e} en el dominio de la funci\'{o}n, pero \'{e}sta debe estar
definida en vecindades de $c$. En particular, si $f$ y $g$ son funciones tales
que $f=g$ en una vecindad de $c$ (unilateral o bilateral, seg\'{u}n el caso),
entonces $\lim_{x\rightarrow c}f(x)=\lim_{x\rightarrow c}g(x)$, si uno de los
dos l\'{\i}mites existe.

\begin{exercise}
Considere, la funci\'{o}n $f$, definida por
\[
f(x)=\left\{
\begin{tabular}
[c]{ll}%
$\frac{x^{3}-1}{x-1}$ & ,si $x\neq1$\\
$0$ & ,si $x=1$%
\end{tabular}
\ \right.
\]
Es claro que $\operatorname*{Dom}(f)=\rz$ y que para $x\neq1$
\[
f(x)=\frac{x^{3}-1}{x-1}=\frac{(x-1)(x^{2}+x+1)}{x-1}=x^{2}+x+1.
\]
Si consideramos la funci\'{o}n $g$, dada por $g(x)=x^{2}+x+1$, $x\in\rz$, es
claro entonces que
\[
\lim_{x\rightarrow1}f(x)=\lim_{x\rightarrow1}g(x)
\]
El \'{u}ltimo l\'{\i}mite, como se establecer\'{a} despu\'{e}s, es $3$,
as\'{\i} que
\[
\lim_{x\rightarrow1}f(x)=3.
\]

\end{exercise}

Finalmente, extendemos la definici\'{o}n de divergencia a $+\infty$ (o
$-\infty$).\newline Escribiremos $\lim_{x\rightarrow c}f(x)=+\infty$ si, para
todo real positivo%
\index{L\'{\i}mites!-- al infinito}
$L>0$, existe $\delta>0$, tal que $f(x)>L$, siempre que $|x-c|<\delta$.
Tenemos tambi\'{e}n
\[
\lim_{x\rightarrow c}f(x)=-\infty\ \Longleftrightarrow\ \lim_{x\rightarrow
c}(-f(x))=+\infty.
\]
Por ejemplo, consideremos $f(x)=\dfrac{1}{x^{2}}.$ Si analizamos el
comportamiento de la funci\'{o}n ( Ver figura \ref{cap2graf11}), se observa
claramente que $\lim\limits_{x\rightarrow0}\dfrac{1}{x^{2}}=\infty.$ Es decir,
cuando $x$ se acerca a $0,$ la funci\'{o}n toma valores cada vez m\'{a}s grandes.

\begin{figure}[H]
\centering
\includegraphics[scale=0.45]%
{fig-2-5.pdf}%
\caption{Gr\'{a}fica de $f\left(  x\right)  =\frac{1}{x^{2}}.$}%
\label{cap2graf11}%
\end{figure}

%TODO
Los teoremas relativos al Algebra de l\'{\i}mites de sucesiones, as\'{\i} como
el criterio del emparedado para la convergencia a cero pueden ser extendidos
sin dificultades mayores a l\'{\i}mites como los definidos anteriormentes.
Para abreviar, introducimos el denominado sistema ampliado de los n\'{u}meros
reales:%
\index{c@$\rz^{\ast}$|textbf}
\begin{equation}
\rz^{\ast}=[-\infty,+\infty]=\rz\cup\{-\infty,+\infty\}
\end{equation}
Un intervalo $\left]  c,+\infty\right[  $ es denominado una vecindad de
$+\infty$. De igual manera, un intervalo\ $\left]  -\infty,c\right[  $ es una
vecindad de $-\infty$. As\'{\i}, al indicar que una variable real tiende a
$+\infty$ ( resp. $-\infty$) queremos decir que dicha variable toma valores en
una vecindad de $+\infty$ (resp. $-\infty$). Un primer teorema importante en
el \'{a}lgebra de l\'{\i}mites es el siguiente. Su demostraci\'{o}n exhaustiva
se propone como ejercicio. Un resultado similar se deduce para funciones
divergentes a $-\infty$ y se deja al lector. De igual manera, el caso en el
que los l\'{\i}mites considerados son unilaterales.

\begin{theorem}
\label{infinitomasinfinito}Sean $f$ y $g$ funciones reales. Entonces:%
\index{L\'{\i}mites!Algebra de --}%
\begin{enumerate}
\item Si $f$ y $g$ divergen a $+\infty$, cuando $x\rightarrow c\in\rz^{\ast}$
entonces $f+g$ y $fg$ divergen a $+\infty$ cuando $x\rightarrow c$.

\item Si $f$ diverge a $+\infty$, cuando $x\rightarrow c\in\rz^{\ast}$, y
$\lim_{x\rightarrow c}g(x)=L\in\rz-\{0\}$, entonces
\begin{enumerate}
\item $fg$ diverge a $+\infty$ cuando $x\rightarrow c$, si $L>0$.
\item $fg$ diverge a $-\infty$ cuando $x\rightarrow c$, si $L<0$.
\end{enumerate}
\end{enumerate}
\end{theorem}

El teorema anterior justifica definir sobre $\rz^{\ast}$, las siguientes
relaciones y operaciones, adem\'{a}s de las ya definidas sobre $\rz$:

\begin{enumerate}
\item Para todo real $x$:
\begin{align}
-\infty<x  &  <+\infty\\
x+(+\infty)  &  =+\infty\\
x+(-\infty)  &  =-\infty\\
x-(+\infty)  &  =-\infty\\
x-(-\infty)  &  =+\infty\\
x(+\infty)  &  =\left\{
\begin{array}
[c]{cc}%
+\infty & ,\text{ si }x>0\\
-\infty & ,\text{ si }x<0
\end{array}
\right. \\
x(-\infty)  &  =\left\{
\begin{array}
[c]{cc}%
+\infty & ,\text{ si }x<0\\
-\infty & ,\text{ si }x>0
\end{array}
\right. \\
\frac{x}{\pm\infty}  &  =0
\end{align}


\item Se tienen tambi\'{e}n:
\begin{align}
+\infty+(+\infty)  &  =+\infty\\
-\infty+(-\infty)  &  =-\infty\\
(+\infty)(+\infty)  &  =+\infty\\
(+\infty)(-\infty)  &  =-\infty\\
(-\infty)(-\infty)  &  =+\infty
\end{align}

\end{enumerate}

La definici\'{o}n de l\'{\i}mite puede darse ahora en general para funciones
reales definidas en vecindades de un punto de $\rz^{\ast}$:\newline Sea $f$
definida en una vecindad de $c\in\rz^{\ast}$, excepto posiblemente en $c$, si
$L\in\rz^{\ast}$, se tiene
\[
\lim_{x\rightarrow c}f(x)=L
\]
si, y solo si para toda vecindad de $L$ existe una vecindad de $c$ tal que
para todo elemento de \'{e}sta \'{u}ltima, en el dominio de $f$, su im\'{a}gen
est\'{a} en la vecindad de $L$.

\section{Teoremas de l\'{\i}mite.}

En esta secci\'{o}n se enuncian, sin demostraci\'{o}n, teoremas importantes
del \'{a}lgebra de l\'{\i}mites de funciones reales .

\begin{theorem}
\label{a1}Si $\lim\limits_{x\rightarrow p}f(x)=L_{1}\ y\ \lim
\limits_{x\rightarrow p}g(x)=L_{2}$. Entonces

\begin{enumerate}
\item $\lim\limits_{x\rightarrow p}\left(  f(x)+g(x)\right)  =\lim
\limits_{x\rightarrow p}f(x)+\lim\limits_{x\rightarrow p}g(x)=L_{1}+L_{2}.$

\item $\lim\limits_{x\rightarrow p}\left(  f(x)-g(x)\right)  =\lim
\limits_{x\rightarrow p}f(x)-\lim\limits_{x\rightarrow p}g(x)=L_{1}-L_{2}.$

\item $\lim\limits_{x\rightarrow p}\left(  f(x)g(x)\right)  =\lim
\limits_{x\rightarrow p}f(x)\lim\limits_{x\rightarrow p}g(x)=L_{1}L_{2}.$

\item $\lim\limits_{x\rightarrow p}\left(  cg(x)\right)  =c\lim
\limits_{x\rightarrow p}g(x)=cL_{2}$ donde $c$ es una constante.

\item $\lim\limits_{x\rightarrow p}\left(  \dfrac{f(x)}{g(x)}\right)
=\dfrac{\lim\limits_{x\rightarrow p}f(x)}{\lim\limits_{x\rightarrow p}%
g(x)}=\dfrac{L_{1}}{L_{2}},\ $si $\ L_{2}$ $\neq0.$

\item Si $Q\left(  x\right)  $ es un polinomio, $\lim\limits_{x\rightarrow
p}Q(x)=Q(p).$
\end{enumerate}
\end{theorem}

\begin{theorem}
\label{limitefuncioncompuesta}
\index{L\'{\i}mite!-- de una funci\'{o}n compuesta}%
Sea $f$ una funci\'{o}n definida en una vecindad de $c\in\rz^{\ast}$ y
supongamos que
\[
\lim_{x\rightarrow c}f(x)=L.
\]
Si $g$ est\'{a} definida en una vecindad de $L$ y
\[
\lim_{u\rightarrow L}g(u)=M.
\]
Entonces $g\circ f$ est\'{a} definida en una vecindad de $c$ y
\[
\lim_{x\rightarrow c}(g\circ f)(x)=\lim_{f(x)\rightarrow L}g(f(x))=M.
\]

\end{theorem}

\begin{corollary}
\label{a2}Si $\lim\limits_{x\rightarrow p}f(x)=L$ entonces
\end{corollary}

\begin{enumerate}
\item $\lim\limits_{x\rightarrow p}[f(x)]^{n}=[\lim\limits_{x\rightarrow
p}f(x)]^{n}=L^{n}$

\item $\lim\limits_{x\rightarrow p}\sqrt[n]{f(x)}=\sqrt[n]{\lim
\limits_{x\rightarrow p}f(x)}=\sqrt[n]{L}$ \ donde $n$ es un entero positivo
impar. \newline( la proposici\'{o}n anterior es v\'{a}lida para $n$ par si
suponemos que $L>0)$
\end{enumerate}

\begin{theorem}
{\bf Teorema del Emparedado: \  }                           %
\index{Teorema!-- del emparedado}%
\label{a3} Sea $\varepsilon>0$. Supongamos que $f,g,h$ estan definidos en una
vecindad $\left]  p-\varepsilon,p+\varepsilon\right[  $ de $p$, y adem\'{a}s
que $f(x)\leq h(x)\leq g(x)$ y $\lim\limits_{x\rightarrow p}f(x)=\lim
\limits_{x\rightarrow p}g(x)=L.$ Entonces $\lim\limits_{x\rightarrow
p}h(x)=L.$
\end{theorem}

\begin{theorem}
{\bf L\'{\i}mites trigonom\'{e}tricos: \  }%
\index{L\'{\i}mites!-- trigonom\'{e}tricos}%
\label{a4}


\begin{enumerate}
\item $\lim\limits_{x\rightarrow0}\dfrac{\operatorname{sen}x}{x}=1$

\item $\lim\limits_{x\rightarrow0}\dfrac{1-\cos x}{x}=0$
\end{enumerate}

\begin{theorem}%
\index{L\'{\i}mite!Teoremas principales de --}%
\label{a7}
\end{theorem}

\begin{enumerate}
\item Sea $p(x)=a_{n}x^{n}+a_{n-1}x^{n-1}+\ldots+a_{1}x+a_{0}$ con $a_{n}%
\neq0$ un polinomio de grado $n.$ Entonces%
\[
\lim_{x\rightarrow\infty}p(x)=\left\{
\begin{tabular}
[c]{cc}%
$\infty$ & si$\ a_{n}>0$\\
$-\infty$ & si$\ a_{n}<0.$%
\end{tabular}
\ \ \ \ \right.
\]


\item $\lim\limits_{x\rightarrow\pm\infty}\dfrac{1}{x^{n}}=0,\ \forall
n\in\nz.$

\item $\lim\limits_{x\rightarrow0^{+}}\dfrac{1}{x}=\infty$\ y\ $\lim
\limits_{x\rightarrow0^{-}}\dfrac{1}{x}=-\infty.$

\item Si $p(x)=\sum\limits_{k=1}^{m}a_{k}x^{k}$ y $q(x)=\sum\limits_{k=1}%
^{n}b_{k}x^{k}$ son polinomios, de grados $m$ y $n$ respectivamente, y si

\begin{enumerate}
\item $p\left(  a\right)  \neq0,q\left(  a\right)  \neq0.$ Entonces
\[
\lim\limits_{x\rightarrow a}\frac{p\left(  x\right)  }{q(x)}=\frac{p\left(
a\right)  }{q\left(  a\right)  }%
\]


\item $p\left(  a\right)  =0,q\left(  a\right)  \neq0.$ Entonces
\[
\lim\limits_{x\rightarrow a}\frac{p\left(  x\right)  }{q(x)}=0
\]


\item $p\left(  a\right)  \neq0,q\left(  a\right)  =0.$ Entonces
\[
\lim\limits_{x\rightarrow a^{+}}\frac{p\left(  x\right)  }{q(x)}=\pm
\infty\wedge\lim\limits_{x\rightarrow a^{-}}\frac{p\left(  x\right)  }%
{q(x)}=\pm\infty
\]


\item $p\left(  a\right)  =0,q\left(  a\right)  =0$. Donde $a$ es un cero de
multiplicidad $k\leq m$ para $p\left(  x\right)  $ y de multiplicidad $l\leq
n$ para $q\left(  x\right)  $ (Es decir, existen polinomios $p_{k}\left(
x\right)  $ de grado $m-k,q_{l}\left(  x\right)  $ de grado $n-l,$ tales que
$p_{k}\left(  a\right)  \neq0$ y $q_{l}\left(  a\right)  \neq0$ y $p\left(
x\right)  =\left(  x-a\right)  ^{k}p_{k}\left(  x\right)  $,$q\left(
x\right)  =\left(  x-a\right)  ^{l}q_{l}\left(  x\right)  )$. Entonces
\[
\lim\limits_{x\rightarrow a}\frac{p\left(  x\right)  }{q(x)}=\left\{
\begin{tabular}
[c]{cc}%
$\pm\infty$ & si $k<l,$\\
$\dfrac{p_{k}\left(  a\right)  }{q_{l}\left(  a\right)  }$ & si $k=l,$\\
$0$ & si $k>l.$%
\end{tabular}
\ \ \ \right.
\]

\end{enumerate}

\item Si $\lim\limits_{x\rightarrow p}f(x)=0$ y $\lim\limits_{x\rightarrow
p}g(x)=c>0$, entonces si $f(x)\ $tiende a $0$ a traves de valores positivos de
$f(x),$%
\[
\lim\limits_{x\rightarrow p}\frac{g(x)}{f(x)}=\infty.
\]


\item Si $\lim\limits_{x\rightarrow p}f(x)=0$ y $\lim\limits_{x\rightarrow
p}g(x)=c>0$, entonces si $f(x)\ $tiende a $0$ a traves de valores negativos de
$f(x),$%
\[
\lim\limits_{x\rightarrow p}\frac{g(x)}{f(x)}=-\infty.
\]


\item Si $\lim\limits_{x\rightarrow p}f(x)=0$ y $\lim\limits_{x\rightarrow
p}g(x)=c<0$, entonces si $f(x)\ $tiende a $0$ a traves de valores positivos de
$f(x),$%
\[
\lim\limits_{x\rightarrow p}\frac{g(x)}{f(x)}=-\infty.
\]


\item Si $\lim\limits_{x\rightarrow p}f(x)=0$ y $\lim\limits_{x\rightarrow
p}g(x)=c<0$, entonces si $f(x)\ $tiende a $0$ a traves de valores negativos de
$f(x),$%
\[
\lim\limits_{x\rightarrow p}\frac{g(x)}{f(x)}=\infty.
\]


\item Si $p(x)=\sum\limits_{k=1}^{m}a_{k}x^{k}$ y $q(x)=\sum\limits_{k=1}%
^{n}b_{k}x^{k}$ son polinomios, de grados $m$ y $n$ respectivamente. Entonces%
\[
\lim\limits_{x\rightarrow\infty}\frac{p\left(  x\right)  }{q(x)}=\left\{
\begin{tabular}
[c]{cc}%
$0$ & si $m<n,$\\
$\dfrac{a_{n}}{b_{n}}$ & si $m=n,$\\
$\pm\infty$ & si $m>n.$%
\end{tabular}
\ \ \ \right.
\]

\end{enumerate}
\end{theorem}
\section{Continuidad}

Sean $r>0$ y $f$ una funci\'{o}n definida en una vecindad $\left]
p-r,p+r\right[  $. La funci\'{o}n $f$ es continua en $x=p,$ si y s\'{o}lo si
\[
\lim\limits_{x\rightarrow p}f(x)=f(p).
\]
Obs\'{e}rvese que si $f$ es continua%
\index{Continuidad|textbf}
en $x=p$, se implican las tres condiciones siguientes:

\begin{enumerate}
\item Existe $\lim\limits_{x\rightarrow p}f(x).$

\item $f$ esta definida en $p$, es decir $f(p)$ existe, y

\item $\lim\limits_{x\rightarrow p}f(x)=f(p).$
\end{enumerate}

De otra parte, se dice que $f$ es discontinua en $x=p,$ si $f$ esta definida
en un intervalo abierto que contiene a $p$ (excepto quiz\'{a}s en $p$) y $f$
no es continua en $p.$ Las discontinuidades%
\index{Funci\'{o}n!-- continua}%
\index{Funci\'{o}n!-- discontinua}
se clasifican en evitables si $\lim\limits_{x\rightarrow p}f(x)$ existe y
esenciales si $\lim\limits_{x\rightarrow p}f(x)$ no existe. En el caso que una
funci\'{o}n $f$ sea discontinua en $x=a$ y dicha discontinuidad sea evitable,
existe una funci\'{o}n $g$ tal que
\[
g(x)=\left\{
\begin{tabular}
[c]{cc}%
$f(x)$ & , si $x\neq a$\\
$\lim\limits_{x\rightarrow a}f(x)$ & , si $x=a$%
\end{tabular}
\ \ \ \right.  .
\]
Es claro que la funci\'{o}n $g$ es continua en $x=a$ y es \textquotedblleft
casi\textquotedblright\ la misma funci\'{o}n $f$. Por ello, usualmente se dice
que $g$ es una
\index{Funci\'{o}n!Redefinici\'{o}n de una --}%
\textquotedblright redefinici\'{o}n de $f$\textquotedblright\ en $x=a$ o
tambien que$\ g$ es una \textquotedblright extensi\'{o}n continua de
$f$\textquotedblright\ en $x=a$.

\begin{remark}
La gr\'{a}fica de una funci\'{o}n continua se puede trazar sin levantar el
l\'{a}piz del papel, mientras que para una funci\'{o}n discontinua esto no
ocurre, por lo general hay un salto en la discontinuidad. Sin embargo, esto no
se puede tomar como una definici\'{o}n formal de continuidad o discontinuidad.
\end{remark}

\begin{definition}
{\bf Continuidad en un intervalo abierto: \ }%
\index{Continuidad!-- en un intervalo}%
Una funci\'{o}n $f:\rz\rightarrow\rz\ $es continua en un intervalo abierto
$\left]  a,b\right[  \subseteq\operatorname*{Dom}\left(  f\right)  $, si es
continua en cada punto del intervalo $\left]  a,b\right[  .$
\end{definition}

Podemos extender, con
\index{Continuidad!-- lateral}%
peque\~{n}as variaciones, la continuidad de una funci\'{o}n en intervalos no
necesariamente abiertos. Sin embargo, para los extremos del intervalo,
incluidos en el dominio de la funci\'{o}n, solo se exigir\'{a} continuidad
unilateral. As\'{\i}, por ejemplo, si $f:\left[  a,b\right[  \subseteq
\operatorname*{Dom}\left(  f\right)  \longrightarrow\rz$ es una funci\'{o}n
diremos que es continua en $\left[  a,b\right[  $, si es continua en $\left]
a,b\right[  $ y por la derecha de $a$, esto \'{u}ltimo coincide con probar que
$\lim\limits_{x\rightarrow a^{+}}f(x)=f(a)$. De manera similar, diremos que
$f:\left]  a,b\right]  \subseteq\operatorname*{Dom}\left(  f\right)
\longrightarrow\rz$ es continua en $\left]  a,b\right]  ,$ si $f$ es continua
en $\left]  a,b\right[  $ y $\lim\limits_{x\rightarrow b^{-}}f(x)=f(b)$. De
igual forma podemos decir:

\begin{definition}
Una funci\'{o}n $f:\rz\rightarrow\rz\ $es continua en un intervalo cerrado
$\left[  a,b\right]  \subseteq\operatorname*{Dom}\left(  f\right)  $, si es
continua en $\left]  a,b\right[  \ $y$\ $si $\lim\limits_{x\rightarrow a^{+}%
}f(x)=f(a)$ y $\lim\limits_{x\rightarrow b^{-}}f(x)=f(b).$
\end{definition}

\subsection{Teorema del valor intermedio}

Intuitivamente el hecho que una funci\'{o}n sea continua en un intervalo
significa que la gr\'{a}fica de $f$ puede dibujarse de un solo trazo, sin
levantar la mano, no presentando as\'{\i} saltos ni \textquotedblleft
agujeros\textquotedblright\ en $\left(  a,b\right)  $. Podemos formalizar esta
apreciaci\'{o}n diciendo que si $f(a)$ y $f(b)$ son distintos, y $f$ es
continua en $[a,b]$ entonces $f$ toma todos los valores entre $f(a)$ y $f(b)$
(ver figura \ref{valorintermedio}), resultado conocido como el teorema del
valor intermedio.
%TODO  revisar
% Sea {\displaystyle f\ }f\  una función continua en un intervalo {\displaystyle [a,b]\ }[a,b]\ . Entonces para cada {\displaystyle u\ }u\  tal que {\displaystyle f(a)<u<f(b)\ }f(a)<u<f(b)\ , existe al menos un {\displaystyle c\ }c\  dentro de {\displaystyle (a,b)\ }(a,b)\  tal que {\displaystyle f(c)=u\ }f(c)=u\ .
\begin{figure}[H]
\centering
\includegraphics[scale=0.5]%
{fig-2-6.pdf}%
\caption{Teorema del valor intermedio}%
\label{valorintermedio}%
\end{figure}

%TODO -------------------------
\begin{theorem} {\bfseries Teorema del valor intermedio}\ %
\index{Teorema!-- del valor intermedio}%
\label{a6} Si $f$ es continua en $[a,b],\ f(a)=A$\ y\ $f(b)=B$,$\ $entonces
para todo $m\in\left]  A,B\right[  $ existe al menos un $c\in\left]
a,b\right[  $ tal que $f(c)=m$. Es decir, la funci\'{o}n toma todos los
valores entre $A$ y $B$.
\end{theorem}

Este importante teorema, es consecuencia directa de los siguientes resultados.

\begin{theorem}
{\bf Bolzano:\ } \label{Bolzano}%
\index{Teorema!-- de Bolzano}%
Si $f$ es continua en $[a,b],\ f(a)\ y\ f(b)$ tienen signos opuestos existe al
menos un $c\in\left]  a,b\right[  $, tal que $f(c)=0$. Es decir, la
ecuaci\'{o}n $f(x)=0$ tiene al menos una soluci\'{o}n en $\left]  a,b\right[
$.
\end{theorem}

\begin{theorem}
{ \bf Conservaci\'{o}n de signo:\ }%
\index{Teorema!-- de conservaci\'{o}n de signos}%
\label{Conservacion}Sea $f$ continua en $c$ y supongamos que $f\left(
c\right)  \neq0.$Existe entonces una vecindad de $\left]  c-r,c+r\right[  $ de
$c,$ donde $f$ tiene el mismo signo de $f\left(  c\right)  .$
\end{theorem}

El teorema de Bolzano
\index{M\'{e}todo!-- de bisecci\'{o}n}%
indica que si $f$ es continua en $[a,b]$ y $f(a)f(b)\neq0$, necesariamente
existe una soluci\'{o}n de la ecuaci\'{o}n $f(x)=0$ en el intervalo $\left]
a,b\right[  $. Al suponer la existencia de una soluci\'{o}n $c$ en dicho
intervalo, puede construirse una sucesi\'{o}n $\{x_{n}\}_{n\in\nz}$
convergente a $c$, considerando subintervalos de longitud $\dfrac{b-a}{2^{n}}%
$, la cual puede usarse entonces para obtener, escogiendo $n$ suficientemente
grande, una aproximaci\'{o}n de dicha soluci\'{o}n. Este procedimiento es
conocido como m\'{e}todo de bisecci\'{o}n y aunque en general puede ser muy
lento en la obtenci\'{o}n de aproximaciones \'{o}ptimas, se puede usar al
menos para acotar la soluci\'{o}n buscada en intervalos de longitudes peque\~{n}as.

\bigskip

Sea $A\subseteq\operatorname*{Dom}\left(  f\right)  $, con $c\in A,$ en el
caso que $f\left(  c\right)  \geq f\left(  x\right)  ,$ para toda $x\in A,$
$x=c$ se llamar\'{a}
\index{M\'{a}ximo!absoluto}%
\index{M\'{\i}nimo!absoluto}%
\textit{m\'{a}ximo absoluto de }$f$\textit{ en }$A$ y si $f\left(  c\right)
\leq f\left(  x\right)  $ para toda $x\in A,$ $x=c$ se dir\'{a}
\textit{m\'{\i}nimo absoluto de }$f$\textit{ en }$A$.%
%TODO


\begin{center}
\includegraphics[scale=0.6]%
{fig-2-7.pdf}
\end{center}%


%TODO 
Cuando $f$ resulta ser una funci\'{o}n continua en $[a,b]\subseteq
\operatorname*{Dom}\left(  f\right)  ,$ se dan comportamientos interesantes.
El m\'{a}s importante de ellos tiene que ver con el resultado conocido como
\emph{Teorema de los valores extremos}, el cual presentamos inmediatamente.

\begin{theorem}
{\bf Teorema de los valores extremos:\ }%
\index{Teorema!-- de los valores extremos}%
\label{Teovalorextre}Si $f$ es continua en un intervalo cerrado $\left[
a,b\right]  ,$ entonces $f$ presenta un m\'{a}ximo absoluto y un m\'{\i}nimo
absoluto en $\left[  a,b\right]  .$ Es decir existen $x_{1},x_{2}\in\left[
a,b\right]  $ tales que para toda $x\in\left[  a,b\right]  $%
\[
f\left(  x_{2}\right)  \leq f\left(  x\right)  \leq f\left(  x_{1}\right)  .
\]

\end{theorem}

Como consecuencias inmediatas de este teorema y del teorema del valor
intermedio se obtienen:

\begin{corollary}
\label{acotamiento}Si $f$
\index{Funci\'{o}n!acotada}%
es continua en el intervalo cerrado $[a,b]$, entonces $f$ es acotada en
$[a,b]$. Es decir, existen n\'{u}meros reales $m$ y $M$ tales que $m\leq
f\left(  x\right)  \leq M.$
\end{corollary}

\begin{corollary}%
\index{Rango}%
Si $f$ es continua en el intervalo cerrado $[a,b],$ entonces el rango de $f$
\[
\operatorname*{Ran}f=\left\{  y\mid y=f\left(  x\right)  ,x\in\lbrack
a,b]\right\}  =:f\left(  [a,b]\right)
\]
es un intervalo cerrado.
\end{corollary}

\subsection{Teoremas sobre continuidad.}%

\index{Continuidad!Teoremas sobre --}%
En esta secci\'{o}n se enuncian, sin demostraci\'{o}n, otros teoremas
importantes que cumplen las funciones continuas.

\begin{theorem}
\label{a5}Si las funciones$\ f\ $y$\ g\ $son continuas en el punto $x=p$,
entonces las funciones:

\begin{enumerate}
\item $f+g,$

\item $f-g,$

\item $fg,$

\item $\dfrac{f}{g}$ para $g\not \equiv 0,$ $g\left(  p\right)  \neq0,$
\end{enumerate}

Son continuas en\emph{ }$x=p.$
\end{theorem}

\begin{theorem}
\label{a10}Si $g$ es continua en $a$ y $f$ es continua en $g(a)$ entonces la
funci\'{o}n $h:=f\circ g$ es continua en $a$.
\end{theorem}

\section{As\'{\i}ntotas.%
\index{As\'{\i}ntotas|textbf}%
}

\begin{definition}{\textbf{As\'{\i}ntota vertical}:\ }
\index{As\'{\i}ntotas!-- verticales}%
Se dice que la recta $x=a$ es una as\'{\i}ntota vertical de la curva $y=f(x),$
si por lo menos uno de los enunciados siguientes es verdadero:

\begin{enumerate}
\item $\lim\limits_{x\rightarrow a^{+}}f(x)=\pm\infty$

\item $\lim\limits_{x\rightarrow a^{-}}f(x)=\pm\infty.$
\end{enumerate}
\end{definition}

\begin{definition}
{\textbf{As\'{\i}ntota horizontal}:\ }%
\index{As\'{\i}ntotas!-- horizontales}%
La recta $y=L$ es una as\'{\i}ntota horizontal de la curva $y=f(x)$ si
\[
\lim\limits_{x\rightarrow\infty}f(x)=L\ \text{\'{o}}\ \lim
\limits_{x\rightarrow-\infty}f(x)=L.
\]

\end{definition}

\begin{definition}
{\bf As\'intota oblicua:\ }%
\index{As\'{\i}ntotas!-- oblicuas}%
La recta $y=mx+b\ $con$\ m\neq0$ es una as\'{\i}ntota inclinada u oblicua de
la curva $y=f(x)$ si
\[
\lim_{x\rightarrow\infty}\left\vert f(x)-(mx+b)\right\vert =0.
\]
En otras palabras, $y=mx+b$ es una as\'{\i}ntota oblicua si se cumplen las
siguientes condiciones:
\begin{enumerate}
\item $\lim\limits_{x\rightarrow\infty}f(x)=\pm\infty$
\item $\lim\limits_{x\rightarrow\infty}\dfrac{f(x)}{x}=m.$
\item $\lim_{x\rightarrow\infty}\left(  f(x)-mx\right)  =b.$
\end{enumerate}
\end{definition}

\section{Ejercicios resueltos}

\begin{example}
Utilizar el resultado obtenido en la serie geometrica para expresar como
racional los siguientes decimales infinitos periodicos.

\begin{enumerate}
\item $0.444444....$

\item $1.388888.....$

\item $2.5526262626.....$
\end{enumerate}
\end{example}

\begin{sol}
\begin{enumerate}
\item $0.444444....=\dfrac{4}{10}+\dfrac{4}{100}+\dfrac{4}{1000}+......=%
{\displaystyle\sum\limits_{n=1}^{\infty}}
4\left(  \dfrac{1}{10}\right)  ^{n}=\dfrac{4}{10}%
{\displaystyle\sum\limits_{n=0}^{\infty}}
\left(  \dfrac{1}{10}\right)  ^{n}$\newline Esta \'{u}ltima es una serie
geometrica la cual converge en consecuencia.%
\begin{align*}
0.444444....  &  =\dfrac{4}{10}%
{\displaystyle\sum\limits_{n=0}^{\infty}}
\left(  \dfrac{1}{10}\right)  ^{n}\\
&  =\dfrac{4}{10}\ast\left(  \dfrac{1}{1-\frac{1}{10}}\right) \\
&  =\frac{4}{9}.
\end{align*}


\item $1.38888.....=\dfrac{13.888...}{10}=\dfrac{13+\frac{8}{10}+\frac{8}%
{100}+\frac{8}{1000}+\cdots}{10}=\dfrac{13+\frac{8}{10}\sum\limits_{n=0}%
^{\infty}\left(  \frac{1}{10}\right)  ^{n}}{10}$\newline Procediendo igual que
en la parte anterior tenemos que%
\begin{align*}
1.3888....  &  =\frac{13+\frac{8}{10}\sum\limits_{n=0}^{\infty}\left(
\frac{1}{10}\right)  ^{n}}{10}\\
&  =\frac{13+\frac{8}{10}\cdot\left(  \dfrac{1}{1-\frac{1}{10}}\right)  }%
{10}\\
&  =\dfrac{25}{18}.
\end{align*}


\item $2.5526262626.....=\dfrac{255.262626..}{100}=\dfrac{255+\dfrac
{26}{10^{2}}+\dfrac{26}{10^{4}}+\dfrac{26}{10^{6}}+....}{100}\newline$en
consecuencia
\begin{align*}
2.5526262626.....  &  =\dfrac{255+\dfrac{26}{100}%
{\displaystyle\sum\limits_{n=0}^{\infty}}
\left(  \dfrac{1}{100}\right)  ^{n}}{100}\\
&  =\dfrac{255+\dfrac{26}{100}\left(  \dfrac{1}{1-\frac{1}{100}}\right)
}{100}\\
&  =\frac{25\,271}{9900}.
\end{align*}

\end{enumerate}
\end{sol}

\begin{example}
Calcule el l\'{\i}mite en la siguientes series

\begin{enumerate}
\item $%
{\displaystyle\sum\limits_{n=1}^{\infty}}
\left(  \dfrac{1}{e^{2n}}\right)  .$

\item $%
{\displaystyle\sum\limits_{n=1}^{\infty}}
\left(  \dfrac{3^{n+1}}{8^{n}}\right)  .$
\end{enumerate}
\end{example}

\begin{sol}
\begin{enumerate}
\item $%
{\displaystyle\sum\limits_{n=1}^{\infty}}
\left(  \dfrac{1}{e^{2n}}\right)  =%
{\displaystyle\sum\limits_{n=1}^{\infty}}
\left(  \dfrac{1}{e^{2}}\right)  ^{n}=%
{\displaystyle\sum\limits_{n=0}^{\infty}}
\left(  \dfrac{1}{e^{2}}\right)  ^{n+1}=\dfrac{1}{e^{2}}%
{\displaystyle\sum\limits_{n=0}^{\infty}}
\left(  \dfrac{1}{e^{2}}\right)  ^{n}\newline$Ahora tenemos una serie
geometrica. por lo tanto%
\begin{align*}
\dfrac{1}{e^{2}}%
{\displaystyle\sum\limits_{n=0}^{\infty}}
\left(  \dfrac{1}{e^{2}}\right)  ^{n}  &  =\dfrac{1}{e^{2}}\left(  \dfrac
{1}{1-\frac{1}{e^{2}}}\right) \\
&  =\dfrac{1}{e^{2}-1}.
\end{align*}


\item $%
{\displaystyle\sum\limits_{n=1}^{\infty}}
\left(  \dfrac{3^{n+1}}{8^{n}}\right)  =%
{\displaystyle\sum\limits_{n=0}^{\infty}}
\left(  \dfrac{3^{n+2}}{8^{n+1}}\right)  =\dfrac{9}{8}%
{\displaystyle\sum\limits_{n=0}^{\infty}}
\left(  \dfrac{3}{8}\right)  ^{n}=\dfrac{9}{8}\left(  \dfrac{1}{1-\frac{3}{8}%
}\right)  =\allowbreak\dfrac{9}{5}.$
\end{enumerate}
\end{sol}

En esta secci\'{o}n presentamos t\'{e}cnicas para calcular l\'{\i}mites, en
cada caso o situaci\'{o}n se presentan varios ejemplos resueltos. El
c\'{a}lculo de l\'{\i}mites presenta alguna dificultad cuando se presentan
formas indeterminadas, como por ejemplo: $\frac{0}{0}$, $0.\infty$,
$\infty-\infty,\ \frac{\infty}{\infty},\ 0^{0}.\ $Cuando no se presentan
formas indeterminadas el c\'{a}lculo del l\'{\i}mite es trivial.

\begin{example}
Calcule $\lim\limits_{x\rightarrow2}(x^{2}+3x-5).$
\end{example}

\begin{sol}
\textbf{\ }Aplicando el teorema \ref{a1} se tiene que
\begin{align*}
\lim\limits_{x\rightarrow2}(x^{2}+3x-5)  &  =2^{2}+3.2-5=5\\
\lim\limits_{x\rightarrow2}(x^{2}+3x-5)  &  =5.
\end{align*}

\end{sol}

\begin{example}
Calcule $\lim\limits_{x\rightarrow3}\dfrac{x^{2}+2x-4}{x^{3}-5}.$
\end{example}

\begin{sol}
Aplicando el teorema \ref{a1} se tiene que
\begin{align*}
\lim\limits_{x\rightarrow3}\dfrac{x^{2}+2x-4}{x^{3}-5}  &  =\frac
{\lim\limits_{x\rightarrow3}x^{2}+2x-4}{\lim\limits_{x\rightarrow3}x^{3}-5}\\
&  =\frac{3^{2}+2.3-4}{3^{3}-5}\\
&  =\frac{11}{22}=\frac{1}{2}.
\end{align*}

\end{sol}

\begin{example}
Calcule $\lim\limits_{x\rightarrow1}\dfrac{x^{2}+5x-6}{x^{2}-1}.$
\end{example}

\begin{sol}
Como $\lim\limits_{x\rightarrow1}x^{2}+5x-6=0$ y $\lim\limits_{x\rightarrow
1}x^{2}-1=0$, se tiene que $\lim\limits_{x\rightarrow1}\dfrac{x^{2}%
+5x-6}{x^{2}-1}$ presenta la forma indeterminada $\frac{0}{0}$. Cuando esto
suceda debemos aplicar nuestros conocimientos de \'{a}lgebra para evitar la
indeterminaci\'{o}n, en este ejemplo basta con factorizar numerador y
denominador. Por lo tanto,
\begin{align*}
\lim\limits_{x\rightarrow1}\dfrac{x^{2}+5x-6}{x^{2}-1}  &  =\lim
\limits_{x\rightarrow1}\frac{\left(  x+6\right)  \left(  x-1\right)  }{\left(
x-1\right)  \left(  x+1\right)  }\\
&  =\lim\limits_{x\rightarrow1}\frac{\left(  x+6\right)  }{\left(  x+1\right)
}\\
&  =\frac{7}{2}.
\end{align*}

\end{sol}

\begin{example}
Calcule $\lim\limits_{x\rightarrow2}\dfrac{x^{4}-16}{x^{3}-8}.$
\end{example}

\begin{sol}
Observamos que el l\'{\i}mite es de la forma $\frac{0}{0}.$ Factoricemos
numerador y denominador para evitar la indeterminaci\'{o}n, por lo cual,%
\begin{align*}
\lim\limits_{x\rightarrow2}\dfrac{x^{4}-16}{x^{3}-8}  &  =\lim
\limits_{x\rightarrow2}\frac{(x-2)(x+2)(x^{2}+4)}{(x-2)(x^{2}+2x+4)}\\
&  =\lim\limits_{x\rightarrow2}\frac{(x+2)(x^{2}+4)}{(x^{2}+2x+4)}\\
&  =\frac{(2+2)(2^{2}+2)}{2^{2}+2.2+4}=\frac{24}{12}=2.
\end{align*}

\end{sol}

\begin{example}
Calcule $\lim\limits_{x\rightarrow-1}\dfrac{x^{3}+2x^{2}-x-2}{x^{3}-3x-2}.$
\end{example}

\begin{sol}
Este l\'{\i}mite es de la forma $\frac{0}{0}.$ Factorizando numerador y
denominador se obtiene,%
\begin{align*}
\lim\limits_{x\rightarrow-1}\dfrac{x^{3}+2x^{2}-x-2}{x^{3}-3x-2}  &
=\lim\limits_{x\rightarrow-1}\dfrac{\left(  x-1\right)  \left(  x+2\right)
\left(  x+1\right)  }{\left(  x-2\right)  \left(  x+1\right)  ^{2}}\\
&  =\lim\limits_{x\rightarrow-1}\dfrac{\left(  x-1\right)  \left(  x+2\right)
}{\left(  x-2\right)  \left(  x+1\right)  }%
\end{align*}
Observe que el numerador tiende a $-2$ cuando $x$ tiende a $-1$, mientras que
el denominador tiende a $0\ $a traves de valores positivos, cuando $x$ tiende
a $-1$ por la izquierda y tiende a $0$ a traves de valores negativos, cuando
$x$ tiende a $-1$ por la derecha. Por lo tanto,%
\[
\lim\limits_{x\rightarrow-1^{+}}\dfrac{\left(  x-1\right)  \left(  x+2\right)
}{\left(  x-2\right)  \left(  x+1\right)  }=\infty\text{ \ y \ }%
\lim\limits_{x\rightarrow-1^{-}}\dfrac{\left(  x-1\right)  \left(  x+2\right)
}{\left(  x-2\right)  \left(  x+1\right)  }=-\infty
\]
se tiene que el l\'{\i}mite no existe.
\end{sol}

\begin{example}
Calcule $\lim\limits_{h\rightarrow0}\dfrac{h}{\sqrt{h+2}-\sqrt{2}}$
\end{example}

\begin{sol}
En este caso tambi\'{e}n se tiene un l\'{\i}mite de la forma $\frac{0}{0},$
con la diferencia que ahora tenemos radicales. Cuando esto ocurra la
t\'{e}cnica mas apropiada es la racionalizaci\'{o}n.%
\begin{align*}
\lim\limits_{h\rightarrow0}\dfrac{h}{\sqrt{h+2}-\sqrt{2}}  &  =\lim
\limits_{h\rightarrow0}\dfrac{h}{\sqrt{h+2}-\sqrt{2}}.\frac{\sqrt{h+2}%
+\sqrt{2}}{\sqrt{h+2}+\sqrt{2}}\\
&  =\lim\limits_{h\rightarrow0}\dfrac{h.(\sqrt{h+2}+\sqrt{2})}{\left(
h+2\right)  -2}\\
&  =\lim\limits_{h\rightarrow0}\frac{h.(\sqrt{h+2}+\sqrt{2})}{h}\\
&  =\lim\limits_{h\rightarrow0}(\sqrt{h+2}+\sqrt{2})=2\sqrt{2}.
\end{align*}

\end{sol}

\begin{example}
\label{ejemplorelcap3}Calcule $\lim\limits_{x\rightarrow2}\dfrac{5-\sqrt
{x^{2}+21}}{\sqrt{x+7}-3}$
\end{example}

\begin{sol}
Este l\'{\i}mite tambi\'{e}n es de la forma $\frac{0}{0}$. Observemos que hay
radicales tanto en el numerador como en el denominador, por lo cual debemos
racionalizar numerador y denominador.%
\begin{align*}
\lim\limits_{x\rightarrow2}\dfrac{5-\sqrt{x^{2}+21}}{\sqrt{x+7}-3}  &
=\lim\limits_{x\rightarrow2}\dfrac{5-\sqrt{x^{2}+21}}{\sqrt{x+7}-3}%
.\dfrac{5+\sqrt{x^{2}+21}}{5+\sqrt{x^{2}+21}}.\dfrac{\sqrt{x+7}+3}{\sqrt
{x+7}+3}\\
&  =\lim\limits_{x\rightarrow2}\dfrac{25-(x^{2}+21)}{(x+7)-3}.\dfrac
{\sqrt{x+7}+3}{5+\sqrt{x^{2}+21}}\\
&  =\lim\limits_{x\rightarrow2}\dfrac{4-x^{2}}{x-2}.\dfrac{\sqrt{x+7}%
+3}{5+\sqrt{x^{2}+21}}\\
&  =\lim\limits_{x\rightarrow2}\dfrac{(2-x)(2+x)}{x-2}.\dfrac{\sqrt{x+7}%
+3}{5+\sqrt{x^{2}+21}}\\
&  =-\lim\limits_{x\rightarrow2}\dfrac{(2+x)}{1}.\dfrac{\sqrt{x+7}+3}%
{5+\sqrt{x^{2}+21}}=-\frac{24}{10}=-\frac{12}{5}.
\end{align*}

\end{sol}

\begin{example}
Calcule $\lim\limits_{x\rightarrow4}\dfrac{x^{2}-16}{\sqrt[3]{x}-\sqrt[3]{4}}$
\end{example}

\begin{sol}
Este l\'{\i}mite tambi\'{e}n es de la forma $\frac{0}{0},$ pero ahora tenemos
ra\'{\i}ces c\'{u}bicas en el denominador. Para racionalizar el denominador
recordemos que%
\[
(\sqrt[3]{a}-\sqrt[3]{b})(\sqrt[3]{a^{2}}+\sqrt[3]{ab}+\sqrt[3]{b^{2}})=a-b
\]%
\begin{align*}
\lim\limits_{x\rightarrow4}\dfrac{x^{2}-16}{\sqrt[3]{x}-\sqrt[3]{4}}  &
=\lim\limits_{x\rightarrow4}\frac{(x-4)(x+4)}{\sqrt[3]{x}-\sqrt[3]{4}}%
.\frac{\sqrt[3]{x^{2}}+\sqrt[3]{4x}+\sqrt[3]{16}}{\sqrt[3]{x^{2}}+\sqrt[3]%
{4x}+\sqrt[3]{16}}\\
&  =\lim\limits_{x\rightarrow4}\frac{(x-4)(x+4)(\sqrt[3]{x^{2}}+\sqrt[3]%
{4x}+\sqrt[3]{16})}{x-4}\\
&  =\lim\limits_{x\rightarrow4}\frac{(x+4)(\sqrt[3]{x^{2}}+\sqrt[3]%
{4x}+\sqrt[3]{16})}{1}=48\sqrt[3]{2}.
\end{align*}

\end{sol}

\begin{example}
Probar que $\lim\limits_{x\rightarrow4}\dfrac{\left|  x-4\right|  } {x-4}$ no existe
\end{example}

\begin{sol}
Calculemos $\lim\limits_{x\rightarrow4^{+}}\dfrac{\left|  x-4\right|  }{x-4}$
y $\lim\limits_{x\rightarrow4^{-}}\dfrac{\left|  x-4\right|  }{x-4}$%
\begin{align*}
\lim\limits_{x\rightarrow4^{+}}\dfrac{\left|  x-4\right|  }{x-4}  &
=\lim\limits_{x\rightarrow4^{+}}\dfrac{x-4}{x-4}=1\\
\lim\limits_{x\rightarrow4^{-}}\dfrac{\left|  x-4\right|  }{x-4}  &
=\lim\limits_{x\rightarrow4^{-}}\dfrac{-\left(  x-4\right)  }{x-4}=-1
\end{align*}
Por lo tanto, $\lim\limits_{x\rightarrow4}\dfrac{\left|  x-4\right|  }{x-4}$
no existe.
\end{sol}

\begin{example}
Calcule $\lim\limits_{x\rightarrow0}x\operatorname{sen}\dfrac{1}{x}$
\end{example}

\begin{sol}
Ya que $0\leq\left|  \operatorname{sen}\frac{1}{x}\right|  \leq1,$ para
$x\neq0.$ Al multiplicar esta desigualdad por $\left|  x\right|  \ $y
aplicando el teorema del emparedado se obtiene el resultado. Es decir,%
\begin{align*}
0  &  \leq\left|  \operatorname{sen}\frac{1}{x}\right|  \leq1\\
0  &  \leq\left|  x\right|  \left|  \operatorname{sen}\frac{1}{x}\right|
\leq\left|  x\right| \\
0  &  \leq\left|  x\operatorname{sen}\frac{1}{x}\right|  \leq\left|  x\right|
\\
\lim\limits_{x\rightarrow0}0  &  \leq\lim\limits_{x\rightarrow0}\left|
x\operatorname{sen}\frac{1}{x}\right|  \leq\lim\limits_{x\rightarrow0}\left|
x\right| \\
0  &  \leq\lim\limits_{x\rightarrow0}\left|  x\operatorname{sen}\frac{1}%
{x}\right|  \leq0
\end{align*}
lo anterior implica que $\lim\limits_{x\rightarrow0}\left|
x\operatorname{sen}\dfrac{1}{x}\right|  =0$, es decir $\lim
\limits_{x\rightarrow0}x\operatorname{sen}\dfrac{1}{x}=0.$
\end{sol}

\begin{example}
Calcule $\lim\limits_{x\rightarrow0}x^{2}\operatorname{sen}\frac{1}{x}$
\end{example}

\begin{sol}
En forma an\'{a}loga al problema anterior:%
\begin{gather*}
-1\leq\operatorname{sen}\frac{1}{x}\leq1\\
-x^{2}\leq x^{2}\operatorname{sen}\frac{1}{x}\leq x^{2}\\
\lim\limits_{x\rightarrow0}(-x^{2})\leq\lim\limits_{x\rightarrow0}%
x^{2}\operatorname{sen}\frac{1}{x}\leq\lim\limits_{x\rightarrow0}x^{2}\\
0\leq\lim\limits_{x\rightarrow0}x^{2}\operatorname{sen}\frac{1}{x}\leq0
\end{gather*}
entonces, $\lim\limits_{x\rightarrow0}x^{2}\operatorname{sen}\dfrac{1}{x}=0.$
\end{sol}

\begin{example}
\label{p7}Calcule $\lim\limits_{x\rightarrow\infty}\dfrac{\operatorname{sen}%
x}{x}$
\end{example}

\begin{sol}
Observe que cuando $x\rightarrow\infty,$ $x>0$. Por cual, se puede dividir por
$x$ la desigualdad $-1\leq\operatorname{sen}x\leq1.$ Entonces,%
\begin{gather*}
\dfrac{-1}{x}\leq\dfrac{\operatorname{sen}x}{x}\leq\dfrac{1}{x}\\
\lim\limits_{x\rightarrow\infty}\dfrac{-1}{x}\leq\lim\limits_{x\rightarrow
\infty}\dfrac{\operatorname{sen}x}{x}\leq\lim\limits_{x\rightarrow\infty
}\dfrac{1}{x}\\
0\leq\lim\limits_{x\rightarrow\infty}\dfrac{\operatorname{sen}x}{x}\leq0
\end{gather*}
entonces$\ \lim\limits_{x\rightarrow\infty}\dfrac{\operatorname{sen}x}{x}=0.$
\end{sol}

\begin{example}
Calcule $\lim\limits_{x\rightarrow0}\dfrac{1-\cos x}{x^{2}}$
\end{example}

\begin{sol}
Este l\'{\i}mite es de la forma $\frac{0}{0}.$ Cuando obtengamos formas
indeterminadas en l\'{\i}mites trigonom\'{e}tricos, se acostumbra modificar la
expresi\'{o}n de tal manera que aparezcan los l\'{\i}mites $\lim
\limits_{x\rightarrow0}\dfrac{\operatorname{sen}x}{x}=1$ o \ $\lim
\limits_{x\rightarrow0}\dfrac{1-\cos x}{x}=0.$ En nuestro ejemplo:%
\begin{align*}
\lim\limits_{x\rightarrow0}\dfrac{1-\cos x}{x^{2}}  &  =\lim
\limits_{x\rightarrow0}\dfrac{1-\cos x}{x^{2}}.\frac{1+\cos x}{1+\cos x}\\
&  =\lim\limits_{x\rightarrow0}\dfrac{1-\cos^{2}x}{x^{2}(1+\cos x)}\\
&  =\lim\limits_{x\rightarrow0}\dfrac{\operatorname{sen}^{2}x}{x^{2}(1+\cos
x)}\\
&  =\lim\limits_{x\rightarrow0}\dfrac{\operatorname{sen}^{2}x}{x^{2}}%
.\lim\limits_{x\rightarrow0}\frac{1}{1+\cos x}\\
&  =1.\frac{1}{2}=\frac{1}{2}.
\end{align*}

\end{sol}

\begin{example}
Calcule $\lim\limits_{x\rightarrow\frac{\pi}{4}}\dfrac{1-\tan x}%
{\operatorname{sen}x-\cos x}$
\end{example}

\begin{sol}
Como $\tan x=\dfrac{\operatorname{sen}x}{\cos x},$ procediendo en forma
an\'{a}loga al problema anterior se tiene:
\begin{align*}
\lim\limits_{x\rightarrow\frac{\pi}{4}}\dfrac{1-\tan x}{\operatorname{sen}%
x-\cos x}  &  =\lim\limits_{x\rightarrow\frac{\pi}{4}}\dfrac{1-\dfrac
{\operatorname{sen}x}{\cos x}}{\operatorname{sen}x-\cos x}\\
&  =\lim\limits_{x\rightarrow\frac{\pi}{4}}\dfrac{\cos x-\operatorname{sen}%
x}{\cos x\left(  \operatorname{sen}x-\cos x\right)  }\\
&  =\lim\limits_{x\rightarrow\frac{\pi}{4}}\frac{-1}{\cos x}=-\sqrt{2}.
\end{align*}

\end{sol}

\begin{example}
Calcule $\lim\limits_{x\rightarrow0}\operatorname{sen}x\operatorname{sen}%
\left(  \frac{1}{x}\right)  $
\end{example}

\begin{sol}
El l\'{\i}mite inicial se puede reescribir, tomando la forma:%
\[
\lim\limits_{x\rightarrow0}\operatorname{sen}x\operatorname{sen}\frac{1}%
{x}=\lim\limits_{x\rightarrow0}\frac{\operatorname{sen}x}{x}.\frac
{\operatorname{sen}\frac{1}{x}}{\frac{1}{x}}.
\]
Observe que $\lim\limits_{x\rightarrow0}\frac{\operatorname{sen}x}{x}=1,$
mientras $\lim\limits_{x\rightarrow0}\frac{\operatorname{sen}\frac{1}{x}%
}{\frac{1}{x}}=0$. El segundo limite se obtiene tomando $u=\frac{1}{x},$ es
claro que $u\rightarrow\infty$ cuando $x\rightarrow0$, y aplicando el problema
a \ref{p7} se tiene que%
\begin{align*}
\lim\limits_{x\rightarrow0}\operatorname{sen}x\operatorname{sen}\left(
\frac{1}{x}\right)   &  =\lim\limits_{x\rightarrow0}\frac{\operatorname{sen}%
x}{x}.\lim\limits_{x\rightarrow0}\frac{\operatorname{sen}\frac{1}{x}}{\frac
{1}{x}}\\
&  =\lim\limits_{x\rightarrow0}\frac{\operatorname{sen}x}{x}.\lim
\limits_{u\rightarrow\infty}\frac{\operatorname{sen}u}{u}\\
&  =1.0=0.
\end{align*}
El resultado tambi\'{e}n se puede obtener, v\'{\i}a teorema del emparedado%
\index{Teorema!-- del emparedado}%
, utilizando la desigualdad
\[
\left\vert \operatorname{sen}\left(  x\right)  \operatorname{sen}\left(
\frac{1}{x}\right)  \right\vert \leq\left\vert \operatorname{sen}\left(
x\right)  \right\vert .
\]

\end{sol}

\begin{example}
Calcule $\lim\limits_{x\rightarrow0}\dfrac{1-\cos2x}{\operatorname{sen}3x}$
\end{example}

\begin{sol}%
\begin{align*}
\lim\limits_{x\rightarrow0}\dfrac{1-\cos2x}{\operatorname{sen}3x}  &
=\lim\limits_{x\rightarrow0}\dfrac{\dfrac{1-\cos2x}{2x}}{\dfrac{3}{2}\left(
\dfrac{\operatorname{sen}3x}{3x}\right)  }\\
&  =\frac{2}{3}\lim\limits_{x\rightarrow0}\dfrac{\dfrac{1-\cos2x}{2x}}%
{\dfrac{\operatorname{sen}3x}{3x}}%
\end{align*}
Ya que $\lim\limits_{x\rightarrow0}\dfrac{1-\cos2x}{2x}=0$ y $\lim
\limits_{x\rightarrow0}\dfrac{\operatorname{sen}3x}{3x}=1,$ se tiene que,%
\begin{align*}
\lim\limits_{x\rightarrow0}\dfrac{1-\cos2x}{\operatorname{sen}3x}  &
=\frac{2}{3}\dfrac{\lim\limits_{x\rightarrow0}\dfrac{1-\cos2x}{2x}}%
{\lim\limits_{x\rightarrow0}\dfrac{\operatorname{sen}3x}{3x}}\\
&  =\frac{2}{3}\cdot\dfrac{0}{1}=0
\end{align*}

\end{sol}

\begin{example}
Calcule $\lim\limits_{x\rightarrow\pi}\dfrac{\operatorname{sen}x}{x-\pi}$
\end{example}

\begin{sol}
Sea $t=x-\pi$, cuando $x\rightarrow\pi,t\rightarrow0,$ por lo tanto%
\begin{align*}
\lim\limits_{x\rightarrow\pi}\dfrac{\operatorname{sen}x}{x-\pi}  &
=\lim\limits_{t\rightarrow0}\dfrac{\operatorname{sen}(t+\pi)}{t}\\
&  =\lim\limits_{t\rightarrow0}\dfrac{\operatorname{sen}t\cos\pi
+\operatorname{sen}\pi\cos t}{t}\\
&  =\lim\limits_{t\rightarrow0}\dfrac{-\operatorname{sen}t}{t}=-1.
\end{align*}

\end{sol}

\begin{example}
Probar que el $\lim\limits_{x\rightarrow0}\dfrac{\sqrt{1-\cos x}}{x}$ no existe
\end{example}

\begin{sol}
Ya que $1-\cos x\geq0$ y $1+\cos x\geq0,$%
\begin{align*}
\lim\limits_{x\rightarrow0}\dfrac{\sqrt{1-\cos x}}{x}  &  =\lim
\limits_{x\rightarrow0}\dfrac{\sqrt{1-\cos x}}{x}.\frac{\sqrt{1+\cos x}}%
{\sqrt{1+\cos x}}\\
&  =\lim\limits_{x\rightarrow0}\dfrac{\sqrt{1-\cos^{2}x}}{x\sqrt{1+\cos x}}\\
&  =\lim\limits_{x\rightarrow0}\dfrac{\left|  \operatorname{sen}x\right|
}{x\sqrt{1+\cos x}}%
\end{align*}
Calculemos ahora $\lim\limits_{x\rightarrow0^{+}}\dfrac{\left|
\operatorname{sen}x\right|  }{x\sqrt{1+\cos x}}$ y $\lim\limits_{x\rightarrow
0^{-}}\dfrac{\left|  \operatorname{sen}x\right|  }{x\sqrt{1+\cos x}}$
\begin{align*}
\lim\limits_{x\rightarrow0^{+}}\dfrac{\left|  \operatorname{sen}x\right|
}{x\sqrt{1+\cos x}}  &  =\lim\limits_{x\rightarrow0^{+}}\dfrac
{\operatorname{sen}x}{x\sqrt{1+\cos x}}\\
&  =\lim\limits_{x\rightarrow0^{+}}\dfrac{\operatorname{sen}x}{x}%
.\lim\limits_{x\rightarrow0^{+}}\frac{1}{\sqrt{1+\cos x}}\\
&  =1.\frac{1}{\sqrt{2}}=\frac{1}{\sqrt{2}}.
\end{align*}%
\begin{align*}
\lim\limits_{x\rightarrow0^{-}}\dfrac{\left|  \operatorname{sen}x\right|
}{x\sqrt{1+\cos x}}  &  =\lim\limits_{x\rightarrow0^{-}}\dfrac
{-\operatorname{sen}x}{x\sqrt{1+\cos x}}\\
&  =\lim\limits_{x\rightarrow0^{-}}\dfrac{-\operatorname{sen}x}{x}%
.\lim\limits_{x\rightarrow0^{-}}\frac{1}{\sqrt{1+\cos x}}\\
&  =-1.\frac{1}{\sqrt{2}}=\frac{-1}{\sqrt{2}}.
\end{align*}
Por lo tanto el l\'{\i}mite no existe, ya que el l\'{\i}mite por la derecha es
diferente al l\'{\i}mite por la izquierda.
\end{sol}

\begin{example}
Calcule $\lim\limits_{x\rightarrow\infty}(\sqrt{x^{2}+3}-x)$
\end{example}

\begin{sol}
Este l\'{\i}mite presenta la forma indeterminada $\infty-\infty$.
Multiplicando y dividiendo por $\sqrt{x^{2}+3}+x$ se tiene:%
\begin{align*}
\lim\limits_{x\rightarrow\infty}(\sqrt{x^{2}+3}-x)  &  =\lim
\limits_{x\rightarrow\infty}\frac{(\sqrt{x^{2}+3}-x)(\sqrt{x^{2}+3}+x)}%
{\sqrt{x^{2}+3}+x}\\
&  =\lim\limits_{x\rightarrow\infty}\frac{(x^{2}+3)-x^{2}}{\sqrt{x^{2}+3}+x}\\
&  =\lim\limits_{x\rightarrow\infty}\frac{3}{\sqrt{x^{2}+3}+x}=0.
\end{align*}

\end{sol}

\begin{example}
Calcule $\lim\limits_{x\rightarrow\infty}\dfrac{6x^{3}+3x^{2}-8}%
{7x^{4}+16x^{2}-12}$
\end{example}

\begin{sol}
Tenemos un l\'{\i}mite de la forma indeterminada $\frac{\infty}{\infty}$. El
argumento usado en estos casos es dividir ambos polinomios por la mayor
potencia de $x.$ Observe que en nuestro problema el grado del numerador es
$3$, mientras el grado del denominador es $4,$ por lo que se dividir\'{a}
tanto numerador como denominador por $x^{4},$ obteniendo:%
\begin{align*}
\lim\limits_{x\rightarrow\infty}\dfrac{6x^{3}+3x^{2}-8}{7x^{4}+16x^{2}-12}  &
=\lim\limits_{x\rightarrow\infty}\dfrac{\dfrac{6x^{3}}{x^{4}}+\dfrac{3x^{2}%
}{x^{4}}-\dfrac{8}{x^{4}}}{\dfrac{7x^{4}}{x^{4}}+\dfrac{16x^{2}}{x^{4}}%
-\dfrac{12}{x^{4}}}\\
&  =\lim\limits_{x\rightarrow\infty}\dfrac{\dfrac{6}{x}+\dfrac{3}{x^{2}%
}-\dfrac{8}{x^{4}}}{7+\dfrac{16}{x^{2}}-\dfrac{12}{x^{4}}}\\
&  =\frac{0}{7}=0.
\end{align*}

\end{sol}

\begin{example}
Calcule $\lim\limits_{x\rightarrow\infty}\dfrac{6x^{2}+2x-1}{3x^{2}+x+1}$
\end{example}

\begin{sol}
En forma an\'{a}loga al ejemplo anterior, dividiendo por $x^{2}$ y aplicando
el teorema \ref{a7},%
\begin{align*}
\lim\limits_{x\rightarrow\infty}\dfrac{6x^{2}+2x-1}{3x^{2}+x+1}  &
=\lim\limits_{x\rightarrow\infty}\dfrac{\dfrac{6x^{2}}{x^{2}}+\dfrac{2x}%
{x^{2}}-\dfrac{1}{x^{2}}}{\dfrac{3x^{2}}{x^{2}}+\dfrac{x}{x^{2}}+\dfrac
{1}{x^{2}}}\\
&  =\lim\limits_{x\rightarrow\infty}\dfrac{6+\dfrac{2}{x}-\dfrac{1}{x^{2}}%
}{3+\dfrac{1}{x}+\dfrac{1}{x^{2}}}\\
&  =\frac{6}{3}=2
\end{align*}

\end{sol}

\begin{example}
Calcular el $\lim\limits_{x\rightarrow\infty}\dfrac{5x^{2}+1}{3x+1}$
\end{example}

\begin{sol}
Dividiendo por $x^{2}$ y aplicando el teorema \ref{a7}%
\begin{align*}
\lim\limits_{x\rightarrow\infty}\dfrac{5x^{2}+1}{3x+1}  &  =\lim
\limits_{x\rightarrow\infty}\dfrac{\dfrac{5x^{2}}{x^{2}}+\dfrac{1}{x^{2}}%
}{\dfrac{3x}{x^{2}}+\dfrac{1}{x^{2}}}\\
&  =\lim\limits_{x\rightarrow\infty}\dfrac{5+\dfrac{1}{x^{2}}}{\dfrac{3}%
{x}+\dfrac{1}{x^{2}}}\\
&  =\infty.
\end{align*}

\end{sol}

\begin{remark}
Compare los resultados obtenidos en los tres ejempos previos, con los
resultados obtenidos al aplicar la parte 9. del teorema \ref{a7}.
\end{remark}

\begin{example}
Sea $f(x)=\operatorname{sen}x\ $calcule $\lim\limits_{h\rightarrow0}%
\dfrac{f(x+h)-f\left(  x\right)  }{h}.$
\end{example}

\begin{sol}%
\begin{align*}
\lim\limits_{h\rightarrow0}\dfrac{f(x+h)-f\left(  x\right)  }{h}  &
=\lim\limits_{h\rightarrow0}\frac{\operatorname{sen}\left(  x+h\right)
-\operatorname{sen}x}{h}\\
&  =\lim\limits_{h\rightarrow0}\frac{\operatorname{sen}x\cos
h+\operatorname{sen}h\cos x-\operatorname{sen}x}{h}\\
&  =\lim\limits_{h\rightarrow0}\frac{\operatorname{sen}x(\cos h-1)}{h}%
+\lim\limits_{h\rightarrow0}\frac{\operatorname{sen}h\cos x}{h}\\
&  =\operatorname{sen}x.\lim\limits_{h\rightarrow0}\frac{\cos h-1}{h}+\cos
x.\lim\limits_{h\rightarrow0}\frac{\operatorname{sen}h}{h}\\
&  =\cos x.
\end{align*}

\end{sol}

\begin{example}
\label{a8}Halle las discontinuidades y las as\'{\i}ntotas de la funci\'{o}n
\[
f\left(  x\right)  =\dfrac{2x^{3}+x^{2}-13x+6}{x^{2}-1}.
\]

\end{example}

\begin{sol}
Se tiene una funci\'{o}n racional, estas funciones son discontinuas en los
puntos donde el denominador es igual a cero. Es decir, cuando $x^{2}-1=0$. Por
lo tanto la funci\'{o}n es discontinua en $x=\pm1.$

Para las as\'{\i}ntotas calculemos los l\'{\i}mites laterales
\begin{align*}
\lim\limits_{x\rightarrow1^{+}}\dfrac{2x^{3}+x^{2}-13x+6}{x^{2}-1}  &
=\lim\limits_{x\rightarrow1^{+}}\frac{\left(  2x-1\right)  \left(  x-2\right)
\left(  x+3\right)  }{(x-1)(x+1)}\\
&  =-\infty\\
\lim\limits_{x\rightarrow-1^{+}}\dfrac{2x^{3}+x^{2}-13x+6}{x^{2}-1}  &
=\lim\limits_{x\rightarrow-1^{+}}\frac{\left(  2x-1\right)  \left(
x-2\right)  \left(  x+3\right)  }{(x-1)(x+1)}\\
&  =-\infty.
\end{align*}
Por lo tanto, la recta $x=1$ y la recta $x=-1$ son as\'{\i}ntotas verticales.
La gr\'{a}fica de la funci\'{o}n no tiene as\'{\i}ntotas horizontales porque%
\[
\lim\limits_{x\rightarrow\infty}\dfrac{2x^{3}+x^{2}-13x+6}{x^{2}-1}=\infty.
\]


Ahora, para encontrar la as\'{\i}ntota oblicua observe que
\begin{align*}
\dfrac{2x^{3}+x^{2}-13x+6}{x^{2}-1}  &  =2x+1+\frac{7-11x}{x^{2}-1}\\
\lim\limits_{x\rightarrow\infty}\left|  \dfrac{2x^{3}+x^{2}-13x+6}{x^{2}%
-1}-\left(  2x+1\right)  \right|   &  =\lim\limits_{x\rightarrow\infty}\left|
\frac{7-11x}{x^{2}-1}\right|  =0
\end{align*}
entonces la recta $y=2x+1$ es una as\'{\i}ntota oblicua.
\end{sol}

\begin{example}
\label{a9}Halle las discontinuidades y las as\'{\i}ntotas de la funci\'{o}n.
\[
f(x)=\dfrac{x^{2}-4}{x^{2}+7x-18}%
\]

\end{example}

\begin{sol}
Se tiene una funci\'{o}n racional. Estas funciones son discontinuas en los
puntos donde el denominador es igual a cero; por lo tanto los valores de
discontinuidad pertenecen al conjunto%
\[
\{ x \mid x^{2}+7x-18 =0\} =\{ x \mid(x+9) ( x-2)=0\}=\{2,-9\}
\]
Por lo tanto la funci\'{o}n es discontinua en $x=2\ $y$\ $en $x=-9.$\newline
Para las as\'{\i}ntotas verticales calculemos los l\'{\i}mite cuando
$x\rightarrow2$ y cuando\ $x\rightarrow-9^{+}$%
\begin{align*}
\lim\limits_{x\rightarrow2}\dfrac{x^{2}-4}{x^{2}+7x-18}  &  =\lim
\limits_{x\rightarrow2}\frac{\left(  x-2\right)  \left(  x+2\right)  }{\left(
x+9\right)  \left(  x-2\right)  }=\lim\limits_{x\rightarrow2}\frac{\left(
x+2\right)  }{\left(  x+9\right)  }\\
&  =\frac{4}{11}\\
\lim\limits_{x\rightarrow-9^{+}}\dfrac{x^{2}-4}{x^{2}+7x-18}  &
=\lim\limits_{x\rightarrow-9^{+}}\frac{\left(  x-2\right)  \left(  x+2\right)
}{\left(  x+9\right)  \left(  x-2\right)  }=\lim\limits_{x\rightarrow-9^{+}%
}\frac{\left(  x+2\right)  }{\left(  x+9\right)  }\\
&  =-\infty
\end{align*}
entonces la recta $x=-9$ es una\ as\'{\i}ntota vertical. La recta $x=2$ no es
as\'{\i}ntota vertical porque el $\lim\limits_{x\rightarrow2}f\left(
x\right)  \ $existe, la existencia de este l\'{\i}mite implica que la
discontinuidad de $f(x)$ en $x=2$ sea evitable, por lo que se puede redefinir
la funci\'{o}n , tomando la forma
\[
g(x)=\left\{
\begin{tabular}
[c]{cc}%
$f(x)$ & $x\neq2$\\
$\dfrac{4}{11}$ & $x=2$%
\end{tabular}
\right.
\]
la funci\'{o}n g es es continua en $x=2$ y es ``casi''\ la misma funci\'{o}n
$f$. \newline Para las as\'{\i}ntotas horizontales calculemos $\lim
\limits_{x\rightarrow\infty}f\left(  x\right)  \ y\ \lim\limits_{x\rightarrow
-\infty}f\left(  x\right)  $%
\begin{align*}
\lim\limits_{x\rightarrow\infty}\dfrac{x^{2}-4}{x^{2}+7x-18}  &
=\lim\limits_{x\rightarrow\infty}\frac{1-\dfrac{4}{x^{2}}}{1+\dfrac{7}%
{x}-\dfrac{18}{x^{2}}}=1\\
\lim\limits_{x\rightarrow-\infty}\dfrac{x^{2}-4}{x^{2}+7x-18}  &
=\lim\limits_{x\rightarrow-\infty}\frac{1-\dfrac{4}{x^{2}}}{1+\dfrac{7}%
{x}-\dfrac{18}{x^{2}}}=1
\end{align*}
Por lo tanto, la recta $y=1$ es una as\'{\i}ntota horizontal. No hay
as\'{\i}ntotas oblicuas.
\end{sol}

\begin{example}
Halle las discontinuidades y las as\'{\i}ntotas de la funci\'{o}n.
\[
f(x)=\frac{\sqrt{x^{2}+15}}{x+1}
\]

\end{example}

\begin{sol}
Como la funci\'{o}n es discontinua en los puntos donde $x+1=0$, entonces es
discontinua en $x=-1$. Adem\'{a}s como
\[
\lim_{x\rightarrow-1^{+}}\frac{\sqrt{x^{2}+15}}{x+1}=\infty
\]
la recta $x=-1$ es una as\'{\i}ntota vertical y la discontinuidad en $x=-1$ es
esencial. Para hallar las as\'{\i}ntotas horizontales calculemos los limites
$\lim_{x\rightarrow\infty}\frac{\sqrt{x^{2}+15}}{x+1}$ y $\lim_{x\rightarrow
-\infty}\frac{\sqrt{x^{2}+15}}{x+1}$; tengamos en cuenta que cuando
$x\rightarrow\infty$ se debe tomar $x=\left|  x\right|  =\sqrt{x^{2}},$ debido
a que $x>0,$ y cuando $x\rightarrow-\infty\ $ se debe tomar $\ -x=\left|
x\right|  =\sqrt{x^{2}}$%
\begin{align*}
\lim_{x\rightarrow\infty}\frac{\sqrt{x^{2}+15}}{x+1}  &  =\lim_{x\rightarrow
\infty}\frac{\sqrt{x^{2}\left(  1+\frac{15}{x^{2}}\right)  }}{x\left(
1+\frac{1}{x}\right)  }=\lim_{x\rightarrow\infty}\frac{\left|  x\right|
\sqrt{1+\frac{15}{x^{2}}}}{x\left(  1+\frac{1}{x}\right)  }=1\\
\lim_{x\rightarrow-\infty}\frac{\sqrt{x^{2}+15}}{x+1}  &  =\lim_{x\rightarrow
-\infty}\frac{\left|  x\right|  \sqrt{1+\frac{15}{x^{2}}}}{x\left(  1+\frac
{1}{x}\right)  }=\lim_{x\rightarrow-\infty}\frac{-\sqrt{1+\frac{15}{x^{2}}}%
}{\left(  1+\frac{1}{x}\right)  }=-1
\end{align*}
entonces las rectas $y=1$ y $y=-1$ son as\'{\i}ntotas horizontales.
Obs\'{e}rvese que $f(7)=1$, esto significa que la funci\'{o}n corta la
as\'{\i}ntota horizontal. ( \textexclamdown \ construya la gr\'{a}fica!)
\end{sol}

\begin{example}
Halle la discontinuidades de $f(x)$.
\[
f\left(  x\right)  =\left\{
\begin{array}
[c]{ccc}%
3x-1 & \text{si} & x<-1\\
x^{2}+5x & \text{si} & -1\leq x\leq1\\
3x^{3} & \text{si} & 1<x
\end{array}
\right.
\]

\end{example}

\begin{sol}
Como $p(x)=3x-1$, $q(x)=x^{2}+5x$ y $r(x)=3x^{3}$ son polinomios, $f(x)$ es
continua en $(-\infty,-1)$$(-1,1)$ y $(1,+ \infty)$, por lo que falta discutir
la continuidad de la funci\'{o}n en $x=-1$ y $x=1$. \newline Inicialmente
veamos si la funci\'{o}n es continua en $x=-1$, para ello calculemos los
l\'{\i}mites laterales $\lim\limits_{x\rightarrow-1^{-}}f\left(  x\right)  $ y
$\lim\limits_{x\rightarrow-1^{+}}f\left(  x\right)  $, obteniendo:%

\[%
\begin{tabular}
[c]{c}%
$\lim\limits_{x\rightarrow-1^{-}}f\left(  x\right)  = \lim
\limits_{x\rightarrow-1^{-}}\left(  3x-1\right)  =-4 $\\
\\
$\lim\limits_{x\rightarrow-1^{+}}f\left(  x\right)  =\lim\limits_{x\rightarrow
-1^{-}}\left(  x^{2}+5x\right)  =-4$\\
\end{tabular}
\]


Como $\lim\limits_{x\rightarrow-1^{-}}f\left(  x\right)  =\lim
\limits_{x\rightarrow-1^{+}}f\left(  x\right)  =-4=f(-1)$, se tiene que la
funci\'{o}n es continua en $x=-1$

De igual forma, para $x=1$ se calculan los l\'{\i}mites laterales, con lo que
se obtiene:
\begin{align*}
\lim\limits_{x\rightarrow1^{-}}f\left(  x\right)   &  =\lim
\limits_{x\rightarrow1^{-}}\left(  x^{2}+5x\right)  =6\\
\lim\limits_{x\rightarrow1^{+}}f\left(  x\right)   &  =\lim
\limits_{x\rightarrow1^{-}}3x^{3}=3
\end{align*}
entonces la funci\'{o}n es discontinua en $x=1$ y la discontinuidad es
esencial porque el $\lim\limits_{x\rightarrow1}f\left(  x\right)  $ no existe.
Observe el salto que presenta la gr\'{a}fica en $x=1.$%
%TODO

\begin{center}
\includegraphics[scale=0.6]%
{ejr-2-8-29.pdf}%
\end{center}


\end{sol}

\begin{example}
Halle las discontinuidades de $f(x)=\left[  \left|  x\right|  \right]  $ y
determine cuales son removibles.
\end{example}

\begin{sol}
Recuerde que la funci\'{o}n
\index{Funci\'{o}n!-- parte entera}%
parte entera dada por $\left[  \left\vert x\right\vert \right]  =n$, donde $n$
es el mayor entero menor o igual que $x$ es decir
\[
f(x)=\left[  \left\vert x\right\vert \right]  =n,\,\hbox{donde}\,n\leq
x<n+1,n\in\mathbb{Z}.
\]
Observe que la funci\'{o}n es constante en los intervalos de la forma $\left[
n-1,n\right[  ,\ $y si $x\in\left[  n-1,n\right[  $, $\left[  \left\vert
x\right\vert \right]  =n-1$ y $x\in\left[  n,n+1\right[  $, $\left[
\left\vert x\right\vert \right]  =n$. Por lo tanto, la funci\'{o}n presenta un
salto en $x=n$, $n\in\mathbb{Z}$. Entonces para $n\in\mathbb{Z}$%
\begin{align*}
\lim_{x\rightarrow n^{+}}\left[  \left\vert x\right\vert \right]   &  =\left[
\left\vert n\right\vert \right]  =n\\
\lim_{x\rightarrow n^{-}}\left[  \left\vert x\right\vert \right]   &  =n-1
\end{align*}
Se tiene que la funci\'{o}n no es continua en $\mathbb{Z}$. Las
discontinuidades son esenciales, porque $\lim\limits_{x\rightarrow n}\left[
\left\vert x\right\vert \right]  $ con $n\in\mathbb{Z}$ no existe
\end{sol}

\begin{example}
Halle las discontinuidades y las as\'{\i}ntotas de la funci\'{o}n.
\[
f(x)=\frac{1}{\left[  \left|  x\right|  \right]  -x}
\]

\end{example}

\begin{sol}
La funci\'{o}n es discontinua en los valores que est\'{a}n en el conjunto $\{
x \mid\left[  \left|  x\right|  \right]  -x=0 \}$. Estos valores ocurren
cuando $x\in\mathbb{Z}$. Por lo tanto la funci\'{o}n es discontinua en
$\mathbb{Z}$ . Adem\'{a}s,
\begin{align*}
\lim\limits_{x\rightarrow n^{+}}\frac{1}{\left[  \left|  x\right|  \right]
-x}  &  =-\infty\text{ para\ }n\in\mathbb{Z}\ \\
\lim\limits_{x\rightarrow n^{-}}\frac{1}{\left[  \left|  x\right|  \right]
-x}  &  =\lim\limits_{x\rightarrow n^{-}}\frac{1}{(n-1)-n}=-1
\end{align*}
por lo tanto la recta $x=n$,$\ n\in\mathbb{Z}$. es una as\'{\i}ntota vertical.
\end{sol}

\begin{example}
Halle las discontinuidades y las as\'{\i}ntotas\ de la funci\'{o}n.
\[
f(x)=\frac{\operatorname{sen}x}{x}%
\]

\end{example}

\begin{sol}
La funci\'{o}n no es continua en $x=0$, pero
\[
\lim_{x\rightarrow0}\frac{\operatorname{sen}x}{x}=1
\]
por lo tanto la discontinuidad es evitable. \newline De otra parte como%
\[
\lim_{x\rightarrow\infty}\frac{\operatorname{sen}x}{x}=0
\]
la recta$\ y=0$ es una as\'{\i}ntota horizontal. Observe en la figura
\ref{senxentrex} que la funci\'{o}n intersecta la recta $y=0$ en infinitos puntos.%


\begin{figure}[H]
\centering
\includegraphics[scale=0.3]%
{ejr-2-8-32.pdf}%
\caption{Gr\'{a}fica de $f\left(  x\right)  =\frac{\operatorname{sen}x}{x}$}%
\label{senxentrex}%
\end{figure}


\end{sol}
%TODO 
\begin{example}
Determine los valores de las constantes $c$ y $k$ que hacen que la funci\'{o}n
sea continua.
\[
f(x)=\left\{
\begin{tabular}
[c]{cl}%
$x+2c$ & , si $\ x<-2$\\
$3cx+k$ & , si $\ -2\leq x\leq1$\\
$3x-2k$ & , si $\ \ x>1$%
\end{tabular}
\right.
\]

\end{example}

\begin{sol}
Para que la funci\'{o}n sea continua se debe tener que $\lim
\limits_{x\rightarrow-2}f(x)$ y $\lim\limits_{x\rightarrow1}f(x)$ existan. Es
decir, $\lim\limits_{x\rightarrow-2^{+}}f(x)=\lim\limits_{x\rightarrow-2^{-}%
}f(x)$ y $\lim\limits_{x\rightarrow1^{+}}f(x)=\lim\limits_{x\rightarrow1^{-}%
}f(x)$. Y como
\begin{align*}
\lim\limits_{x\rightarrow-2^{+}}f(x)  &  =\lim\limits_{x\rightarrow-2^{+}%
}(3cx+k)=-6c+k,\\
\lim\limits_{x\rightarrow-2^{-}}f(x)  &  =\lim\limits_{x\rightarrow-2^{-}%
}(x+2c)=-2+2c,\\
\lim\limits_{x\rightarrow1^{+}}f(x)  &  =\lim\limits_{x\rightarrow1^{+}%
}(3x-2k)=3-2k,\\
\lim\limits_{x\rightarrow1^{-}}f(x)  &  =\lim\limits_{x\rightarrow1^{-}%
}(3cx+k)=3c+k,
\end{align*}
se obtiene el sistema de ecuaciones
\[
\left\{
\begin{tabular}
[c]{ccc}%
$-6c+k$ & $=$ & $-2+2c$\\
$3-2k$ & $=$ & $3c+k$%
\end{tabular}
\right.
\]
cuyo conjunto soluci\'{o}n est\'{a} dado por: $c=\frac{1}{3}$ \ y
\ $k=\frac{2}{3}$.
\end{sol}

\begin{example}
Halle las discontinuidades y las as\'{\i}ntotas de la funci\'{o}n.
\[
f(x)=\left\{
\begin{tabular}
[c]{cl}%
$x^{2}$ & $x\leq0$\\
$\tan x$ & $0\leq x\leq\pi$\\
$x+1$ & $x\geq\pi$%
\end{tabular}
\right.
\]

\end{example}

\begin{sol}
La funci\'{o}n $\tan x=\dfrac{\operatorname{sen}x}{\cos x}$ $0\leq x\leq\pi$
es discontinua en los puntos donde $\cos x=0$, entonces es discontinua en
$x=\frac{\pi}{2}$ y como
\[
\lim_{x\rightarrow\frac{\pi}{2}^{+}}f(x)=-\infty,
\]
la recta $x=\dfrac{\pi}{2}$ es una as\'{\i}ntota vertical. Adem\'{a}s,
\begin{align*}
\lim_{x\rightarrow0^{+}}f(x)  &  =\lim_{x\rightarrow0^{+}}\tan x=0=\lim
_{x\rightarrow0^{-}}x^{2}=0=\lim_{x\rightarrow0^{-}}f(x)\\
\smallskip\lim_{x\rightarrow\pi^{+}}f(x)  &  =\lim_{x\rightarrow\pi^{+}%
}\left(  x+1\right)  =\pi+1\neq0=\lim_{\pi\rightarrow0^{-}}\tan x=\lim
_{x\rightarrow\pi^{-}}f(x)\\
&  \lim_{x\rightarrow\pi^{+}}f(x)
\end{align*}
Por lo tanto, se concluye que la funci\'{o}n es discontinua en $x=\pi$ y en
$x=\frac{\pi}{2}$
\end{sol}

\begin{example}
Sea $h\left(  x\right)  =\left(  f\circ g\right)  \left(  x\right)  $ donde
$g(x)=\sqrt{x-1}$ y $f\left(  x\right)  =x^{2}+3.$ \textquestiondown Donde h
es continua?
\end{example}

\begin{sol}
$g:\left[  1,\infty\right[  \rightarrow\left[  0,\infty\right[  $ es una
funci\'{o}n continua en $\left[  1,\infty\right[  $, $f$ es continua en todo
$\rz$ y por lo tanto continua en$\ \left[  0,\infty\right[  $. Aplicando el
teorema \ref{a10} se tiene que la funci\'{o}n $h$ es continua $\left[
1,\infty\right[  .$ Obs\'{e}rvese que%
\begin{align*}
h\left(  x\right)   &  =f[g(x)]=f[\sqrt{x-1}]\text{ }x\in\left[
1,\infty\right[ \\
&  =\left(  \sqrt{x-1}\right)  ^{2}+3=x-1+3=x+2\text{, para }x\in\left[
1,\infty\right[
\end{align*}

\end{sol}

\newpage

\section{Ejercicios propuestos}

\begin{enumerate}
\item Demuestre que todo intervalo abierto es una vecindad de todos sus puntos.

\item Un conjunto $A$ de n\'{u}meros reales se denomina
\index{Conjunto!-- abierto}%
abierto si es vecindad de todos sus puntos. \textquestiondown Cuales de los
siguientes conjuntos son abiertos? ($a$ y $b$ son reales fijos dados).

\begin{enumerate}
\item $\rz$.

\item $\nz$.

\item $\gz$.

\item $\qz$.

\item $[2,3[$

\item $[a,b[$.

\item $[a,b]$.

\item $]a,b]$.

\item La uni\'{o}n de dos intervalos abiertos (de radio finito).

\item La intersecci\'{o}n de dos intervalos abiertos.
\end{enumerate}

\item Muestre que un conjunto finito no es abierto. \textquestiondown Puede
afirmarse lo mismo para conjuntos contables en general?

\item Determine, en cada caso, el conjunto de todos los puntos de
acumulaci\'{o}n del conjunto dado.

\begin{enumerate}
\item $[0,2]$, $]0,2]$ y $[0,+\infty\lbrack$.

\item $[a,b]$, donde $a$ y $b$ son reales dados.

\item $\nz$, $\qz$ y $\rz$.

\item El rango de la sucesi\'{o}n $\left\{  \frac{1}{n}\right\}  _{n\in\nz}$.
\end{enumerate}

\item Un conjunto de n\'{u}meros reales es%
\index{Conjunto!cerrado}
cerrado si contiene al conjunto de todos sus puntos de acumulaci\'{o}n.
Demuestre que un conjunto es ce\-rrado si, y solo si su complemento es
abierto. Muestre que $\rz$ es abierto y cerrado al mismo tiempo, lo mismo que
el conjunto vac\'{\i}o. \textquestiondown Existen conjuntos que no sean ni
abiertos ni cerrados?

\item En cada caso determine, si existe, el l\'{\i}mite de la sucesi\'{o}n
$\{x_{n}\}$ dada. En caso de no existir, indique si la sucesi\'{o}n diverge o
no a $+\infty$. Justifique todas sus respuestas, indicando los teoremas y
propiedades aplicadas.

\begin{enumerate}
\item $x_{n}=n^{3}$.

\item $x_{n}=\dfrac{1}{n+3}$.

\item $x_{n}=\dfrac{n}{3n-4}$.

\item $x_{n}=\sqrt{n^{2}+n+1}$.

\item $x_{n}=\dfrac{3n^{2}-5n+6}{5n^{2}+3n-4}$.

\item $x_{n}=\dfrac{n^{4}+3n^{3}-n^{2}+5}{n^{5}-n^{4}}$.

\item $x_{n}=\dfrac{\sqrt{n+1}}{\sqrt{n+2}}$.

\item $x_{n}=\sqrt{\dfrac{n^{3}+n^{2}+n}{4n^{3}+n^{2}+3n}}$.

\item $x_{n}=\left(  \dfrac{x^{2}}{x^{2}+1}\right)  ^{n}$, donde $x$ es un
real fijo.

\item $x_{n}=\dfrac{(-1)^{n}}{(2n)!}$.

\item $x_{n}=\sqrt[n]{2}+\sqrt{3^{-n}}$.

\item $x_{n}=(x+2)^{n}$, $x\in\rz$.

\item $x_{n}=\sum\limits_{k=0}^{n}\dfrac{-1}{5^{n}}$.

\item $x_{n}=\sum\limits_{k=0}^{n}\sqrt{2^{-n}}$.

\item $x_{n}=\sum\limits_{k=3}^{n}2^{-n}$.

\item $x_{n}=\sum\limits_{k=0}^{n}x^{2n}$.
\end{enumerate}

\item Muestre que si una sucesi\'{o}n real converge, su l\'{\i}mite es un real \'{u}nico.

\item Demuestre que si $x_{n}\rightarrow+\infty$, entonces $\frac{1}{x_{n}%
}\rightarrow0$.

\item Demuestre que si una sucesi\'{o}n es decreciente, entonces converge si,
y solo si es acotada.

\item Demuestre el siguiente criterio de
\index{Cauchy!Criterio de --}%
Cauchy para la convergencia de
\index{Sucesiones!Convergencia de --}%
sucesiones reales:\newline$\{x_{n}\}_{n\in\nz}$ es convergente si, y solo si
para todo real $r>0$, existe un n\'{u}mero natural $N$ tal que siempre que
$n,m\geq N$, entonces
\[
|x_{n}-x_{m}|<r.
\]


\item Dada una sucesi\'{o}n real $\{x_{n}\}_{n\in\nz}$, una sucesi\'{o}n
$\{y_{n}\}_{n\in\nz}$ es una subsucesi\'{o}n
\index{Subsucesi\'{o}n|textbf}%
de la primera si existe una funci\'{o}n $f:\nz\longrightarrow\nz$,
estrictamente creciente ($f(n)>f(m)$, siempre que $n>m$) tal que para todo
$n\in\nz$ se tiene $y_{n}=x_{f(n)}$. Demuestre que $\{x_{n}\}_{n\in\nz}$ es
convergente a $L$ si, y solo si lo es toda subsucesi\'{o}n de ella.

\item Demuestre que si $L$ es un punto de acumulaci\'{o}n de un conjunto
$A\subseteq\rz$, entonces existe una sucesi\'{o}n, de rango infinito,
contenida en $A$ y que converge a $L$.

\item Demuestre que toda sucesi\'{o}n acotada contiene una subsucesi\'{o}n convergente.

\item En cada caso, determine en $\rz^{\ast}$, si existen, los l\'{\i}mites
indicados. Justifique brevemente sus c\'{a}lculos.

\begin{enumerate}
\item $\lim\limits_{x\rightarrow2}(x^{3}+3x^{2}-2x-10)$.

\item $\lim\limits_{x\rightarrow0}\dfrac{\sqrt{x+1}}{x+3}$.

\item $\lim\limits_{x\rightarrow2}\dfrac{x^{2}-4}{x-2}$.

\item $\lim\limits_{x\rightarrow a}\dfrac{x^{n}-a^{n}}{x-a}$.

\item $\lim\limits_{x\rightarrow0}\dfrac{\sqrt{a+x}-\sqrt{a}}{x}$.

\item $\lim\limits_{x\rightarrow3}\dfrac{x^{3}-27}{x-3}$.

\item $\lim\limits_{x\rightarrow+\infty}\dfrac{x^{2}+3x-2}{3x^{2}+x-1}$.

\item $\lim\limits_{x\rightarrow-\infty}\dfrac{x^{2}+3x-2}{3x^{2}+x-1}$.

\item $\lim\limits_{x\rightarrow+\infty}\dfrac{\sqrt{x^{2}+x+1}}{x-1}$.

\item $\lim\limits_{x\rightarrow-\infty}\dfrac{\sqrt{x^{2}+1}}{x-1}$.

\item $\lim\limits_{x\rightarrow+\infty}(\sqrt{x^{2}+1}-x)$.

\item $\lim\limits_{x\rightarrow+\infty}(x^{3}+x+2)$.

\item $\lim\limits_{x\rightarrow-\infty}(x^{3}+x+2)$.

\item $\lim\limits_{t\rightarrow+\infty}(\sqrt{t^{2}+1}-\sqrt{t^{2}-1})$.

\item $\lim\limits_{t\rightarrow-\infty}(\sqrt{t^{2}+1}-\sqrt{t^{2}-1})$.

\item $\lim\limits_{\theta\rightarrow+\infty}\dfrac{cos\theta}{\theta}$.
\end{enumerate}

\item Sea $f(x)=\left\{
\begin{array}
[c]{ccc}%
x^{3}+1 &  & ,\text{ si }x<0\\
2 &  & ,\text{ si }x=0\\
x^{2}+x+1 &  & ,\text{ si }x>0
\end{array}
\right.  $. Determine:

\begin{enumerate}
\item $\lim\limits_{x\rightarrow0^{+}}f(x)$.

\item $\lim\limits_{x\rightarrow0^{-}}f(x)$.

\item $\lim\limits_{x\rightarrow0}f(x)$.

\item $\lim\limits_{x\rightarrow2}f(x)$.

\item $\lim\limits_{x\rightarrow-2}f(x)$.

\item $\lim\limits_{h\rightarrow0^{+}}\dfrac{f(h)-f(0)}{h}$.

\item $\lim\limits_{h\rightarrow0^{-}}\dfrac{f(h)-f(0)}{h}$.

\item $\lim\limits_{h\rightarrow0}\dfrac{f(1+h)-f(1)}{h}$.
\end{enumerate}

\item Si $x\in\rz$, la parte entera de $x$ es el mayor entero menor o igual a
$x$. Se denota por $[|x|]$. Definamos la funci\'{o}n parte entera por
$f(x)=[|x|]$, para todo $x\in\rz$. Haga una gr\'{a}fica de $f$ y Muestre que
$\lim_{x\rightarrow c}f(x)$ existe si, y solo si $c\notin\gz$
\textquestiondown Qu\'{e} sucede si $c\in\gz$ ?

\item Demuestre que para todo real $c$:
\[
\lim_{x\rightarrow c}|x|=|c|.
\]


\item Sea $f$ una funci\'{o}n real tal que $\lim_{x\rightarrow c}f(x)=k>0$.
Demuestre que si $g(x)\rightarrow0^{+}$, cuando $x\rightarrow c$, entonces
$(f/g)(x)\rightarrow+\infty$ cuando $x\rightarrow c$. Establezca resultados
similares para l\'{\i}mites unilaterales. Igualmente para cuando $k<0$ y/o si
$g(x)\rightarrow0^{-}$.

\item Utilice el ejercicio anterior para decidir la divergencia a $+\infty$ o
$-\infty$ de las funciones indicadas a continuaci\'{o}n.

\begin{enumerate}
\item $f(x)=\dfrac{x+3}{x^{2}+2x+1}$, cuando $x\rightarrow-1$.

\item $f(x)=-\dfrac{\sqrt{x+3}}{x^{2}+2x+1}$, cuando $x\rightarrow-1$.

\item $f(x)=\dfrac{x+3}{x-2}$, cuando $x\rightarrow2$.

\item $f(x)=\dfrac{x}{x^{2}-1}$, cuando $x\rightarrow1$.

\item $f(x)=\dfrac{x}{x^{2}-1}$, cuando $x\rightarrow-1$.
\end{enumerate}

\item En cada uno de los siguientes ejercicios calcular el l\'{\i}mite (en
caso de existir) .

\begin{enumerate}
\item $\lim\limits_{x\rightarrow1}\dfrac{x^{2}+3x-5}{x^{2}+7}.$

\item $\lim\limits_{x\rightarrow2}\dfrac{3x^{2}+2x-16}{x^{2}+5x-14}.$

\item $\lim\limits_{x\rightarrow3}\dfrac{x^{3}-3x^{2}-x+3}{x^{3}-5x^{2}%
+3x+9}.$

\item $\lim\limits_{x\rightarrow0}\dfrac{\sqrt{x+2}-\sqrt{2}}{x}.$

\item $\lim\limits_{x\rightarrow1}\dfrac{5-\sqrt{x^{2}+24}}{\sqrt{x+8}-3}.$

\item $\lim\limits_{x\rightarrow0}x^{2}\operatorname{sen}\left(  \dfrac
{1}{\sqrt[3]{x}}\right)  .$

\item $\lim\limits_{x\rightarrow0}\dfrac{\sqrt[3]{\left(  x+2\right)  ^{2}%
}-\sqrt[3]{4}}{x}.$

\item $\lim\limits_{x\rightarrow8}\dfrac{\sqrt{7+\sqrt[3]{x}}-3}{x-8}.$

\item $\lim\limits_{x\rightarrow2^{+}}\dfrac{\left[  \left\vert x\right\vert
\right]  -1}{\left[  \left\vert x\right\vert \right]  -x}.$

\item $\lim\limits_{x\rightarrow4^{+}}\dfrac{3x}{\sqrt{x^{2}-16}}.$

\item $\lim\limits_{x\rightarrow1}\dfrac{\sqrt{x^{3}+2x^{2}-1}-\sqrt
{9x^{3}-5x^{2}-4x}}{\sqrt{3x^{2}+3x-4}-\sqrt{8x^{3}+3x^{2}+7}}.$

\item $\lim\limits_{x\rightarrow-\infty}\left(  \sqrt{x^{2}-2x}-\sqrt{x^{2}%
+7}\right)  .$

\item $\lim\limits_{x\rightarrow-1^{+}}\dfrac{9x^{3}+3x^{2}-5x+1}{x^{2}%
+2x+1}.$

\item $\lim\limits_{x\rightarrow2^{+}}\dfrac{\sqrt{x^{2}-4}}{x-2}.$

\item $\lim\limits_{x\rightarrow3^{-}}\dfrac{\sqrt{9-x^{2}}}{x-3}.$

\item $\lim\limits_{x\rightarrow1^{+}}\dfrac{x-1}{\sqrt{2x-x^{2}}-1}.$

\item $\lim\limits_{x\rightarrow2^{-}}\dfrac{x-2}{2-\sqrt{4x-x^{2}}}.$

\item $\lim\limits_{x\rightarrow1^{-}}\dfrac{\left[  \left\vert x^{2}%
\right\vert \right]  -1}{x^{2}-1}.$

\item $\lim\limits_{t\rightarrow3}\dfrac{\sqrt{t-1}}{t^{2}-6t+9}.$

\item $\lim\limits_{x\rightarrow0}\dfrac{\sqrt{x^{2}+1}-\sqrt[4]{x+1}}{x^{2}%
}.$

\item $\lim\limits_{x\rightarrow1}\dfrac{x^{5}-4x^{3}+2x+1}{x^{3}+2x-3}.$

\item $\lim\limits_{x\rightarrow a^{+}}\dfrac{\sqrt{a}-\sqrt{x}-\sqrt{x-a}%
}{\sqrt{x^{2}-a^{2}}}.$
\end{enumerate}

\item Calcule los siguientes l\'{\i}mites

\begin{enumerate}
\item $\lim\limits_{x\rightarrow\frac{\pi}{6}}\frac{\sqrt{3}}{2}\left(
\dfrac{2\sin x-1}{x-\frac{\pi}{6}}\right)  .$

\item $\lim\limits_{x\rightarrow\frac{\pi}{2}}\dfrac{1-\sin x}{\cos x}.$

\item $\lim\limits_{y\rightarrow0^{+}}\dfrac{\sin4y}{\cos3y-1}.$

\item $\lim\limits_{x\rightarrow0}\dfrac{2\tan^{2}x}{x^{2}}.$

\item $\lim\limits_{x\rightarrow0}\dfrac{3x^{2}+5x}{\operatorname{sen}2x}.$

\item $\lim\limits_{x\rightarrow\infty}\dfrac{x^{2}-\operatorname{sen}%
x}{2x^{2}+\cos x}.$

\item $\lim\limits_{x\rightarrow0}f\left(  x\right)  ,$ si
\[
f\left(  x\right)  =\left\{
\begin{array}
[c]{lll}%
\dfrac{1-\cos x^{2}}{x^{3}} & ,\text{ si} & x<0\\
2\cos\left(  x+\frac{\pi}{3}\right)  -1 & ,\text{ si} & x\geq0.
\end{array}
\right.
\]


\item $\lim\limits_{x\rightarrow0}\dfrac{\tan^{5}\left(  2x\right)  }{3x^{5}%
}.$

\item $\lim\limits_{x\rightarrow0}\dfrac{1-\cos5x}{\sin4x}.$

\item $\lim\limits_{x\rightarrow0}\dfrac{\cos2x}{5x^{2}+3x}.$
\end{enumerate}

\item Calcule los siguientes l\'{\i}mites

\begin{enumerate}
\item
\[
\lim\limits_{x\rightarrow\frac{1}{2}^{+}}\frac{\allowbreak30x^{2}%
-13x-28x^{3}+8x^{4}+2}{\allowbreak10x^{2}-x+4x^{3}-64x^{4}+64x^{5}-\frac{1}%
{2}}.
\]


\item
\[
\lim\limits_{x\rightarrow\frac{1}{2}^{-}}\frac{11x-18x^{2}+4x^{3}+8x^{4}%
-2}{12x^{3}-4x^{2}-5x+2}.
\]


\item
\[
\lim\limits_{x\rightarrow+\infty}\left(  \frac{1}{\sqrt{2x^{2}+5x+1}%
-\sqrt{2x^{2}+3x}}\right)  .
\]


\item
\[
\lim\limits_{x\rightarrow-\infty}\left(  \sqrt[3]{27x^{3}+4x+1}-\sqrt[3]%
{27x^{3}+3x^{2}}\right)  .
\]


\item
\[
\lim\limits_{x\rightarrow-\infty}\left(  \frac{\sqrt{x^{2}+2+\sqrt
{x^{4}+4+\sqrt{x^{8}+8}}}}{\sqrt[3]{x^{3}+3}}\right)  .
\]


\item
\[
\lim\limits_{x\rightarrow1}\frac{\left\vert 3x+4x^{2}-6x^{3}-2x^{4}%
+3x^{5}-2\right\vert }{x^{3}-x^{2}-x+1}.
\]


\item
\[
\lim\limits_{x\rightarrow1}\frac{\left\vert x^{2}-1\right\vert -x^{2}+1}%
{x-1}.
\]

\end{enumerate}

\item En los ejercicios siguientes hallar las discontinuidades y
as\'{\i}ntotas de las siguientes funciones.

\begin{enumerate}
\item $f\left(  x\right)  =\dfrac{x-1}{x^{2}-1}.$

\item $f\left(  x\right)  =\dfrac{5x^{2}-6x-8}{x^{2}+5x-6}.$

\item $f\left(  x\right)  =\dfrac{x^{3}-x^{2}-4x+4}{x^{2}-1}.$

\item $f\left(  x\right)  =\dfrac{1-\sqrt{1-4x^{2}}}{x^{2}}.$

\item $f\left(  x\right)  =\dfrac{\operatorname{sen}\left(  x^{2}-1\right)
}{x-1}.$

\item $f(x)=\dfrac{x-1}{\left\vert x^{2}-1\right\vert }.$

\item $f\left(  x\right)  =\left\{
\begin{tabular}
[c]{ccc}%
$\dfrac{1}{x+1}$ & , si & $x<-1$\\
$\sqrt{1-x^{2}}$ & , si & $-1\leq x<1$\\
$1-x$ & , si & $x\geq1.$%
\end{tabular}
\right.  $

\item $f\left(  x\right)  =\left\{
\begin{tabular}
[c]{ccc}%
$x+\dfrac{\pi}{2}$ & , si & $x<-\dfrac{\pi}{2}$\\
$\tan x$ & , si & $-\dfrac{\pi}{2}<x<\dfrac{\pi}{2}$\\
$\dfrac{1}{x+\pi}$ & , si & $x\geq\dfrac{\pi}{2}.$%
\end{tabular}
\right.  $
\end{enumerate}

\item Pruebe las siguientes igualdades

\begin{enumerate}
\item
\[
\lim_{x\rightarrow0}\dfrac{1-x}{1-\operatorname{sen}\dfrac{\pi x}{2}%
}=\allowbreak1.
\]


\item
\[
\lim_{x\rightarrow0}\left(  \dfrac{1}{\operatorname{sen}^{2}x}-\dfrac{1}%
{x^{2}}\right)  =\allowbreak\frac{1}{3}.
\]


\item
\[
\lim_{x\rightarrow0}\left(  1-\cos x\right)  \cot x=\allowbreak0.
\]


\item
\[
\lim_{x\rightarrow\infty}x\operatorname{sen}\dfrac{a}{x}=\allowbreak a.
\]


\item
\[
\lim_{x\rightarrow\tfrac{\pi}{2}}\left(  \dfrac{x}{\cot x}-\dfrac{\pi}{2\cos
x}\right)  =\allowbreak-1.
\]


\item
\[
\lim_{x\rightarrow1}\left(  \dfrac{1}{2\left(  1-\sqrt{x}\right)  }-\dfrac
{1}{3\left(  1-\sqrt[3]{x}\right)  }\right)  =\allowbreak\frac{1}{12}.
\]


\item
\[
\lim_{x\rightarrow0}\dfrac{\tan8x}{\tan5x}=\allowbreak\frac{8}{5}.
\]


\item
\[
\lim_{x\rightarrow-2^{+}}\dfrac{3x^{4}+11x^{2}+12x^{3}-4x-4}{5x^{4}%
+34x^{3}+84x^{2}+88x+32}=\allowbreak-\infty.
\]


\item
\[
\lim_{x\rightarrow0}\left(  \dfrac{4}{x^{2}}-\dfrac{2}{1-\cos x}\right)
=\allowbreak-\frac{1}{3}.
\]


\item
\[
\lim_{x\rightarrow0^{+}}\dfrac{x-\operatorname{sen}x}{\left(
x\operatorname{sen}x\right)  ^{\frac{3}{2}}}=\allowbreak\frac{1}{6}.
\]

\end{enumerate}

\item Calcular el l\'{\i}mite:
\[
\lim_{h\rightarrow0}\dfrac{f(x+h)-f\left(  x\right)  }{h}%
\]
para cada una de las siguientes funciones.

\begin{enumerate}
\item $f\left(  x\right)  =x^{2}.$

\item $f(x)=x^{3}.$

\item $f(x)=\sqrt{x+1}.$

\item $f(x)=\dfrac{1}{\sqrt{x}}.$

\item $f(x)=\dfrac{1}{x}.$

\item $f(x)=\cos x.$
\end{enumerate}

\item En los siguientes ejercicios se definen las funciones $f$ y $g$. en cada
ejercicio encontrar una formula para $h(x)=\left(  f\circ g\right)  (x)$ y
determinar los intervalos en los que $h$ es continua

\begin{enumerate}
\item $f(x)=x^{2}+1$ \ y \ $g(x)=\sqrt{x}.$

\item $f(x)=\sqrt{x}$ \ y \ $g(x)=\dfrac{1}{x-2}.$

\item $f(x)=\dfrac{1}{x-2}$ \ y \ $g(x)=\sqrt{x}.$

\item $f(x)=x^{2}+1$ \ y \ $g(x)=\sqrt{x-1}.$
\end{enumerate}

\item D\'{e} un ejemplo de una funci\'{o}n que no sea continua en $x=1$ para
la cual $\lim\limits_{x\rightarrow1}f(x)$ existe pero $f(1)$ no existe.

\item D\'{e} un ejemplo de una funci\'{o}n que no sea continua en $x=1$ para
la cual existen $\lim\limits_{x\rightarrow1}f(x)$ y $f(1).$

\item Sea $f\left(  x\right)  $ una funci\'{o}n continua en el intervalo
$[-a,a]$. Probar que
\[
\lim_{x\rightarrow0}x^{2}f\left(  x\right)  =0.
\]


\item Demuestre que la ecuaci\'{o}n $x^{3}+x-1=0$ tiene una soluci\'{o}n en el
intervalo $[0,1].$

\item Supongamos que $\lim\limits_{x\rightarrow a}g(x)=0,$ de un ejemplo donde
la funcion $h(x)=\dfrac{f(x)}{g(x)}$ tenga una discontinuidad evitable en
$x=a.$

\item la funci\'{o}n $\dfrac{\operatorname{sen}x}{x}$ no es continua en $x=0$,
redefina la funci\'{o}n tal que sea continua en $x=0.$

\item Determine si las funciones dadas son continuas en el punto dado. En caso
de ser discontinua en el punto clasifique la discontinuidad y en caso de que
la discontinuidad es evitable(removible), redefina la funci\'{o}n

\begin{enumerate}
\item
\[
f\left(  x\right)  =\left\{
\begin{array}
[c]{ccc}%
\dfrac{\sin x-x}{\frac{1}{6}x^{2}} & \text{, si} & x\neq0\\
0 & \text{, si} & x=0
\end{array}
\right.
\]
en $x=0.$

\item
\[
f\left(  x\right)  =\left\{
\begin{tabular}
[c]{ccc}%
$x^{2}+3x-2$ & , si & $x\leq1$\\
$5x-3$ & , si & $x>1$%
\end{tabular}
\ \right.
\]
en $x=1.$

\item
\[
f\left(  x\right)  =\left\{
\begin{array}
[c]{lll}%
5\operatorname{sen}x-2 &  & \text{si }x\leq\tfrac{\pi}{2}\\
3\cos x+3 &  & \text{si }x>\tfrac{\pi}{2}%
\end{array}
\right.
\]
en $x=\dfrac{\pi}{2}.$

\item
\[
h\left(  x\right)  =\left\{
\begin{array}
[c]{ccc}%
x^{2}-1 & \text{, si} & x<-1\\
10-10x^{2} & \text{, si} & -1\leq x\leq1\\
x^{2}-1 & \text{, si} & 1<x
\end{array}
\right.
\]
en $x=-1$ y en $x=1.$

\item
\[
f\left(  x\right)  =\left\{
\begin{array}
[c]{ccc}%
6-x & \text{, si} & x\leq2\\
x^{3}-2x & \text{, si} & x>2
\end{array}
\right.
\]
en $x=2.$

\item
\[
f\left(  x\right)  =\left\{
\begin{array}
[c]{lll}%
\dfrac{1-\cos x^{2}}{x^{3}} &  & \text{si }x<0\\
2\cos\left(  x+\frac{\pi}{3}\right)  -1 &  & \text{si }x\geq0
\end{array}
\right.
\]
en $x=0.$

\item
\[
f\left(  x\right)  =\left\{
\begin{array}
[c]{lll}%
x^{3}-1 &  & \text{si }x\leq1\\
3x-3 &  & \text{si }x>1
\end{array}
\right.
\]
en $x=1.$

\item
\[
g\left(  x\right)  =\left\{
\begin{array}
[c]{ccc}%
\dfrac{5x^{3}-21x^{2}+24x-4}{2x^{3}-5x^{2}-4x+12} &  & \text{si }x<2\\
&  & \\
\dfrac{9}{7} &  & \text{si }x=2\\
&  & \\
-\dfrac{15}{28}\left(  \dfrac{\sqrt[3]{14x^{3}+3x^{2}+1}-5}{x-2}\right)  &  &
\text{si }x>2
\end{array}
\right.
\]
en $x=2.$
\end{enumerate}

\item Una funci\'{o}n $f$ est\'{a} definida como sigue:
\[
f\left(  x\right)  =\left\{
\begin{tabular}
[c]{ccc}%
$3ax^{3}-2b$ & , si & $x\leq-1$\\
$2x-4b$ & ,si & $-1<x\leq1$\\
$ax^{2}+3$ & , si & $1\leq x$%
\end{tabular}
\ \ \ \right.
\]
Obtenga valores de $a$ y $b$ para los cuales la funci\'{o}n sea continua en
$x=1$ y en $x=-1.$

\item Una funci\'{o}n $f$ est\'{a} definida como sigue:
\[
f\left(  x\right)  =\left\{
\begin{array}
[c]{lll}%
5\operatorname{sen}x-2 &  & \text{si }x\leq\dfrac{\pi}{2}\\
\dfrac{ax}{\pi}+b &  & \text{si }\dfrac{\pi}{2}<x<\pi\\
3\cos x+4 &  & \text{si }x\geq\pi
\end{array}
\right.
\]
siendo $a$ y $b$ constantes. Determine valores de las constantes $a$ y $b$,
para que la funci\'{o}n sea continua.

\item Determine si las funciones dadas a continuaci\'{o}n son continuas. En
caso de presentarse puntos de discontinuidad, clasifique dicha discontinuidad
y en caso de que la discontinuidad sea evitable o removible, redefina la funci\'{o}n.

\begin{enumerate}
\item $f\left(  x\right)  =\left\{
\begin{tabular}
[c]{ccc}%
$\dfrac{5-\sqrt{x^{2}+24}}{\sqrt{x+8}-3}$ & , si & $x<1$\\
$1$ & ,si & $x=1$\\
$\dfrac{15x-9x^{2}+x^{3}-7}{8x^{2}-13x-x^{3}+6}$ & , si & $1<x.$%
\end{tabular}
\ \ \ \ \right.  $

\item $f\left(  x\right)  =\left\{
\begin{tabular}
[c]{ccc}%
$\dfrac{3x^{2}+2x-16}{x^{2}+5x-14}$ & , si & $x<2$\\
$-1$ & ,si & $x=2$\\
$\dfrac{\sqrt{7+x}-3}{x^{2}-4}$ & , si & $2<x.$%
\end{tabular}
\ \ \ \ \right.  $

\item $f\left(  x\right)  =x\operatorname{sen}\dfrac{1}{x}.$

\item $g\left(  x\right)  =\dfrac{x^{2}-2x-8}{x-4}.$

\item $f\left(  t\right)  =\dfrac{7t^{3}+4t}{2t^{3}-t^{2}+3}.$

\item $h\left(  x\right)  =\dfrac{1}{\left\vert x\right\vert +1}-\dfrac{x^{2}%
}{2}.$

\item $f\left(  x\right)  =\dfrac{x+2}{\cos x}.$

\item $h\left(  x\right)  =\dfrac{x^{2}+3x-10}{x-2}.$

\item $f\left(  x\right)  =\dfrac{\left\vert x\right\vert }{x^{2}}.$

\item $g\left(  x\right)  =e^{-\tfrac{1}{x^{2}}}.$

\item $g\left(  x\right)  =\left\{
\begin{tabular}
[c]{ccc}%
$\dfrac{\cos x-1}{x^{2}}$ & , si & $x>0$\\
$1$ & , si & $x=0$\\
$\dfrac{-x^{2}-x}{2\operatorname{sen}x}$ & , si & $x<0.$%
\end{tabular}
\ \ \ \right.  $

\item $h\left(  z\right)  =\dfrac{\sqrt{7+z}-3}{z^{2}-4}.$

\item $f\left(  x\right)  =\left\{
\begin{tabular}
[c]{ccc}%
$1-x$ & , si & $x\leq2$\\
$x^{3}-2x$ & , si & $x>2.$%
\end{tabular}
\ \ \ \right.  $

\item $g\left(  x\right)  =\dfrac{1-\cos x}{x^{2}}.$
\end{enumerate}

\item Determine la convergencia o divergencia de cada una de las siguientes
sucesiones. Si convergen establezca el l\'{\i}mite

\begin{enumerate}
\item $a_{n}=\left\{  \dfrac{n^{2}-n+1}{n^{3}+2}\right\}  _{n=0}^{\infty}.$

\item $a_{n}=\left\{  \dfrac{5n^{2}-5n+2}{3n^{2}-4}\right\}  _{n=0}^{\infty}.$

\item $a_{n}=\left\{  \dfrac{5n^{3}-6}{n+2}\right\}  _{n=0}^{\infty}.$

\item $a_{n}=\left\{  \sqrt{n}-\sqrt{n-2}\right\}  _{n=2}^{\infty}.$

\item $a_{n}=\left\{  \dfrac{\sin n}{n}\right\}  _{n=1}^{\infty}.$

\item $a_{n}=\left\{  \ln\sqrt{n}-\ln\sqrt{n-2}\right\}  _{n=2}^{\infty}.$

\item $a_{n}=\left\{  \left(  \frac{1}{5}\right)  ^{n}\right\}  _{n=1}%
^{\infty}.$

\item $a_{n}=\left\{  \sqrt[n]{10}\right\}  _{n=1}^{\infty}.$

\item $a_{n}=\left\{  \sqrt[n]{n-1}\right\}  _{n=1}^{\infty}.$

\item $a_{n}=\left\{  \dfrac{3^{n}}{n!}\right\}  _{n=0}^{\infty}.$

\item $a_{n}=\left\{  n2^{-n}\right\}  _{n=0}^{\infty}.$

\item $a_{n}=\left\{  \dfrac{3^{n}}{n!}\right\}  _{n=0}^{\infty}.$

\item $a_{n}=\left\{  \dfrac{n!}{\left(  n+2\right)  !}\right\}
_{n=0}^{\infty}.$

\item $a_{n}=\left\{  \left(  -1\right)  ^{n}\sin\left(  n^{-1}\right)
\right\}  _{n=1}^{\infty}.$
\end{enumerate}
%TODO revisar que pasa con \ldots
%\item Utilizar el resultado obtenido en la serie geométrica para expresar como
%racional los siguientes decimales infinitos periódicos.
%
%\begin{enumerate}
%\item $0.55555\ldots$
%
%\item $3.25252525\ldots$
%
%\item $4.247777777\ldots$
%
%\item $8.037373737\ldots$
%
%\item $2.99999\ldots$
%\end{enumerate}

\item Calcule el l\'{\i}mite en la siguientes series geométricas.

\begin{enumerate}
\item $%
{\displaystyle\sum\limits_{n=1}^{\infty}}
\left(  \dfrac{2}{5}\right)  ^{n}.$

\item $%
{\displaystyle\sum\limits_{n=1}^{\infty}}
\left(  \dfrac{5}{9}\right)  ^{n}.$

\item $%
{\displaystyle\sum\limits_{n=0}^{\infty}}
\dfrac{4^{n+1}}{5^{n}}.$

\item $%
{\displaystyle\sum\limits_{n=1}^{\infty}}
\dfrac{1}{e^{2n}}.$

\item $%
{\displaystyle\sum\limits_{n=1}^{\infty}}
\dfrac{\left(  -1\right)  ^{n}}{3^{n}}$

\item $%
{\displaystyle\sum\limits_{n=1}^{\infty}}
\left(  \dfrac{-3}{4}\right)  ^{n}.$
\end{enumerate}
\end{enumerate}

\clearpage


