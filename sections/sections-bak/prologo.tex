%% This document created by Scientific Word (R) Version 3.5
%\usepackage[spanish]{babel}%
%\addtolength{\headwidth}{\marginparsep}
%\addtolength{\headwidth}{\marginparwidth}
%\renewcommand{\sectionmark}[1]{\markright{\thesection\ #1}}
%\fancyhf{yj}
%\pagestyle{fancy}
%\renewcommand{\chaptermark}[1]%
%{\markboth{\MakeUppercase{\thechapter.\ #1}}{}}

\documentclass[12pt]{book}%
\usepackage{amsbsy}
\usepackage{amsmath}
\usepackage{multicol}
\usepackage{zahlen}
\usepackage[dvips]{graphicx}
\usepackage{fancyhdr}
\usepackage[latin1]{inputenc}
\usepackage{amsfonts}
\usepackage{amssymb}
\usepackage{color}
\usepackage{titlesec}
\usepackage{titletoc}
\newcommand{\bigrule}{\titlerule[0.5mm]}
\titleformat{\chapter}[display]
{\sl\bfseries\Huge} { \sc\Large\chaptertitlename\ \\
\bfseries\LARGE \textcolor{white}{.....}\thechapter}{1ex}
{\titlerule \vspace{4mm} \filleft {\filcenter}} [\vspace{0.5mm}
\bigrule]


%
\setcounter{MaxMatrixCols}{30}
%TCIDATA{OutputFilter=latex2.dll}
%TCIDATA{Version=5.00.0.2570}
%TCIDATA{CSTFile=book.cst}
%TCIDATA{Created=Monday, August 04, 2003 08:59:00}
%TCIDATA{LastRevised=Sunday, June 13, 2004 15:48:06}
%TCIDATA{<META NAME="GraphicsSave" CONTENT="32">}
%TCIDATA{<META NAME="SaveForMode" CONTENT="1">}
%TCIDATA{<META NAME="DocumentShell" CONTENT="General\Blank Document">}
%TCIDATA{PageSetup=72,72,72,72,0}
%TCIDATA{AllPages=
%F=36,\PARA{038<p type="texpara" tag="Body Text" >\hfill \thepage}
%}
\setlength{\textheight}{20 cm} \setlength{\textwidth}{14cm}
\headheight-0.5cm \oddsidemargin+1.5 cm \evensidemargin+1.0 cm
\topmargin+1.0 cm \headsep+1.0 cm \pagestyle{fancy}
\renewcommand{\chaptermark}[1]{\markboth{\MakeUppercase {\chaptername \ \thechapter.\ #1}}{}}
\renewcommand{\sectionmark}[1]{\markright{\MakeUppercase{\thesection.\ #1}}}
\renewcommand{\headrulewidth}{0.5pt}
\renewcommand{\footrulewidth}{0.5pt}
\newcommand{\helv}{\fontfamily{phv}\fontseries{b}\fontsize{9}{11}\selectfont}
\fancyhf{} \fancyhead[LE,RO]{\helv \thepage} \fancyhead[LO]{\helv
\rightmark} \fancyhead[RE]{\helv \leftmark} \fancyfoot[LE,RO]{}
\fancyfoot[RO,LE]{\helv Casta�eda / Prato / Jim�nez  }
\fancyfoot[CO,RE]{}
\newtheorem{theorem}{Teorema}[section]
\newtheorem{algorithm}{Algoritmo}[section]
\newtheorem{corollary}{Corolario}[chapter]
\newtheorem{conjecture}[theorem]{Conjetura}
\newtheorem{lemma}{Lema}[section]
\newtheorem{proposition}{Proposici\'on}[section]
\newtheorem{axiom}{Axioma}[section]
\newtheorem{remark}{Observaci\'on}[section]
\newtheorem{example}{Ejemplo}[section]
\newtheorem{exercise}{Ejercicio}[section]
\newtheorem{definition}{Definici\'on}[section]
\newenvironment{proof}[1][Prueba]{\textbf{#1.} }{\ \rule{0.5em}{0.5em}}
\newenvironment{exam}[1][Ejercicio]{\textbf{#1.} }{\ \rule{0.5em}{0.5em}}
\newenvironment{sol}[1][Soluci�n]{\textbf{#1.} }
\clearpage
\begin{document}
\pagestyle{plain}

\chapter*{Pr�logo}
 El presente texto es, en principio, sobre problemas y
ejercicios de
 C\'alculo diferencial de funciones reales de variable real; en el futuro,
 sin embargo, pretende convertirse en un texto de C\'alculo diferencial. Por esa
 raz\'on, m\'as que una simple enumeraci\'on de teoremas y resultados b\'asicos
  que fundamenten las soluciones de los ejercicios y problemas presentados,
   incluimos el desarrollo te\'orico de los primeros cap\'{\i}tulos. Estamos completamente
    convencidos  de la necesidad, por parte del estudiante, de una fundamentaci\'on s\'olida
     en los principios te\'oricos, como
   condici\'on previa para obtener resultados cualitativos y cuantitativos \'optimos en los cursos
   de c\'alculo.
   \paragraph{}La denominaci\'on provisional del texto como ``Problemario"
   no debe llevar a confundirlo con un simple ``recetario". A trav\'es de nuestra ya larga experiencia
   en la ense\~nanza del C\'alculo, hemos sido testigos de la ``capacidad" del estudiante
    promedio-e incluso de un buen n\'umero por debajo de ese promedio-  de repetir cabalmente
    soluciones prefabricadas a ejercicios y problemas de "aplicaciones", casi de manera
     autom\'atica, sin una consciencia clara de los, por lo general, realmente pocos principios
     subyacentes en sus c\'alculos y procedimientos. Principios
     que deben ser presentados con claridad y, en lo posible,
     rigurosidad por el profesor, evitando a toda costa dar a los estudiantes la falsa y nefasta
     impresi\'on de que el C\'alculo y, en general, las
     Matem\'aticas, no son m\'as que un compendio de resultados
     inertes que deben ser memorizados y aplicados los correspondientes en una
     situaci\'on espec\'{\i}fica, de un n\'umero pr\'acticamente ilimitado posible de ellas.
     \paragraph{}Dado que el texto es sobre C\'alculo
     diferencial de {\em funciones reales de variable real}, los objetos sobre los cuales
se trabajar\'a son funciones definidas en subconjuntos del
conjunto de los n\'umeros reales y con valores o ``im\'agenes"
reales. En el primer cap\'{\i}tulo se presentan los principales
t\'opicos relativos a dichas funciones y los cuales son
importantes para el desarrollo mismo del curso. Algunas de las
definiciones que se presentan son tambi\'en v\'alidas para
funciones en general. Si bien dicho primer cap\'{\i}tulo se supone
conocido por un curso previo de {\em Algebra de reales}, lo hemos
incluido para facilitar, por una parte, el tr\'ansito hacia el
curso mismo de C\'alculo y, por otra, con fines diagn\'osticos.
\paragraph{}Prerrequisitos importantes para el curso son:
\begin{enumerate}
\item El Algebra de reales. Es decir, el conocimiento y manejo
b\'asico de la estructura de campo $(\rz,+,\cdot)$, constituida
por el conjunto de los n\'umeros reales, $\rz$, con las
operaciones de adici\'on y multiplicaci\'on. Otras propiedades,
topol\'ogicas (geom\'etricas), necesarias se introducen en
 el cap\'{\i}tulo 2. \item Trigonometr\'{\i}a b\'asica. En
particular, manejo de las razones trigonom\'etricas para \'angulos
cualesquiera y las identidades fundamentales.
\end{enumerate}
Se supone un conocimiento de tales t\'opicos gracias a un curso
previo de Algebra y Trigonometr\'{\i}a. Para consultas sobre estos
temas se recomiendan \cite{Taylor}, \cite{Vance} y \cite{leh},
especialmente. Tambi\'en textos de m\'as reciente edici\'on como
\cite{Lei} y \cite{Swo} pueden ser de utilidad.
\paragraph{}Se ha buscado presentar ilustraciones de los conceptos centrales con
 ejercicios y problemas modelos resueltos, proponiendo luego ejrcicios que buscan fomentar
 las capacidades de observaci\'on, an\'alisis, interpretaci\'on y s\'{\i}tesis de los
 estudiantes. Estamos abiertos a las sugerencias y cr\'{\i}ticas tanto de estudiantes como de
 colegas, convencidos de que a trav\'es de ellas nuestro proyecto recibir\'a la retroalimentaci\'on
  necesaria para mejorar y crecer.

  \vskip0.5cm
  Los autores.

% cambiamos el formato de los cap�tulos
% por defecto se usar�n caracteres de tama�o \Huge en negrita
% contenido de la etiqueta
% "Cap�tulo" o "Ap�ndice" en tama�o \Large en lugar de \Huge
% l�nea horizontal
% texto alineado a la derecha
% n�mero de cap�tulo en tama�o \Large
% espacio m�nimo entre etiqueta y cuerpo
% texto del cuerpo alineado a la derecha
% despu�s del cuerpo, dejar espacio vertical y trazar l�nea horizontal gruesa
%\renewcommand{\mtctitle}{Contenido del capitulo}
%\evensidemargin = 18pt
%\oddsidemargin = 18pt
% cambiamos el formato de los cap�tulos
% por defecto se usar�n caracteres de tama�o \Huge en negrita
% contenido de la etiqueta
% "Cap�tulo" o "Ap�ndice" en tama�o \Large en lugar de \Huge
% l�nea horizontal
% texto alineado a la derecha
% n�mero de cap�tulo en tama�o \Large
% espacio m�nimo entre etiqueta y cuerpo
% texto del cuerpo alineado a la derecha
% despu�s del cuerpo, dejar espacio vertical y trazar l�nea horizontal gruesa
%\renewcommand{\mtctitle}{Contenido del capitulo}
%\evensidemargin = 18pt
%\oddsidemargin = 18pt
% cambiamos el formato de los cap�tulos
% por defecto se usar�n caracteres de tama�o \Huge en negrita
% contenido de la etiqueta
% "Cap�tulo" o "Ap�ndice" en tama�o \Large en lugar de \Huge
% l�nea horizontal
% texto alineado a la derecha
% n�mero de cap�tulo en tama�o \Large
% espacio m�nimo entre etiqueta y cuerpo
% texto del cuerpo alineado a la derecha
% despu�s del cuerpo, dejar espacio vertical y trazar l�nea horizontal gruesa
%\renewcommand{\mtctitle}{Contenido del capitulo}
%\evensidemargin = 18pt
%\oddsidemargin = 18pt
% cambiamos el formato de los cap�tulos
% por defecto se usar�n caracteres de tama�o \Huge en negrita
% contenido de la etiqueta
% "Cap�tulo" o "Ap�ndice" en tama�o \Large en lugar de \Huge
% l�nea horizontal
% texto alineado a la derecha
% n�mero de cap�tulo en tama�o \Large
% espacio m�nimo entre etiqueta y cuerpo
% texto del cuerpo alineado a la derecha
% despu�s del cuerpo, dejar espacio vertical y trazar l�nea horizontal gruesa
%\renewcommand{\mtctitle}{Contenido del capitulo}
%\evensidemargin = 18pt
%\oddsidemargin = 18pt
% cambiamos el formato de los cap�tulos
% por defecto se usar�n caracteres de tama�o \Huge en negrita
% contenido de la etiqueta
% "Cap�tulo" o "Ap�ndice" en tama�o \Large en lugar de \Huge
% l�nea horizontal
% texto alineado a la derecha
% n�mero de cap�tulo en tama�o \Large
% espacio m�nimo entre etiqueta y cuerpo
% texto del cuerpo alineado a la derecha
% despu�s del cuerpo, dejar espacio vertical y trazar l�nea horizontal gruesa
%\renewcommand{\mtctitle}{Contenido del capitulo}
%\evensidemargin = 18pt
%\oddsidemargin = 18pt

%TCIDATA{OutputFilter=latex2.dll}
%TCIDATA{Version=5.00.0.2570}
%TCIDATA{LaTeXparent=0,0,Libropdf.tex}
%TCIDATA{ChildDefaults=%
%chapter:1,page:1
%}


\begin{thebibliography}{99}                                                                                               %
%\addcontentsline{toc}{chapter}{Bibliograf\'{\i}a}\cfoot[Indice de materias]{Indice de materias}

\bibitem[1]{A}\textsc{Apostol, Tom. }\textit{Calculus}, Volumen 1, 2a.
edici\'{o}n, Barcelona, Editorial Revert\'{e}, 1988.

\bibitem[2]{leh}\textsc{Lehmann, Charles.} \textit{\'{A}lgebra}, M\'{e}xico,
Limusa. 1981.

\bibitem[3]{Lei}\textsc{Leithold, Louis.} \textit{\'{A}lgebra y
Trigonometr\'{\i}a con Geometr\'{\i}a anal\'{\i}tica}, M\'{e}xico, Harla, 1994.

\bibitem[4]{Rudin}\textsc{Rudin, Walter. }\textit{Principles of Mathematical
Analysis}, 3a. edici\'{o}n, McGraw-Hill, 1987.

\bibitem[5]{S}\textsc{Spivak, Michael. }\textit{Calculus: C\'{a}lculo
infinitesimal}, Volumen 1, Barcelona, Editorial Revert\'{e}, 1987.

\bibitem[6]{St}\textsc{Stewart, James. }\textit{C\'{a}lculo: Conceptos y
contextos}, M\'{e}xico, Thomson, 2001.

\bibitem[7]{Swo}\textsc{Swokowski, Earl y Jeffrey Cole.} \textit{\'{A}lgebra y
Trigonometr\'{\i}a con Geometr\'{\i}a anal\'{\i}tica}, 10a. edici\'{o}n,
M\'{e}xico, Thomson, 2002.

\bibitem[8]{Taylor}\textsc{Taylor, Howard y Thomas Wade.}
\ \textit{Matem\'{a}ticas b\'{a}sicas}, M\'{e}xico, Limusa, 1975.

\bibitem[9]{Vance}\textsc{Vance, Elbridge.}\ \textit{\'{A}lgebra y
Trigonometr\'{\i}a}, 2a. edici\'{o}n, Fondo Educativo Iberoamericano. 1978.

\bibitem[10]{Lang}\textsc{Lang, Serge.}\textbf{ }\textit{An\'{a}lisis
matem\'{a}tico}, M\'{e}xico, Addison-Wesley, 1990.
\end{thebibliography}


\end{document}
