\contentsline {chapter}{\numberline {1}Funciones reales de variable real}{1}
\contentsline {section}{\numberline {1.1}Definiciones b\'asicas}{1}
\contentsline {paragraph}{}{2}
\contentsline {paragraph}{}{3}
\contentsline {paragraph}{}{4}
\contentsline {section}{\numberline {1.2}Gr\'aficas de funciones reales{}\tmspace +\thinmuskip {.1667em}de variable real}{4}
\contentsline {section}{\numberline {1.3}\'Algebra de funciones}{8}
\contentsline {section}{\numberline {1.4}Composici\'on de funciones}{11}
\contentsline {paragraph}{\vspace \medskipamount }{12}
\contentsline {paragraph}{}{16}
\contentsline {paragraph}{}{17}
\contentsline {section}{\numberline {1.5}Sucesiones y series de n\'umeros reales}{17}
\contentsline {paragraph}{}{18}
\contentsline {paragraph}{}{20}
\contentsline {section}{\numberline {1.6}Ejercicios resueltos}{21}
\contentsline {section}{\numberline {1.7}Ejercicios propuestos}{31}
\contentsline {chapter}{\numberline {2}L\IeC {\'\i }mite y continuidad}{47}
\contentsline {section}{\numberline {2.1}Topolog\IeC {\'\i }a de la recta real}{47}
\contentsline {section}{\numberline {2.2}L\IeC {\'\i }mite de sucesiones}{49}
\contentsline {section}{\numberline {2.3}La serie geom\'etrica}{56}
\contentsline {section}{\numberline {2.4}L\IeC {\'\i }mites de funciones reales de varia\discretionary {-}{}{}ble real}{59}
\contentsline {section}{\numberline {2.5}Teoremas de l\IeC {\'\i }mite}{65}
\contentsline {section}{\numberline {2.6}Continuidad}{68}
\contentsline {subsection}{\numberline {2.6.1}Teorema del valor intermedio}{69}
\contentsline {subsection}{\numberline {2.6.2}Teoremas sobre continuidad}{72}
\contentsline {section}{\numberline {2.7}As\IeC {\'\i }ntotas}{72}
\contentsline {section}{\numberline {2.8}Ejercicios resueltos}{74}
\contentsline {section}{\numberline {2.9}Ejercicios propuestos}{94}
\contentsline {chapter}{\numberline {3}Derivaci\'on}{109}
\contentsline {section}{\numberline {3.1}Definici\'on de derivada}{109}
\contentsline {subsection}{\numberline {3.1.1}Derivada de orden superior}{113}
\contentsline {subsection}{\numberline {3.1.2}Interpretaci\'on geom\'etrica de la derivada}{114}
\contentsline {subsection}{\numberline {3.1.3}Interpretaci\'on f\IeC {\'\i }sica de la derivada}{115}
\contentsline {section}{\numberline {3.2}Teoremas relativos a la derivada}{116}
\contentsline {section}{\numberline {3.3}Derivaci\'on impl\IeC {\'\i }cita}{118}
\contentsline {subsection}{\numberline {3.3.1}Derivada de la funci\'on inversa}{120}
\contentsline {subsection}{\numberline {3.3.2}Derivaci\'on logar\IeC {\'\i }tmica}{121}
\contentsline {section}{\numberline {3.4}Ejercicios resueltos}{122}
\contentsline {section}{\numberline {3.5}Ejercicios propuestos}{140}
\contentsline {chapter}{\numberline {4}Aplicaciones de la derivada}{155}
\contentsline {section}{\numberline {4.1}M\'aximos y m\IeC {\'\i }nimos}{155}
\contentsline {section}{\numberline {4.2}Teorema del valor medio}{159}
\contentsline {section}{\numberline {4.3}Funciones mon\'otonas}{162}
\contentsline {subsection}{\numberline {4.3.1}Funciones crecientes y decrecientes}{162}
\contentsline {subsection}{\numberline {4.3.2}Criterio de la primera derivada}{166}
\contentsline {section}{\numberline {4.4}Concavidad}{166}
\contentsline {subsection}{\numberline {4.4.1}Concavidad hacia arriba y hacia abajo}{166}
\contentsline {subsection}{\numberline {4.4.2}Criterio de la segunda derivada}{168}
\contentsline {section}{\numberline {4.5}Ejemplos resueltos}{169}
\contentsline {subsection}{\numberline {4.5.1}Ejemplos varios}{169}
\contentsline {subsection}{\numberline {4.5.2}Razones relacionadas}{177}
\contentsline {subsection}{\numberline {4.5.3}Optimizaci\'on}{181}
\contentsline {section}{\numberline {4.6}Ejercicios propuestos}{190}
\contentsline {section}{Bibliograf\'{\i }a}{207}
\contentsfinish 
