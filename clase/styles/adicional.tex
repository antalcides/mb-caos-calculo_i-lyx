
%%%%%%%%%%%%%%%%%%%%%%  Def Colores %%%%%%%%%%%%%%%%%%%%%%%%%%%%%%%%%%%%%%%%%%% 
\definecolor{bistre}{rgb}{.75,.50,.30}
\definecolor{Maroon}{rgb}{0.5,0.0,0.0}%{.80,.80,.95}
\definecolor{fondpaille}{cmyk}{0,0,0.1,0}
\pagecolor{fondpaille} 
%\color{Maroon}
\tkzSetUpColors[background=fondpaille,text=Maroon]
%%%%%%%%%%%%%%%%%%%%%%%%%---- Comandos adicionales-------------%%%%%%%%%%%%%%%%
%%%%%%%%%%%%%%%%% ------- adornos ------------------%%%%%%%%%%%%%%%%%%%%%%%%%%
\newfont{\adornos}{fourier-orns}% ornamentos
\newcommand{\volutaleft}{\adornos\mbox{\symbol{91}}} % adorno voluta
\newcommand{\volutaright}{\adornos\mbox{\symbol{92}}} % adorno voluta
\newcommand{\florleft}{\adornos\mbox{\symbol{98}}} % adorno flor
\newcommand{\florright}{\adornos\mbox{\symbol{99}}} % adorno flor
\newcommand{\manoright}{\adornos\mbox{\symbol{116}}} % adorno mano izquierda
\newcommand{\manoleft}{\adornos\mbox{\symbol{117}}} % adorno mano derecha
\newfont{\librotex}{manfnt}% figuras del libto de Knuth "The TeX Book"
\newcommand{\curvasr}{\librotex\mbox{\textcolor{red!50!black}{\symbol{126}}}} % curvas peligrosas dcha
\newcommand{\curvasl}{\librotex\mbox{\textcolor{red!50!black}{\symbol{127}}}} % curvas peligrosas dcha\symbol{127}}} % curvas peligrosas izqda
\newfont{\noesta}{psyr}
\renewcommand{\notin}{\noesta\mbox{\symbol{207}}}
%\newfont{\implicadcha}{mtsyt}
\renewcommand{\Longrightarrow}{\implicadcha\mbox{\symbol{144}}}
\renewcommand{\funcreal}[2]{\,#1\!:#2\rightarrow \R\,}
\newcommand{\func}[2]{\mbox{$\,#1:#2\rightarrow \R\,$}}
\newcommand{\abs}[1]{\lvert{#1}\rvert}  % Valor absoluto
\newcommand{\tl}{\mbox{$^{\mspace{2mu}\prime}$}}
\newcommand{\tlo}{\mbox{$^{\prime}$}}
\renewcommand{\baselinestretch}{1.1}% interlineado
\newcommand{\en}{\!\in\!}
\newcommand{\Fin}{\hfill \emph{(q.e.d)}\,\,\sqbox{red!50!black}}
\newcommand{\epos}{\ensuremath{\varepsilon>0}}
\newcommand{\eps}{\ensuremath{\varepsilon}}
\newcommand{\infinity}{\mifuente\mbox{\symbol{49}}} %Símbolo infinito
\newcommand{\sskip}{\vspace{2mm}}
\newcommand{\ms}{\mspace}
\renewcommand{\Lim}[3]{\mbox{$\displaystyle{\lim_{#2\to #3}#1}$}}
\DeclareMathOperator{\e}{e}
\newcommand{\setbig}[1]{\big\{ #1 \big\}}
\newcommand{\nN}{\ensuremath{n\!\in\!\mathbb N}}
\newcommand{\conj}[1]{\mbox{$\overline{\rule{0mm}{1.8mm}#1}$}}
\newcommand{\set}[1]{\left\lbrace #1 \right\rbrace}
\newcommand{\Rp}{\ensuremath{{\mathbb R}^{+}}}
\newcommand{\Rpo}{\ensuremath{{\mathbb R}^{+}_{\,\rm o}}}
\newcommand{\Rm}{\ensuremath{{\mathbb R}^{-}}}
\newcommand{\ff}{\ensuremath{\varphi}}
\newcommand{\vx}{\ensuremath{\mathbf{x}}}
\newcommand{\vy}{\ensuremath{\mathbf{y}}}
\newcommand{\vz}{\ensuremath{\mathbf{z}}}
\newcommand{\va}{\ensuremath{\mathbf{a}}}
\newcommand{\vb}{\ensuremath{\mathbf{b}}}
\newcommand{\vc}{\ensuremath{\mathbf{c}}}
\newcommand{\vu}{\ensuremath{\mathbf{u}}}
\newcommand{\vv}{\ensuremath{\mathbf{v}}}
\newcommand{\vw}{\ensuremath{\mathbf{w}}}
\newcommand{\vr}{\ensuremath{\mathbf{r}}}
\newcommand{\vf}{\ensuremath{\mathbf{F}}}
\newcommand{\vn}{\ensuremath{\mathbf{n}}}
\newcommand{\vt}{\ensuremath{\mathbf{t}}}
\newcommand{\norma}[1]{\left\|{#1}\right\|}
\newcommand{\modulo}[1]{\left\lvert{#1}\right\rvert}
\renewcommand{\le}{\leqslant}
\newcounter{estrategia}[chapter]
\newenvironment{estrategia}{\refstepcounter{estrategia}\textcolor{red!50!black}{Estrategia. \, \theestrategia\,}}{\\ \linea\medskip}
\newcommand{\hecho}{\noindent{~\hfill\Large\color{red}{\Smiley{}} }}% usa marvosym
 % Entorno y contador para ejercicios propuestos
        \newcounter{propuesto}
        \def\theejerciciobis{\arabic{propuesto}}
        \newenvironment{ejercicios propuestos}{{\subsection{Ejercicios
        propuestos}}%
        \vspace{-5mm}\tikz \path[draw=orange,line width=3pt] (0,1) -- (10,1) ;\newline\begin{list}{\bfseries{\thepropuesto.}}{}}
        {\end{list}\bigskip}
        \def\propuesto{\refstepcounter{propuesto}\item}

        % Entorno y contador para ejercicios resueltos
        \newcounter{resuelto}
        \def\theejercicio{\arabic{resuelto}}
        \newenvironment{ejercicios resueltos}{{\subsection{Ejercicios
        resueltos}}%
        \vspace{-5mm}\tikz \path[draw=orange, line width=3pt] (0,1) -- (1,1) -- (10,1) ;\newline
        \rule{0mm}{6mm}¡Antes de ver la soluci\'on de un ejercicio debes intentar resolverlo!\newline
       %\indent \psline[linewidth=.75pt](-.5,0)(10,0)\newline
        \begin{list}{{\bfseries Ejercicio resuelto\ {{\theresuelto}}}}{}}
        {\end{list}\bigskip}
        \def\resuelto{\refstepcounter{resuelto}\item}
        %%%%%%%%%%%%%%%%%%%%%%%%%%%%%%%%%%%%%%%%%%%%%%%%%%%%%%%%%%%%%%%%%%%%%%%
        \newcommand\MarginFig[4][width=\marginparwidth]{%
\marginpar{\includegraphics[#1]{#2}
\captionof{figure}{#3}
\label{#4}}
}
\newcommandx{\notamargen}[2][1=]{\todo[linecolor=blue!50!black,backgroundcolor=yellow!20!white,bordercolor=blue,#1]{#2}}
\reversemarginpar