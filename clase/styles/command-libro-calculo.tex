\theoremstyle{plain}% default
\newtcbtheorem[number within=section]{definicion}%
 {\textsc{Definici\'on}}{theorem style=plain, sharp
corners,enhanced,colframe=blue!50!black,colback=yellow!20!white,coltitle=red!50!black,fonttitle=\upshape\bfseries\large,fontupper=\itshape,
drop lifted shadow=blue!70!black!50!white,boxrule=0.4pt}{definicion}
%
%
%fontupper=\itshape\large
%
\newtcbtheorem[use counter from=definicion]{teorema}%
 {\textsc{Teorema}}{theorem style=plain,sharp
corners, enhanced,colframe=blue!50!black,colback=yellow!20!white,
coltitle=red!50!black,fonttitle=\upshape\bfseries\large,fontupper=\itshape,
drop lifted shadow=blue!70!black!50!white,boxrule=0.4pt}{teorema}
%
%
%
%
\newtcbtheorem[use counter from=definicion]{lema}%
  {\textsc{Lema}}{theorem style=plain, sharp
corners,enhanced,colframe=blue!50!black,colback=yellow!20!white,
coltitle=red!50!black,fonttitle=\upshape\bfseries\large,fontupper=\itshape,drop lifted shadow=blue!70!black!50!white,boxrule=0.4pt}{lema}
%
%
%
%
\newtcbtheorem[use counter from=definicion]{corolario}%
  {\textsc{Colorario}}{theorem style=plain,sharp
corners, enhanced,colframe=blue!50!black,colback=yellow!20!white,
coltitle=red!50!black,fonttitle=\upshape\bfseries\large,fontupper=\itshape,
drop lifted shadow=blue!70!black!50!white,boxrule=0.4pt}{colorario}
%
%
%
%
%
\newtcbtheorem[use counter from=definicion]{proposicion}%
  {\textsc{Proposici\'on}}{theorem style=plain,sharp
corners, enhanced,colframe=blue!50!black,colback=yellow!20!white,
coltitle=red!50!black,fonttitle=\upshape\bfseries\large,fontupper=\itshape,
drop lifted shadow=blue!70!black!50!white,boxrule=0.4pt}{proposicion}
%
%
%
\newtcbtheorem[number within=chapter]
{observacion}%
{\textsc{Observaci\'on}}{theorem style=plain,sharp
corners, enhanced,colframe=blue!50!black,colback=blue!5!white,
coltitle=red!50!black,fonttitle=\upshape\bfseries\normalsize,fontupper=\itshape,
drop lifted shadow=blue!70!black!50!white,boxrule=0.6pt}{observacion}
%
%
 % 
\newtcbtheorem[auto counter,number within=section]{notacion}%
  {\textsc{Notaci\'on}}{fonttitle=\bfseries\upshape\large, fontupper=\slshape,
     arc=0mm, colback=blue!0!white,colframe=blue!0!white}{notacion}
%
%
%
\theoremstyle{definition}
\newtheorem{Ejemplo}{Ejemplo}[section]
%---------------------------------------------------------------
%{\begin{pspicture}(0.62,0)(2.6,0.4)
%\psline[linecolor=red!50!black, linewidth=1pt](0.6,-0.1)(3.8,-0.1)
%\psline[linecolor=red!50!black, linewidth=1pt](0.6,-0.1)(0.6,0.4)
%\rput(1.7,0.1){{\large\textsc{Ejemplo}}}
%\end{pspicture}}
%
%---------------------------------------------------------------
%\theoremstyle{plain}
\newtheorem{ejer}{Ejercicios
\begin{pspicture}(0,0)(0,0)
%\psgrid
\psline[linecolor=red!50!black, linewidth=1pt](-2.1,-0.2)(1.5,-0.2)
\psline[linecolor=red!50!black, linewidth=1pt](-2.1,-0.2)(-2.1,0.5)
\psline[linecolor=red!50!black, linewidth=1pt](1.5,-0.2)(1.5,0.5)
\psline[linecolor=red!50!black, linewidth=1pt](-2.1,0.5)(1.5,0.5)
\end{pspicture}
}[section]

%\hspace{-0.67cm}
%
\newcommand{\solucion}{ \textcolor{red!50!black}{ \textsc{\bf Soluci\'on: }}}
%
%\newcommand{\acc}[1]{
%{\begin{pspicture}(0,0)(2.5,0.5)
%%\psgrid
%%\psline[linecolor=Turquoise, linewidth=2pt](0,0)(2.5,0)
%%\psline[linecolor=Turquoise, linewidth=2pt](0,0)(0,0.5)
%\rput(1.5,0.2){\Large \textcolor{MidnightBlue}{Actividad #1}}
%\end{pspicture}}}
%
%
\def\QEDmark{\ensuremath{\square}}
%
\def\proof{\paragraph{\textcolor{red!50!black}{ \textsc{\bf Soluci\'on.}}}}
\def\endproof{\hfill\color{red!50!black}$\blacksquare$}
%
%%%%%%%%%%%%%%%%%%%%%%%%%%%%% para la demostración %%%%%%%%%%%%%%%%%%%%%
\newcommand{\demostracion}{ \textcolor{red!50!black}{\hspace{-0.67cm}\textsc{ \bf Demostraci\'{o}n.}\,}}
%
%
%%%%%%%%%%%%%% MALLA %%%%%%%%
  \newpsobject{malla}{psgrid}{subgriddiv=1,griddots=10,gridlabels=6pt}
\allowdisplaybreaks
\setlength{\parindent}{0pt}
%
%
%%%%%%%%%%%%%%%%%%%%%%%%%%%%% DERIVAR  %%%%%%%%%%%%%%%%%%%%%
%\providecommand{\derive}[2]{\frac{d }{ d  #2}\left[#1\right]}
%
\providecommand{\derive}[2]{\dfrac{d #1 }{ d  #2}}
%
\providecommand{\derivee}[2]{\dfrac{d^2#1 }{ d  #2}}
%%%%%%%%%%%%%%%%%%%%%%%%%%%%%%%%%%%%%%%%%%%%%%%%%%%%%%%%%%%%%%%%%%%
\newcommand{\Co}{\mathbb C}
\newcommand{\F}{\mathbb F}
\newcommand{\J}{\mathbb J}
\newcommand{\K}{\mathbb K}
\newcommand{\N}{\mathbb N}
\newcommand{\Po}{\mathbb P}
\newcommand{\Q}{\mathbb Q}
\newcommand{\R}{\mathbb R}
\newcommand{\Z}{\mathbb Z}
%
%%%%%%%%%%%%%%%%%%%%%%%%%%%%%%%%%%%%%%%%%%%%%%
%FLECHAS
\newcommand{\infinitot}{t\to\infty}  % t tiende hacia infinito
\newcommand{\infiniton}{n\to\infty}  % n tiende hacia infinito
\newcommand{\hacia}{\longrightarrow}
\newcommand{\ssi}{\longleftrightarrow}
\newcommand{\Ssi}{\Longleftrightarrow}
%%%%%%%%%%%%%%%%%%%%%%%%%%%%%%%%%%%%%%%%%%%%%%%%%%%%%%%%%%%%%%%%%%%%%%%%%%%%%%%
\newcommand{\circulo}{\marginpar{\vspace{0.3cm}\hspace{-16.7cm}$\circledS$}{}}%LIMITES
\newcommand{\nlim}{\lim\limits_{n\to\infty}}
\newcommand{\tlim}{\lim\limits_{t\to\infty}}
\newcommand{\jlim}{\lim\limits_{j\to\infty}}
\newcommand{\klim}{\lim\limits_{k\to\infty}}
\newcommand{\mlim}{\lim\limits_{m\to\infty}}
\newcommand{\rlim}{\lim\limits_{r\to\infty}}
\newcommand{\Btag}{\tag*{$\Box$}}
%%%%%%%%%%%%%%%%%%%%%%%%%%%%%%%%%%%%%%%%%%%%%%%%%%%%%%%%%%%%%%%%%%%%%%%%%%%%%%%%
%SUMAS
\newcommand{\Sumai}{\sum\limits_{i=1}^n}     %Suma desde i=1 hasta n (con el \limits)
\newcommand{\sumai}{\sum_{i=1}^n}             %Suma desde i=1 hasta n (sin el \limits)
\newcommand{\Sumaj}{\sum\limits_{j=0}^{n-1}}      %Suma desde j=0 hasta n-1
\newcommand{\Suman}{\sum\limits_{n=1}^\infty}   %Suma desde n=1 hasta  infinito
\newcommand{\suman}{\sum\limits_{n=0}^\infty}   %Suma desde n=0 hasta  infinito
\newcommand{\jSuma}{\sum\limits_{j=1}^\infty}   %Suma desde j=1 hasta  infinito
\newcommand{\jsuma}{\sum_{j=1}^\infty}   %Suma desde j=1 hasta  infinito
\newcommand{\Sumak}{\sum\limits_{k=0}^\infty}     %Suma desde k=0 hasta infinito (con el \limits)
\newcommand{\sumak}{\sum\limits_{k=1}^\infty}     %Suma desde k=1 hasta infinito (con el \limits)
%%%%%%%%%%%%%%%%%%%%%%%%%%%%%%%%%
\newcommand{\dis}{\displaystyle}
%\newcommand{\Int}{\displaystyle\int}
%\newcommand{\rig}{\rightarrow}
%\newcommand{\lef}{\leftarrow}
%\newcommand{\Rig}{\Rightarrow}
%
%%%%%% Funciones trigonometrica e hiperbolicas %%%%%%%%%
\newcommand{\sen}{\operatorname{\sen}}
%\newcommand{\cos}{\operatorname{\cos}}
\newcommand{\arcsec}{\mathop{\rm arcsec}\nolimits}
\newcommand{\arcsen}{\mathop{\rm arcsen}\nolimits}
\newcommand{\arccot}{\mathop{\rm arccot}\nolimits}
\newcommand{\arccsc}{\mathop{\rm arccsc}\nolimits}
\newcommand{\senh}{\mathop{\rm senh}\nolimits}
\newcommand{\secanteh}{\mathop{\rm sech}\nolimits}
\newcommand{\cosecanteh}{\mathop{\rm csch}\nolimits}
%
\renewcommand{\partname}{Semana}
%\newtheorem{proof}{Remark}
%\renewcommand*{\proofname}{Solution}
%
\def\@sqrt[#1]{\root #1\of}
%
%\renewcommand{\rmdefault}{phv}
%\renewcommand{\sfdefault}{phv}
\normalfont
%%%%%%%%%%%%%%%% cajas espciales%%%%%%%%%%%%%%%%%%
%
%%%%%%%%%%%%%%%%%%%% Otros comandos %%%%%%%%%%%%%%%%%%%%%%%%%%%%%%%
%%%%%%%%%%%%%%% ------ colores ---------------%%%%%%%%%%%%%%%%%%%%%
\definecolor{problemblue}{RGB}{100,134,158}
\definecolor{titlebgdark}{RGB}{10,76,115}
\definecolor{ocre}{RGB}{10,76,115}
\definecolor{ptctitle}{RGB}{10,76,115}
\definecolor{ptcbackground}{RGB}{212,237,252}
\definecolor{titlebglight}{RGB}{191,233,251}
%%%%%%%%%%%%%%%%%%%%%%%%%%%%%%%%%%%%%%%%%%%%%%%%%%%%%
\newcommand\peque{\@setfontsize\peque{8}{9}}
\newcommand{\remark}{\colorbox{titlebgdark}{\color{white}{Nota:}} }
\newcommand{\prop}{\colorbox{titlebgdark}{\color{white}{Proposición:}} }
\newcommand{\dem}{\colorbox{titlebgdark}{\color{white}{Demostraci\'on:}} }
\newcommand{\notacionn}{\colorbox{titlebgdark}{\color{white}{Notaci\'on:}} }
\newcommand{\sol}{\colorbox{titlebgdark}{\color{white}{\noun{Soluci\'on}:}}\ }
\newcommand{\resp}{\colorbox{titlebgdark}{\color{white}{\noun{Respuesta}:}}\ }
\newcommand{\general}{\colorbox{titlebgdark}{\color{white}{\noun{En general}:}} }
\newcommand{\obs}{\colorbox{titlebgdark}{\color{white}{\noun{Observaci\'on}:}} }
\newcommand{\intro}{\colorbox{titlebgdark}{\color{white}{\large\noun{Introducci\'on}:}} }
\newcommand{\conclu}{\colorbox{titlebgdark}{\color{white}{\noun{Conclusi\'on}:}} }
\newcommand{\resu}{\colorbox{titlebgdark}{\color{white}{\noun{Resumen}:}} }
\newcommand{\expli}{\colorbox{titlebgdark}{\color{white}{\noun{Explicaci\'on}:}}\ }
\newcommand{\ej}{\colorbox{titlebgdark}{\color{white}{\noun{Ejemplos}:}} }
\newcommand{\acc}[1]{\fcolorbox{MidnightBlue}{white}{\color{black}{\noun{Actividad}:\, #1}} }
\newcommand{\cuadro}[2]{\colorbox{#1}{\color{white}{\noun{#2}:}} }
\newcommand{\fin}{\hfill\boxempty}
%%%%%%%%%%%%%%%%%%%%%%%%%%%%%%%%%%%%%%%%%%%%%%%%%%%%%%%%%%%%%%%
%%%%%%%%%%%%%%%%%%%%%%%%%%% cajas para remark %%%%%%%%%%%%%%%%%%%%%%%%%%5
%%%%%%%%%%%%%%%%%%%%%%%%%%%%%%%%%%%
\definecolor{boxbg}{RGB}{179,222,255}
\tcbset{
  common/.style={
    before=\vskip2\baselineskip\noindent,
    after=\vskip2\baselineskip,
    enhanced,
    colback=boxbg,
    frame code={},
    fontupper=\normalsize,
  }
}
%
%%%%%%%%%%%%%%%%%%%%%%%%%%%%%%%%%%%%%%%%%%%%%%%%%%%%%
%
\newtcolorbox{ideabox}{
common,
interior code={
  \filldraw[ultra thick,densely dashed,fill=boxbg,draw=black,rounded corners=10pt] (interior.north west) rectangle (interior.south east);
  \node at  ([xshift=-20pt,yshift=8pt]interior.north east) {\includegraphics[width=1.5cm,angle=-30]{lightbulb}};
  }
\vspace*{40pt}}
%
%%%%%%%%%%%%%%%%%%%%%%%%%%%%%%%%%%%%%%%%%%%%%%%%%%%%
%
\newtcolorbox{questionbox}{
common,
interior code={
  \filldraw[ultra thick,densely dashed,fill=boxbg,draw=black,rounded corners=10pt] (interior.north west) rectangle (interior.south east);
  \node at  ([xshift=-20pt,yshift=8pt]interior.north east) {\includegraphics[width=1.5cm,angle=-30]{questionmark}};
  }
\vspace*{40pt}}
%
%%%%%%%%%%%%%%%%%%%%%%%%%%%%%%%%%%%%%%%%%%%%%%%%%%%%%%
%
\newtcolorbox{apunte}{
common,
interior code={
  \filldraw[ultra thick,densely dashed,fill=boxbg,draw=black,rounded corners=10pt] (interior.north west) rectangle (interior.south east);
  \node at  ([xshift=-30pt,yshift=8pt]interior.north east) {\includegraphics[width=1.5cm]{apuntes}};
  }
\vspace*{20pt}}
%
%%%%%%%%%%%%%%%%%%%%%%%%%%%%%%%%%%%%%%%%%
\newtcolorbox{analiza}{
common,
interior code={
  \filldraw[ultra thick,densely dashed,fill=boxbg,draw=black,rounded corners=10pt] (interior.north west) rectangle (interior.south east);
  \node at  ([xshift=-30pt,yshift=10pt]interior.north east) {\includegraphics[width=1.5cm]{image/warning}};
  }
\vspace*{20pt}}
%
%%%%%%%%%%%%%%%%%%%%%%%%%%%%%%%%%%%%%%%%
%
%%%%%%%%%%%%%%%%%%%%%%%%%%%%%%%%%%%%%%%%%%%%%%%%
\newtcolorbox{piensa}{
common,
interior code={
  \filldraw[ultra thick,densely dashed,fill=boxbg,draw=black,rounded corners=10pt] (interior.north west) rectangle (interior.south east);
  \node at  ([xshift=-30pt,yshift=10pt]interior.north east) {\includegraphics[width=1.5cm]{image/tipp}};
  }
\vspace*{20pt}}
%%%%%%%%%%%%%%%%%%%%%%%%%%%%%%%%%%%%%%%%%%%%%%%%%%
\newtcolorbox{resumen}{
common,
interior code={
  \filldraw[ultra thick,densely dashed,fill=boxbg,draw=black,rounded corners=10pt] (interior.north west) rectangle (interior.south east);
  \node at  ([xshift=-30pt,yshift=8pt]interior.north east) {\includegraphics[angle=-30,width=1.5cm]{image/resumen}};
  }
\vspace*{20pt}}
%%%%%%%%%%%%%%%%%%%%%%%%%%%%%%%%%%%%%%%%%%%%%%%%%%%%%%%%%%%%%%%%%%%%%%%
%%%%%%%%%%%%%%%%%%%%%%%%%%%%%%%%%%%%notas para recordar algo%%%%%%%%%%%%%%
\definecolor{paper}{RGB}{239,227,157}
\usetikzlibrary{decorations.pathmorphing}
\newenvironment{notax}[1]{
\begin{tikzpicture}[pencildraw/.style={ %
    decorate,
    decoration={random steps,segment length=2pt,amplitude=1pt}
    } %
]
\node[
preaction={fill=black,opacity=.5,transform canvas={xshift=1mm,yshift=-1mm}},
pencildraw,draw,fill=paper,text width=.8\textwidth,inner sep=5mm] 
{#1};
\end{tikzpicture}
}{\vskip 20pt}
%%%%%%%%%%%%%%%%%%%%%%%%%%%%%%%%%%%%%%%%%%%%%%%%%%%%
\usetikzlibrary{arrows,shadows} 
\newtcolorbox{nota}{%
    enhanced jigsaw, breakable, % allow page breaks
    frame hidden, % hide the default frame
    overlay={%
        \draw [
            fill=ptcbackground, % fill paper
            draw=yellow!20!white, % boundary colour
            decorate, % decoration
            decoration={random steps,segment length=2pt,amplitude=1pt},
            drop shadow, % shadow
        ]
        % top line
        (frame.north west)--(frame.north east)--
        % right line
        (frame.north east)--(frame.south east)--
        % bottom line
        (frame.south east)--(frame.south west)--
        % left line
        (frame.south west)--(frame.north west);
    },
    % paragraph skips obeyed within tcolorbox
    parbox=false,
}
%%%%%%%%%%%%%%%%%%%%%%%%%%%%%%%%%%%%%%%%%%%%%%%%%%%%%%%
%%%%%%%%%%%%%%%%%%%%%%%%%%%%%%%%%%%%%%%%%%%%%%%%%%%
%
\newcounter{praproblem}[chapter]
\renewcommand\thepraproblem{\thesection.\arabic{praproblem}} 
\newtcolorbox{praproblem}{
  before=\bigskip\centering,
    after=\bigskip,
  breakable,
  enhanced,
  colback=white,
  boxrule=0pt,
  arc=0pt,
  outer arc=0pt,
  fontupper=\small,
  title=Problemas para practicar~\thepraproblem,
  fonttitle=\bfseries\sffamily\large\strut,
  coltitle=problemblue,
  colbacktitle=problemblue,
  title style={
    left color=orange!60,
    right color=white,
    middle color=white
  },
  overlay={
    \draw[line width=1.5pt,problemblue] (title.north west) -- (title.north east);
    \draw[line width=1.5pt,problemblue] (frame.south west) -- (frame.south east);
  }
}
\BeforeBeginEnvironment{praproblem}{\refstepcounter{praproblem}}
%
%%%%%%%%%%5
%
\newenvironment{tproblem}[1]{
     
     \begin{praproblem}
     \begin{multicols}{2}
     #1
     \end{multicols}
     }{\end{praproblem}}
%
%%%%%%%%%%%%%%%%%%%%%%%%%%%%%%%%%%%%%%%%%%
%%%%%%%%%%%%%%%%%%%%%%%%%%%%%%%%%%%%%%%%%%%%%%%%%%%%%%
%
\newtcolorbox[auto counter,number within=section]{desafio}{
  breakable,
  enhanced,
  colback=white,
  boxrule=0pt,
  arc=0pt,
  outer arc=0pt,
  title=\textcolor{white}{Problemas desafiantes:~\thetcbcounter,}
  fonttitle=\bfseries\sffamily\large\strut,
  coltitle=problemblue,
  colbacktitle=problemblue,
%  title style={
%  %exercisebgblue
%  interior style={fill=idiomsgreen}
%  },
  overlay={
    \draw[line width=1.5pt,problemblue] (frame.south west) -- (frame.south east);
  }
}

%
%%%%%
%%%%%%%%%%%%%%%%%%%%%%%%%%%%%%%%%%%%%%%%%%%%%%%%%%%%%%%%%%%%%%%%%%%%%%%%%%%%%%%%%%%%%
%%%%%%%%%%%%%%%%%%%%%%%%%%%% problemas %%%%%%%%%%%%%%%%%%%%%%%%%%%%%%%%%%%
\newcommand{\inline}{\refstepcounter{equation}~~\mbox{\color{blue!50!black}(\theequation)}}%enumera eq inline
%%%%%%%%%%%%%%%%%%%%%%% cambiar margenes %%%%%%%%%%%%%%%%%%%%%%%%
\def\ifta{0}\def\iftb{0}%
\def\txa#1{\ifthenelse{\equal{\ifta}{1}}{\typeout{#1}}{}}%%
\def\settext#1#2#3#4#5{%
  \txa{\string\textheight: #1}\txa{\string\textwidth: #2}%
  \global\setlength{\textheight}{#1}%
  \global\setlength{\textwidth}{#2}%
  \global\setlength{\evensidemargin}{#3}%
  \global\setlength{\oddsidemargin}{#4}%
  \global\setlength{\marginparwidth}{#5}%
  \@change@text%
                       }
  \def\@change@text{%
  \global\setlength{\@colht}{\textheight}%
  \txa{\string\@colht:\the\@colht}%
  \global\setlength{\@colroom}{\textheight}%
  \global\setlength{\vsize}{\textheight}%
  \global\setlength{\columnwidth}{\textwidth}%
  \if@twocolumn%
    \advance\columnwidth-\columnsep \divide\columnwidth\tw@%
    \@firstcolumntrue%
  \fi%
  \global\setlength{\hsize}{\columnwidth}%
  \global\setlength{\linewidth}{\hsize}%
}
%%%%%%%%%%%%%%%%%%%%%%%%%%%%%%%%%%%%%%%%%%%%%%%%%%%%%%%%%
\newenvironment{changemargin}[5]
{
\begin{list}{}
{
\global\setlength{\textheight}{#1}%
  \global\setlength{\textwidth}{#2}
\setlength{\topsep}{0pt}
\setlength{\evensidemargin}{0pt}%
\setlength{\oddsidemargin}{0pt}
\setlength{\leftmargin}{#3}%
\setlength{\rightmargin}{#4}%
\setlength{\listparindent}{\parindent}%
\setlength{\itemindent}{\parindent}%
\setlength{\parsep}{\parskip}%
\hoffset #5
}
\item[]
}
{\end{list}}
%
%%%%%%%%%cambiamargen%%%%%%%%%%%%%%%
%
\newenvironment{cambiamargen}[5]
{
\begin{list}{}
{
\global\setlength{\textheight}{#1}%
 \setlength{\topmargin}{#2}
\setlength{\evensidemargin}{0pt}%
\setlength{\oddsidemargin}{0pt}
\setlength{\leftmargin}{-}%
\setlength{\rightmargin}{#4}%
\setlength{\listparindent}{\parindent}%
\setlength{\itemindent}{\parindent}%
\setlength{\parsep}{\parskip}%
\hoffset #5
}
\item[]
}
{\end{list}}
%
%%%%%%%%%%%%%%%%%%%%-------------------------%%%%%%%%%%%%%%%%%%%%%%%%%%%%%
\newcommand{\problemas}[1]
{
\section{Ejercicios propuestos}
\small
\begin{changemargin}{23.2cm}{18cm}{0cm}{0cm}{0cm}
%\begin{multicols}{2}
 \noindent #1
%\end{multicols}
{\setlength{\parindent}{0mm}\color{blue!50!black}\rule{\linewidth}{1mm}}
\end{changemargin}
}
%%%%%%%%%%%%%%%%%%%%%%%%%%%%%%%%%%%%%%%%%%%%%%%%%%%%%%%%%%%%%%%%%%%%%%%%%%%%
%%%%%%%%%%%%%%%%%%%%%%%%solucion%%%%%%%%%%%%%%%%%%
\newcounter{solucion}[chapter]
%\renewcommand\theexample{\thesection.\arabic{example}}

\tcbset{solucionbox/.style={%
 title={Soluci\'on},
 breakable,
 leftrule=0pt,
 arc=0pt,
 colback=white,
 colframe=titlebgdark,
 enhanced,
 colbacktitle=white,
 coltitle=blue!50!black,
 titlerule=0pt,
 enlarge left by=-4mm,
 width=\linewidth+4mm,
 enlarge top by=2pt,
 overlay unbroken={%
 \draw[titlebgdark,line width=2pt] (frame.north west)++(0,0.25mm) --++(4cm,0pt) ;
                      \draw[white,line width=10mm] (frame.south west) --++(0cm,0pt) node (P) {};
                      \fill[titlebgdark] (P) rectangle ++(6pt,6pt) ;
                      },%
                      %%%%%%%%%%%%%%%
 overlay first={
 \draw[titlebgdark,line width=2pt] (frame.north west)++(0,1pt) --++(4cm,0pt);
 },%
                  %%%%%%%%%%%%%%%%%%%%%%%5
 overlay last={
 \draw[white,line width=10mm] (frame.south west) --++(8cm,0pt) node (P) {};
                       \fill[titlebgdark] (P) rectangle ++(6pt,6pt) ;},%
 }%
}

\newenvironment{solucionn}{%
  \tcolorbox[solucionbox]}%
 {\endtcolorbox}
 %%%%%%%%%%%%%%%%%%%%%%%%%%%%%%%%%%%%%%%%%%%%%%%%%%%%%%%%%%%%%%%%%%%%%%%%%%%%%%%%%%%%%%%%%%%%%%%%
 %%%%%%%%%%%%%%%%%%%%%%%%prueba%%%%%%%%%%%%%%%%%%
\newcounter{prueba}[chapter]
%\renewcommand\theexample{\thesection.\arabic{example}}

\tcbset{pruebabox/.style={%
 title={Prueba},
 breakable,
 leftrule=0pt,
 arc=0pt,
 colback=white,
 colframe=cyan,
 enhanced,
 colbacktitle=white,
 coltitle=cyan,
 titlerule=0pt,
 enlarge left by=-4mm,
 width=\linewidth+4mm,
 enlarge top by=2pt,
 overlay unbroken={\draw[cyan,line width=2pt] (frame.north west)++(0,0.25mm) --++(4cm,0pt);
                      \draw[white,line width=10mm] (frame.south west) --++(8cm,0pt) node (P) {};
                      \fill[cyan] (P) rectangle ++(6pt,6pt) ;},%
 overlay first={\draw[cyan,line width=2pt] (frame.north west)++(0,1pt) --++(4cm,0pt);},%
 overlay last={\draw[white,line width=10mm] (frame.south west) --++(8cm,0pt) node (P) {};
                       \fill[cyan] (P) rectangle ++(6pt,6pt) ;},%
 }%
}

\newenvironment{prueba}{%
  \tcolorbox[pruebabox]}%
 {\hfill \textcolor{ptctitle}{$\blacksquare$}(q.e.d)\endtcolorbox}
 %%%%%%%%%%%%%%%%%%%%%%%%%%%%%%%%%%%%%%%%%%%%%%%%%%%%%%%%%%%%%%%%%%%%%%%%%%%%%%%%%%%%%%%%%%%%%%%%
 \tcbset{demsbox/.style={%
 title={\bf Demostraci\'on:},
 breakable,
 leftrule=0pt,
 arc=0pt,
 colback=white,
 colframe=red!50!black,
 enhanced,
 colbacktitle=white,
 coltitle=red!50!black,
 titlerule=0pt,
 enlarge left by=-4mm,
 width=\linewidth+4mm,
 enlarge top by=2pt,
 overlay unbroken={\draw[red!50!black,line width=2pt] (frame.north west)++(0,0.25mm) --++(4cm,0pt);
                      \draw[white,line width=10mm] (frame.south west) --++(8cm,0pt) node (P) {};
                      \fill[red!50!black] (P) rectangle ++(6pt,6pt) ;},%
 overlay first={\draw[red!50!black,line width=2pt] (frame.north west)++(0,1pt) --++(4cm,0pt);},%
 overlay last={\draw[white,line width=10mm] (frame.south west) --++(8cm,0pt) node (P) {};
                       \fill[red!50!black] (P) rectangle ++(6pt,6pt) ;},%
 }%
}

\newenvironment{dems}{%
  \tcolorbox[demsbox]}%
 {\hfill \textcolor{ptctitle}{$\blacksquare$}(q.e.d)\endtcolorbox}
 %
 %%%%%%%%%%%%%%%%%%%%%%%%%%%%%%%%%%%%%%%%%%%%%%%%%%%%%%%%%%%%%%%%%%%%%%%%\\
 \newenvironment{solver}{\hspace*{-30pt}
    \colorbox{blue!50!black}{%
    \parbox[c][16pt][c]{50pt}{%
    \centering\textcolor{white}{\SectionFont\rmfamily Soluci\'on}}}
    \color{blue!50!black}{\rule{\dimexpr\textwidth-5pt-2\fboxsep\relax}{2pt}}\\[10pt]
       \color{black}
       }{\\[10pt] \color{blue!50!black}{\rule{\dimexpr \textwidth+30pt\relax}{2pt}}\\[20pt]}
      %%%%%%%%%%%%%%%%%%%%%%%%%%%%%%%%%%%%%%%%%%%%%%%%%%%%%%%%%%%%%%%%%%%%%%%%%%%%%
      % normal box
\newcommand{\sqboxs}{1.2ex}% the square size
\newcommand{\sqboxf}{0.6pt}% the border in \sqboxEmpty
\newcommand{\sqbox}[1]{\textcolor{#1}{\rule{\sqboxs}{\sqboxs}}}
% empty box
\newcommand{\sqboxEmpty}[1]{%
  \begingroup
  \setlength{\fboxrule}{\sqboxf}%
  \setlength{\fboxsep}{-\fboxrule}%
  \textcolor{#1}{\fbox{\rule{0pt}{\sqboxs}\rule{\sqboxs}{0pt}}}%
  \endgroup
}

      \newenvironment{probar}{
      \hspace*{-30pt}\colorbox{blue!50!black}{%
    \parbox[c][16pt][c]{80pt}{%
      \centering\textcolor{white}{\SectionFont\rmfamily Demostraci\'on}}}
      \vspace{-1.2\baselineskip}
       \color{blue!50!black}{\rule{\dimexpr\textwidth-5pt-6\fboxsep\relax}{2pt}}
        \\[10pt]     
       \noindent \color{black}\\[10pt]
      }{ \hfill \sqboxEmpty{red!50!black}            
       \color{blue!50!black}{\rule{\dimexpr \textwidth+30pt\relax}{2pt}}\\[20pt]}
